\documentclass[a4paper, 12pt]{article}

\usepackage[italian]{babel}
\usepackage[version=3]{mhchem}
\usepackage{textgreek}
\usepackage{gensymb}
\usepackage{chemfig}
\usepackage{tikz}
\usepackage{fullpage}
\usetikzlibrary{shapes,arrows}

\tikzstyle{block} = [rectangle, draw, text centered, minimum height=1.5em]
\tikzstyle{line} = [draw, -latex']
\tikzstyle{reverseline} = [draw, latex'-]
\tikzstyle{thickline} = [draw, -latex', line width = 0.4mm]

\usepackage[utf8]{inputenc}

\newcommand*\derivesubmol[4]{%
    \saveexpandmode\saveexploremode\expandarg\exploregroups
    \csname @\ifcat\relax\noexpand#2first\else second\fi oftwo\endcsname
        {\expandafter\StrSubstitute\@car#2\@nil}
        {\expandafter\StrSubstitute\csname CF@@#2\endcsname}
    {\@empty#3}{\@empty#4}[\temp@]%
    \csname @\ifcat\relax\noexpand#1first\else second\fi oftwo\endcsname
        {\expandafter\let\@car#1\@nil}
        {\expandafter\let\csname CF@@#1\endcsname}\temp@
    \restoreexpandmode\restoreexploremode
}

\setcrambond{2pt}{}{}

\definesubmol{adenina}{N*5([::-18]-*6(-N=-N=(-NH_2)-)=-N=-)}

\definesubmol{s1}{-[2]!{adenina}}
\definesubmol{s2}{-[6,.6]O-PO_3^{2-}}
\definesubmol{s3}{-[6,.6]OH}

\definesubmol{ribosio}{%
    -[::-90,2]%
        (%
            -[::115,1.176]O%
            -[::-50,1.176]%
        )%
    <[::45,0.8]%
        (%
          (!{s3})
          -[::45,,,,line width=2pt,shorten <=-.5pt,shorten >=-.5pt]%
              (!{s2})%
          >[::45,0.8]%
              (!{s1})%
         )%
}

\def \CoA {\tiny\chemfig{HS-CH_2-CH_2-NH-C(=[:90]O)-CH_2-CH_2-NH-C(=[:90]O)-CH(-[:270]OH)-C(-[:90]CH_3)(-[:270]CH_3)-CH_2-O-PO_2^{-}-O-PO_2^{-}-O-CH_2!{ribosio}}}

\def \ACP {\tiny\chemfig{HS-CH_2-CH_2-NH-C(=[:90]O)-CH_2-CH_2-NH-C(=[:90]O)-CH(-[:270]OH)-C(-[:90]CH_3)(-[:270]CH_3)-CH_2-O-PO_2^{-}-O-PO_2^{-}-O-CH_2-CH(-[:90]C(-[:90]...)=O)(-[:270]NH(-[:270]...))}}

\date{}

\setatomsep{15pt}

\title{%
  Vitamina B5 \\
  \large Funzioni biologiche}

\begin{document}

\begin{titlepage}

\maketitle

\tableofcontents

\section{Introduzione}
La quinta vitamina del gruppo B, corrispondente all'acido pantonenico, ricopre numerose funzioni biologiche di fondamentale importanza in tutti gli esseri viventi.\\
Essa espleta il suo ruolo principale in quanto componente essenziale del \textbf{coenzima A}, molecola centrale del metabolismo.\\
Rientra inoltre nella \textbf{proteina di trasporto di acili (ACP)}, componente importante nell'anabolismo degli acidi grassi.

\end{titlepage}

\section{Coenzima A}
\begin{center}\CoA\end{center}
Il coenzima A è coinvolto in reazioni di trasferimento di gruppi acetilici e acilici, relativi al metabolismo ossidativo e al catabolismo.\\
Grazie al dominio adenosinico, CoA è in grado di legarsi agli enzimi che lo richiedono, metre quello fosfopanteinico agisce nel legame dei substrati e nel loro spostamento da un centro catalitico all'altro.\\
Il legame fra un gruppo acilico od acetilico, e quello tiolico della fosfopanteina, porta alla formazione di un tioestere, rispettivamente \textbf{acil-CoA} o \textbf{acetil-CoA}. Essi sono \textbf{composti ad alta energia}, a causa della natura instabile del legame tioestereo, e possono quindi partecipare a numerose reazioni biochimiche.

\subsection{Metabolismo dei carboidrati}
\subsubsection{Anabolismo}
CoA ha un ruolo fondamentale nel ciclo degli acidi tricarbossilici. Esso infatti partecipa a:
\begin{itemize}
\item \textbf{decarbossilazione del piruvato}, proveniente dal metabolismo glicolitico dei carboidrati, con la formazione di \textbf{acetil-CoA}.\\ Quest'ultimo è il punto di ingresso del ciclo, in quanto reagisce con ossalacetato per formare \textbf{acido citrico}. La reazione è catalizzata dall'enzima piruvato-deidrogenasi (\textbf{PDH}).
\item \textbf{decarbossilazione di $\alpha$-chetoglutarato}, con formazione di \textbf{succinil-CoA}.\\ Esso, oltre ad essere convertito in succinato nella successiva tappa del ciclo, può reagire con la glicina per formare acido $\delta$-aminolevulinico, precursore del \textbf{gruppo eme}. Da ciò deriva l'importanza della vitamina B5 per la corretta sintesi di \textbf{emoglobina}, e dunque per il trasporto di ossigeno, e dei \textbf{citocromi}, per quello di elettroni.
\end{itemize}

\subsection{Metabolismo dei lipidi}
\subsubsection{Catabolismo}
Il coenzima A è richiesto per due reazione del ciclo della \textbf{$\beta$-ossidazione} degli acidi grassi, durante cui due unità carboniose sono rimosse per ciascun ciclo, formando \textbf{acetil-CoA}.
\subsubsection{Anabolismo}
CoA partecipa alla \textbf{via metabolica dell'acido mevalonico}, che inizia con la condensazione di due molecole di acetil-CoA formando acetoacetil-CoA. Esso reagisce poi con una terza unità di acetil-CoA, dando luogo all'\textbf{acido mevalonico}.\\
L'acido mevalonico è il precursore degli \textbf{isoprenoidi}, e dunque anche degli \textbf{steroidi} attraverso lo squalene.
È quindi chiara l'importanza del coenzima per la sintesi di colesterolo, ormoni steroidei e altri lipidi, e per la modificazione di proteine mediante isoprenilazione. \\
La produzione di acido mevalonico è controllata mediante un meccanismo di \textbf{feedback negativo}: un eccesso di colesterolo all'interno della cellula, infatti, oltre a ridurre l'espressione del recettore per LDL, è in grado di inibire due enzimi impiegati nella via, \textbf{HMG-CoA sintasi} e \textbf{HMG-CoA reduttasi}.

\begin{center}
\begin{tikzpicture}[node distance = 1.3cm, auto]

    \node (1) {acetil-CoA + acetoacetil-CoA};
    \node (2) [below of=1]{HMG-CoA sintasi};
    \node (3) [block, below of=2]{HMG-CoA};
    \node (4) [below of=3]{HMG-CoA reduttasi};
    \node (5) [block, below of=4]{acido mevalonico};
    \node (6) [block, below of=5]{colesterolo};
    \node (7) [below of=6]{LDL plasmatiche};
    \node (8) [right of=6, node distance = 5cm]{recettore LDL};

    \path [line] (1) -- (2);
    \path [line] (2) -- (3);
    \path [line] (3) -- (4);
    \path [line] (4) -- (5);
    \path [dashed, thickline] (5) -- (6);
    \path [reverseline] (6) -- (7);
    \path [dashed, line] (6) --++ (-3,-0) |- (4) node [midway, left] {$\ominus$};
    \path [dashed, line] (6) --++ (-3,0) |- (2);
    \path [dashed, line] (6) -- (8) node [very near end]{$\ominus$};

\end{tikzpicture}
\end{center}

\subsection{Metabolismo degli amminoacidi}
CoA rientra nel processamento della \textbf{leucina}, in quanto il suo chetoacido, ottenuto per deaminazione, reagisce con il coenzima formando acido acetoacetico e acetil-CoA.

\section {Proteina trasportatrice di acili}
\begin{center}\ACP\end{center}
\textbf{ACP} fa parte del complesso della \textbf{sintasi degli acidi grassi}, ed è quindi coinvolta nella biosintesi di tali composti.\\
Oltre ad introdurli con la dieta, infatti, buona parte dei lipidi derivano dal metabolismo dei carboidrati, convertiti in \textbf{piruvato} durante la glicolisi.\\
Il piruvato, prodotto nel citoplasma, diffonde passivamente nella matrice mitocondriale. In essa viene ossidato ad acetil-CoA, materiale di partenza per la sintesi di acidi grassi.\\
Acetil-CoA viene esportato dal mitocondrio come citrato, e si riforma nel citoplasma, dove avviene la sintesi lipidica.\\
Il complesso enzimatico della sintasi prevede due subunità identiche, ciascuna contenente gli enzimi necessari alla biosintesi. Esse sono attive solamente quando si combinano, con orientamento antiparallelo, nel formare un \textbf{omodimero}.

\subsection{Metabolismo dei lipidi}
\subsubsection{Anabolismo}
Acetil-CoA viene carbossilato, in una reazione dipendente da biotina come donatore, in \textbf{malonil-CoA}.\\
Il gruppo acilico di malonil-CoA viene successivamente ceduto all'\textbf{estremità tiolica} di ACP, con formazione di malonil-ACP. Esso reagisce con acetil-ACP, analogamente derivato da acetil-CoA, per formare \textbf{acetoacetil-ACP}.\\
Esso, dopo riduzione e deidratazione a butirril-ACP, reagisce nuovamente con malonil-ACP. Il composto così ottenuto possiede sei atomi di carbonio: quattro dati da butirril-ACP, due da malonil-ACP. Successive reazioni con malonil-ACP permettono l'elongazione del composto di due atomi di carbonio ad ogni ciclo, fino a produrre \textbf{acido palmitico} (16:0).\\
L'acido palmitico può essere ulteriormente allungato o desaturato, per produrre gli altri acidi grassi sintetizzabili dall'organismo.

\section{Modificazione delle proteine}
Oltre agli effetti sul metabolismo finora descritti, un vasto campo di attività del coenzima A, e dunque dell'acido pantotenico, è l'alterazione delle proteine allo scopo di modificarne attività e altre proprietà.\\
Le variazioni strutturali delle proteine possono ad esempio determinarne la \textbf{collocazione nelle membrane} plasmatiche o in quelle intracellulari, le \textbf{interazioni} con altre proteine e l'\textbf{indirizzamento} a specifici organelli o strutture.\\
Molte proteine subiscono l'\textbf{aggiunta covalente di unità carboniose fornite da CoA}, in qualità di donatore, oppure grazie ad esso sintetizzate.\\
Le tre principali categorie di modificazione richiedente CoA sono:
\begin{itemize}
\item \textbf{acilazione}: aggiunta di radicali acilici di lunghezza variabile.\\ I due acidi grassi a lunga catena più comunemente addizionati come gruppi acilici sono quello \textbf{miristico} (14:0), su residui di glicina, e quello \textbf{palmitico}, sulla catena laterale di residui di cisteina.\\
La \textbf{palmitoilazione} si realizza mediante un legame tioestereo \textbf{instabile}, di facile idrolisi. Sono perciò possibili cicli di palmitoilazione e depalmitoilazione adatti a regolare la funzionalità di una proteina a seconda delle esigenze. Interessa ad esempio la subunità $\alpha$ delle proteine G, recettori di membrana, proteina del citoscheletro, di gap junction e neuronali, ed enzimi come l'acetilcolina-esterasi. È inoltre necessaria al distacco delle vescicole dalle cisterne del Golgi.\\
La \textbf{miristoilazione} invece prevede la formazione di un legame ammidico \textbf{stabile}. Anch'essa è applicata alle subunità $\alpha$ e ad enzimi, oltre che a fattori di ribosilazione di ADP, chinasi e proteine del sistema immunitario e alla recuperina, coinvolta nel ripristino dell'eccitabilità della visione.
\item \textbf{acetilazione}: particolare acilazione in cui la catena R è un semplice metile.\\ È frequente all'estremità amminica di proteine solubili al fine di alterarne l'affinità per recettori o altre proteine.
\item \textbf{prenilazione}: aggiunta di catene isopreniche.
Avviene ad esempio su residui di cisteina di motivi CAAX (cisteina, amminoacido alifatico e residuo C-terminale), per aggiunta di \textbf{farnesile} o \textbf{geranilgeranile}.\\
Sono soggette a prenilazione le proteine \textbf{Ras}, coinvolte nella trasduzione del segnale, le \textbf{Rab}, che regolano il traffico vescicolare, le lamìne nucleari, la subunità $\gamma$ delle proteine G e varie chinasi.
\end{itemize}
Esse possono essere \textbf{co-traduzionali}, ovvero attuate durante la sintesi del peptide, oppure \textbf{post-traduzionali}.

\end{document}
