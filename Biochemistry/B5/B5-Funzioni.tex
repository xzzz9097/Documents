\documentclass[a4paper, 12pt]{article}

\usepackage[italian]{babel}
\usepackage[version=3]{mhchem}
\usepackage{textgreek}
\usepackage{gensymb}
\usepackage{chemfig}
\usepackage{tikz}
\usetikzlibrary{shapes,arrows}

\tikzstyle{block} = [rectangle, draw, text centered, minimum height=1.5em]
\tikzstyle{line} = [draw, -latex']
\tikzstyle{reverseline} = [draw, latex'-]
\tikzstyle{thickline} = [draw, -latex', line width = 0.4mm]

\usepackage[utf8]{inputenc}

\newcommand*\derivesubmol[4]{%
    \saveexpandmode\saveexploremode\expandarg\exploregroups
    \csname @\ifcat\relax\noexpand#2first\else second\fi oftwo\endcsname
        {\expandafter\StrSubstitute\@car#2\@nil}
        {\expandafter\StrSubstitute\csname CF@@#2\endcsname}
    {\@empty#3}{\@empty#4}[\temp@]%
    \csname @\ifcat\relax\noexpand#1first\else second\fi oftwo\endcsname
        {\expandafter\let\@car#1\@nil}
        {\expandafter\let\csname CF@@#1\endcsname}\temp@
    \restoreexpandmode\restoreexploremode
}

\setcrambond{2pt}{}{}

\definesubmol{adenina}{N*5([::-18]-*6(-N=-N=(-NH_2)-)=-N=-)}

\definesubmol{s1}{-[2]!{adenina}}
\definesubmol{s2}{-[6,.6]O-PO_3^{2-}}
\definesubmol{s3}{-[6,.6]OH}

\definesubmol{ribosio}{%
    -[::-90,2]%
        (%
            -[::115,1.176]O%
            -[::-50,1.176]%
        )%
    <[::45,0.8]%
        (%
          (!{s3})
          -[::45,,,,line width=2pt,shorten <=-.5pt,shorten >=-.5pt]%
              (!{s2})%
          >[::45,0.8]%
              (!{s1})%
         )%
}

\def \CoA {\tiny\chemfig{HS-CH_2-CH_2-NH-C(=[:90]O)-CH_2-CH_2-NH-C(=[:90]O)-CH(-[:270]OH)-C(-[:90]CH3)(-[:270]CH3)-CH_2-O-PO_2^{-}-O-PO_2^{-}-O-CH_2!{ribosio}}}

\date{}

\setatomsep{15pt}

\title{%
  Vitamina B5 \\
  \large Funzioni biologiche}

\begin{document}

\begin{titlepage}

\maketitle

\tableofcontents

\section{Introduzione}
La quinta vitamina del gruppo B, corrispondente all'acido pantonenico, ricopre numerose funzioni biologiche di fondamentale importanza in tutti gli esseri viventi.\\
Essa espleta il suo ruolo principale in quanto componente essenziale del \textbf{coenzima A}, molecola centrale del metabolismo.\\
Rientra inoltre nella \textbf{proteina di trasporto di acili (ACP)}, componente importante nell'anabolismo degli acidi grassi.

\end{titlepage}

\section{Coenzima A}
\begin{center}\CoA\end{center}
Il coenzima A è coinvolto in reazioni di trasferimento di gruppi acetilici e acilici, relativi al metabolismo ossidativo e al catabolismo.\\
Grazie al dominio adenosinico, CoA è in grado di legarsi agli enzimi che lo richiedono, metre quello fosfopanteinico agisce nel legame dei substrati e nel loro spostamento da un centro catalitico all'altro.\\
Il legame fra un gruppo acilico od acetilico, e quello tiolico della fosfopanteina, porta alla formazione di un tioestere, rispettivamente \textbf{acil-CoA} o \textbf{acetil-CoA}. Essi sono \textbf{composti ad alta energia}, a causa della natura instabile del legame tioestereo, e possono quindi partecipare a numerose reazioni biochimiche.

\subsection{Metabolismo dei carboidrati}
\subsubsection{Anabolismo}
CoA ha un ruolo fondamentale nel ciclo degli acidi tricarbossilici. Esso infatti partecipa a:
\begin{itemize}
\item \textbf{decarbossilazione del piruvato}, proveniente dal metabolismo glicolitico dei carboidrati, con la formazione di \textbf{acetil-CoA}.\\ Quest'ultimo è il punto di ingresso del ciclo, in quanto reagisce con ossalacetato per formare \textbf{acido citrico}. La reazione è catalizzata dall'enzima piruvato-deidrogenasi (\textbf{PDH}).
\item \textbf{decarbossilazione di $\alpha$-chetoglutarato}, con formazione di \textbf{succinil-CoA}.\\ Esso, oltre ad essere convertito in succinato nella successiva tappa del ciclo, può reagire con la glicina per formare acido $\delta$-aminolevulinico, precursore del \textbf{gruppo eme}. Da ciò deriva l'importanza della vitamina B5 per la corretta sintesi di \textbf{emoglobina}, e dunque per il trasporto di ossigeno, e dei \textbf{citocromi}, per quello di elettroni.
\end{itemize}

\subsection{Metabolismo dei lipidi}
\subsubsection{Catabolismo}
Il coenzima A è richiesto per due reazione del ciclo della \textbf{$\beta$-ossidazione} degli acidi grassi, durante cui due unità carboniose sono rimosse per ciascun ciclo, formando \textbf{acetil-CoA}.
\subsubsection{Anabolismo}
CoA partecipa alla \textbf{via metabolica dell'acido mevalonico}, che inizia con la condensazione di due molecole di acetil-CoA formando acetoacetil-CoA. Esso reagisce poi con una terza unità di acetil-CoA, dando luogo all'\textbf{acido mevalonico}.\\
L'acido mevalonico è il precursore degli \textbf{isoprenoidi}, e dunque anche degli \textbf{steroidi} attraverso lo squalene.
È quindi chiara l'importanza del coenzima per la sintesi di colesterolo, ormoni steroidei e altri lipidi, e per la modificazione di proteine mediante isoprenilazione. \\
La produzione di acido mevalonico è controllata mediante un meccanismo di \textbf{feedback negativo}: un eccesso di colesterolo all'interno della cellula, infatti, oltre a ridurre l'espressione del recettore per LDL, è in grado di inibire due enzimi impiegati nella via, \textbf{HMG-CoA sintasi} e \textbf{HMG-CoA reduttasi}.

\begin{center}
\begin{tikzpicture}[node distance = 1.3cm, auto]

    \node (1) {acetil-CoA + acetoacetil-CoA};
    \node (2) [below of=1]{HMG-CoA sintasi};
    \node (3) [block, below of=2]{HMG-CoA};
    \node (4) [below of=3]{HMG-CoA reduttasi};
    \node (5) [block, below of=4]{acido mevalonico};
    \node (6) [block, below of=5]{colesterolo};
    \node (7) [below of=6]{LDL plasmatiche};
    \node (8) [right of=6, node distance = 5cm]{recettore LDL};

    \path [line] (1) -- (2);
    \path [line] (2) -- (3);
    \path [line] (3) -- (4);
    \path [line] (4) -- (5);
    \path [dashed, thickline] (5) -- (6);
    \path [reverseline] (6) -- (7);
    \path [dashed, line] (6) --++ (-3,-0) |- (4) node [midway, left] {$\ominus$};
    \path [dashed, line] (6) --++ (-3,0) |- (2);
    \path [dashed, line] (6) -- (8) node [very near end]{$\ominus$};

\end{tikzpicture}
\end{center}

\subsection{Metabolismo degli amminoacidi}
CoA rientra nel processamento della \textbf{leucina}, in quanto il suo chetoacido, ottenuto per deaminazione, reagisce con il coenzima formando acido acetoacetico e acetil-CoA.


\end{document}
