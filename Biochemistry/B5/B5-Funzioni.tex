\documentclass[a4paper, 12pt]{article}

\usepackage[italian]{babel}
\usepackage[version=3]{mhchem}
\usepackage{textgreek}
\usepackage{gensymb}
\usepackage{chemfig}

\usepackage[utf8]{inputenc}

\newcommand*\derivesubmol[4]{%
    \saveexpandmode\saveexploremode\expandarg\exploregroups
    \csname @\ifcat\relax\noexpand#2first\else second\fi oftwo\endcsname
        {\expandafter\StrSubstitute\@car#2\@nil}
        {\expandafter\StrSubstitute\csname CF@@#2\endcsname}
    {\@empty#3}{\@empty#4}[\temp@]%
    \csname @\ifcat\relax\noexpand#1first\else second\fi oftwo\endcsname
        {\expandafter\let\@car#1\@nil}
        {\expandafter\let\csname CF@@#1\endcsname}\temp@
    \restoreexpandmode\restoreexploremode
}

\setcrambond{2pt}{}{}

\definesubmol{adenina}{N*5([::-18]-*6(-N=-N=(-NH_2)-)=-N=-)}

\definesubmol{s1}{-[2]!{adenina}}
\definesubmol{s2}{-[6,.6]O-PO_3^{2-}}
\definesubmol{s3}{-[6,.6]OH}

\definesubmol{ribosio}{%
    -[::-90,2]%
        (%
            -[::115,1.176]O%
            -[::-50,1.176]%
        )%
    <[::45,0.8]%
        (%
          (!{s3})
          -[::45,,,,line width=2pt,shorten <=-.5pt,shorten >=-.5pt]%
              (!{s2})%
          >[::45,0.8]%
              (!{s1})%
         )%
}

\def \CoA {\tiny\chemfig{HS-CH_2-CH_2-NH-C(=[:90]O)-CH_2-CH_2-NH-C(=[:90]O)-CH(-[:270]OH)-C(-[:90]CH3)(-[:270]CH3)-CH_2-O-PO_2^{-}-O-PO_2^{-}-O-CH_2!{ribosio}}}

\date{}

\setatomsep{15pt}

\title{%
  Vitamina B5 \\
  \large Funzioni biologiche}

\begin{document}

\maketitle

\section{Introduzione}
La quinta vitamina del gruppo B, corrispondente all'acido pantonenico, ricopre numerose funzioni biologiche di fondamentale importanza in tutti gli esseri viventi.\\
Essa espleta il suo ruolo principale in quanto componente essenziale del \textbf{coenzima A}, molecola centrale del metabolismo.\\
Rientra inoltre nella \textbf{proteina di trasporto di acili (ACP)}, componente importante nell

\section{Coenzima A}
\begin{center}\CoA\end{center}
Il coenzima A è coinvolto in reazioni di trasferimento di gruppi acetilici e acilici, relativi al metabolismo ossidativo e al catabolismo.\\
Grazie al dominio adenosinico, CoA è in grado di legarsi agli enzimi che lo richiedono, metre quello fosfopanteinico agisce nel legame dei substrati e nel loro spostamento da un centro catalitico all'altro.\\
Il legame fra un gruppo acilico od acetilico, e quello tiolico della fosfopanteina, porta alla formazione di un tioestere, rispettivamente \textbf{acil-CoA} o \textbf{acetil-CoA}. Essi sono \textbf{composti ad alta energia}, a causa della natura instabile del legame tioestereo, e possono quindi partecipare a numerose reazioni biochimiche.
\subsection{Ciclo dell'acido citrico}
CoA ha un ruolo fondamentale nel ciclo degli acidi tricarbossilici. Esso infatti partecipa a:
\begin{itemize}
\item \textbf{decarbossilazione del piruvato}, proveniente dal metabolismo glicolitico dei carboidrati, con la formazione di \textbf{acetil-CoA}. Quest'ultimo è il punto di ingresso del ciclo, in quanto reagisce con ossalacetato per formare \textbf{acido citrico}.
\item \textbf{decarbossilazione di $\alpha$-chetoglutarato}, con formazione di \textbf{succinil-CoA}
\end{itemize}

\end{document}
