\documentclass[a4paper, 12pt]{article}

\usepackage[italian]{babel}
\usepackage[version=3]{mhchem}
\usepackage{textgreek}
\usepackage{gensymb}
\usepackage{chemfig}
\usepackage{tikz}
\usepackage{fullpage}
\usetikzlibrary{shapes,arrows, positioning,calc}

\tikzstyle{block} = [rectangle, draw, text centered, minimum height=1.5em]
\tikzstyle{line} = [draw, -latex']
\tikzstyle{linewithouttip} = [draw, -]
\tikzstyle{reverseline} = [draw, latex'-]
\tikzstyle{thickline} = [draw, -latex', line width = 0.4mm]

\usepackage[utf8]{inputenc}

\makeatletter
\definearrow5{-n>}{
	\CF@arrow@shift@nodes{#3}%
	\expandafter\draw\expandafter[\CF@arrow@current@style,-CF@full](\CF@arrow@start@node)--(\CF@arrow@end@node)node[pos=.2](narrow@arctangent@i){} node[pos=.5](narrow@arctangent@ii){};%
	\edef\CF@tmp@str{[\ifx\@empty#1\@empty draw=none,\fi-CF@full]}%
	\expandafter\draw\CF@tmp@str (narrow@arctangent@i)%
		arc[radius=\CF@compound@sep*\CF@current@arrow@length*\ifx\@empty#4\@empty0.333\else#4\fi,start angle=\CF@arrow@current@angle+90,%
		delta angle=-\ifx\@empty#5\@empty60\else#5\fi]node(narrow@start){};
	\edef\CF@tmp@str{[\ifx\@empty#2\@empty draw=none,\fi-CF@full]}%
	\expandafter\draw\CF@tmp@str (narrow@arctangent@ii)%
		arc[radius=\CF@compound@sep*\CF@current@arrow@length*\ifx\@empty#4\@empty0.333\else#4\fi,start angle=\CF@arrow@current@angle+90,%
		delta angle=-\ifx\@empty#5\@empty60\else#5\fi]node(narrow@end){};
	\edef\CF@tmp@str{\if\string-\expandafter\@car\detokenize{#4.}\@nil+\else-\fi}%
	\CF@arrow@display@label{#1}{0}\CF@tmp@str{narrow@start}{#2}{1}\CF@tmp@str{narrow@end}%
}
\makeatother

\newcommand*\derivesubmol[4]{%
    \saveexpandmode\saveexploremode\expandarg\exploregroups
    \csname @\ifcat\relax\noexpand#2first\else second\fi oftwo\endcsname
        {\expandafter\StrSubstitute\@car#2\@nil}
        {\expandafter\StrSubstitute\csname CF@@#2\endcsname}
    {\@empty#3}{\@empty#4}[\temp@]%
    \csname @\ifcat\relax\noexpand#1first\else second\fi oftwo\endcsname
        {\expandafter\let\@car#1\@nil}
        {\expandafter\let\csname CF@@#1\endcsname}\temp@
    \restoreexpandmode\restoreexploremode
}

\setcrambond{2pt}{}{}

\definesubmol{adenina}{N*5([::-18]-*6(-N=-N=(-NH_2)-)=-N=-)}

\definesubmol{s1}{-[2]!{adenina}}
\definesubmol{s2}{-[6,.6]O-PO_3^{2-}}
\definesubmol{s3}{-[6,.6]OH}

\definesubmol{ribosiofosfatoadenina}{%
    -[::-90,2]%
        (%
            -[::115,1.176]O%
            -[::-50,1.176]%
        )%
    <[::45,0.8]%
        (%
          (!{s3})
          -[::45,,,,line width=2pt,shorten <=-.5pt,shorten >=-.5pt]%
              (!{s2})%
          >[::45,0.8]%
              (!{s1})%
         )%
}

\definesubmol{ribosioadenina}{%
    -[::-90,2]%
        (%
            -[::115,1.176]O%
            -[::-50,1.176]%
        )%
    <[::45,0.8]%
        (%
          (!{s3})
          -[::45,,,,line width=2pt,shorten <=-.5pt,shorten >=-.5pt]%
          >[::45,0.8]%
              (!{s1})%
         )%
}

\def \Pantotenato {
  \tiny
  \chemfig{HO-C(=[:90]O)-CH_2-CH_2-NH-C(=[:90]O)-CH(-[:270]OH)-C(-[:90]CH_3)(-[:270]CH_3)-CH_2-OH}
}

\def \CoA {
\tiny
\schemestart
$\underbrace{\chemfig{HS-CH_2-CH_2-NH-}}_{\beta\textnormal{-mercapto-etilammina}}$
$\underbrace{\chemfig{C(=[:90]O)-CH_2-CH_2-NH-C(=[:90]O)-CH(-[:270]OH)-C(-[:90]CH_3)(-[:270]CH_3)-CH_2-O}}_{\textnormal{acido pantotenico}}$
$\underbrace{\chemfig{PO_2^{-}-O-PO_2^{-}-O-CH_2!{ribosiofosfatoadenina}}}_{\textnormal{3'-fosfoadenosina difosfato}}$
\schemestop
}

\def \ACP {
\tiny
\schemestart
$\underbrace{\chemfig{HS-CH_2-CH_2-NH-C(=[:90]O)-CH_2-CH_2-NH-C(=[:90]O)-CH(-[:270]OH)-C(-[:90]CH_3)(-[:270]CH_3)-CH_2-O-PO_2^{-}-}}_{\textnormal{4'-fosfopanteteina}}$
$\underbrace{\chemleft. \chemfig{O-PO_2^{-}-O-CH_2-CH(-[:90]C(-[:90]...)=O)(-[:270]NH(-[:270]...))}\chemright\}}_{\textnormal{serina}}$
proteina
\schemestop
}

\date{}

\author{
  \normalsize
  Francesco Viviani\\
  \and
  \normalsize
  Marco Petriccione\\
  \and
  \normalsize
  Mehmet Ata Atis\\
  \and
  \normalsize
  Marco Albera
}

\setatomsep{15pt}

\title{%
  Vitamina B5 \\
  \large Acido pantotenico: funzione biologica nell'uomo
}

\begin{document}

\begin{titlepage}

\maketitle

\tableofcontents

\section{Introduzione}
La quinta vitamina del gruppo B, corrispondente all'acido pantotenico, ricopre numerose funzioni biologiche di fondamentale importanza in tutti gli esseri viventi.\\
Essa espleta il suo ruolo principale in quanto componente essenziale del \textbf{coenzima A}, molecola centrale del metabolismo.\\
Rientra inoltre nella \textbf{proteina di trasporto di acili (ACP)}, componente importante nell'anabolismo degli acidi grassi.

\end{titlepage}

\section{Chimica}
\\

\begin{center}\Pantotenato\end{center}

\begin{center}
  \tiny
  \begin{tabular}{ | l | l | }
    \hline
    Nome IUPAC & Acido 3-[(2R)-2,4-diidrossi-3,3-dimetil-butanamido]propanoico \\ \hline
    Formula bruta & \chemfig{C_9H_{17}NO_5} \\ \hline
    Massa molecolare & 219 g/mol \\ \hline
    Solubilità & 2.11 g/mL (molto solubile) \\
    \hline
  \end{tabular}
\end{center}

L'acido pantotenico o pantenolo (vitamina B5 o vitamina W) deriva dalla fusione, tramite legame carboamidico, di una molecola di \textbf{$\beta$-alanina} con una molecola di \textbf{acido pantoico}, derivato dall'acido butirrico.\\
La forma chirale biologicamente attiva è solamente quella \textbf{destrogira}. La forma levogira può fungere da antagonista dell'isomero destrogiro.

\subsection{Storia}
Come in tutte le scoperte delle vitamine idrosolubili, anche la vitamina B5 è stata studiata su batteri e cellule eucariotiche di funghi e animali.\\
Per il suo carattere \textbf{idrofilo}, la sua natura \textbf{polifunzionale}, il fatto di \textbf{precipitare} e \textbf{cristallizzarsi} facilmente, l’acido pantotenico è stato difficile da isolare direttamente mediante i classici metodi usati nella prima metà del ‘900.\\
Gli studi di Williams, Elvehjem e Jukes furono determinanti nella scoperta delle funzioni biochimiche di tale vitamina.\\
Nel \textbf{1933 R. J. Williams} e il suo gruppo hanno isolato una sostanza acida da materiale biologico che agiva come \textbf{fattore di crescita nel lievito} e lo chiamarono acido \textbf{pantotenico}: il termine deriva dal greco e significa letteralmente \textit{dappertutto}, indicando l'\textbf{ubiquitarietà} del composto all'interno comuni alimenti.\\
Nel \textbf{1939} si stabilì essere una \textbf{vitamina} quando si vide che era identico al fattore del filtrato nei ratti e galline, e agiva da \textbf{anti-dermatite}: furono Koehn e Elvehjem a compiere questa scoperta, determinando che il composto aveva funzione di \textbf{anti-pellagra}. Infatti iniettandolo in galline affette scoprirono che queste non sviluppavano più la malattia.\\
Nel \textbf{1940} si arriva alla sintesi chimica della vitamina per opera di Williams e Major.\\
Nel \textbf{1947} Lipmann identifica e spiega la sua funzione biochimica nel \textbf{coenzima A}.\\
A metà degli anni ’60 si identifica la presenza dell’acido pantotenico nei trasportatori di membrana (\textbf{ACP}).

\subsection{Isolamento}
Williams isolò diversi grammi di acido pantotenico dal fegato di pecora attraverso metodi di cromatografia e frazionamento.
Il processo prevedeva:
\begin{enumerate}
\item \textbf{Autolisi del fegato} per ottenere un filtrato chiaro
\item \textbf{Rimozione delle basi} tramite assorbimento dalla terra da follone (nome comune, minerario e commerciale, di ogni miscela di argilla ad elevata plasticità utilizzata nella decolorazione, filtratura e purificazione di oli e grassi di origine animale, minerale e vegetale)
\item \textbf{Assorbimento dei principi} sul carboncino e successiva eluizione
\item \textbf{Evaporazione} fino alla secchezza in presenza di ossalato di brucina (composto organico simile alla stricnina, usato per il riconoscimento analitico dei nitrati)
\item \textbf{Estrazione} della componente con cloroformio per ottenere pantotenato di brucina (insieme ad altri sali di brucina)
\item \textbf{Distribuzione frazionale} della brucina tra cloroformio e acqua
\item \textbf{Conversione al sale di calcio}
\item \textbf{Frazionamento del sale di calcio} mediante vari solventi
\end{enumerate}

\subsection{Caratteristiche chimico-fisiche}
La configurazione sterica dell'acido pantotenico è fondamentale nel riconoscimento enzimatico delle reazioni biochimiche: infatti soltanto la sua forma destrogira è attiva. Nelle sintesi della vitamina si forma sempre una miscela racemica.\\
Non è presente in natura nella forma libera, ma si può ritrovare come componente strutturale del \textbf{coenzima A} e della \textbf{4'-fosfopanteteina}. Quest’ultima è unita a un gruppo prostetico di trasportatori di membrana coinvolto nella sintesi di acidi grassi.\\
L’acido pantotenico è un \textbf{olio} di color \textbf{giallo pallido}, estremamente \textbf{igroscopico} e \textbf{inodore}.
È inoltre \textbf{instabile} al calore, alle basi ed agli acidi ed è \textbf{solubile} in acqua.

\section{Digestione e assorbimento}

\subsection{Fonti alimentari}
L’acido pantotenico è praticamente \textbf{ubiquitario} in ogni tipo di alimento, che sia di origine vegetale o animale, in una quantità media di 20-50 {\textmu}g/g.\\
È particolarmente abbondante in \textbf{organi animali} come fegato, rene, cervello e cuore, ma anche nel tuorlo d’uovo, nelle noccioline, nella farina integrale e nelle fave.\\
Si trova invece in minori quantità nella carne magra, nel latte, nelle patate e nei vegetali.\\
Cibi altamente raffinati come zuccheri, grassi e oli ne sono completamente privi, mentre la \textbf{suscettibilità dell’acido pantotenico alla temperatura e all’ossidazione} fa si che alimenti processati  e inscatolati perdano una consistente quota di vitamina B5. Infatti farina raffinata perde fino al 37-47\% del contenuto di questo composto, così come l’inscatolamento di carne, pesce e di latticini produce una perdita del 20-35\%.\\
Le perdite più importanti avvengono però durante la \textbf{conservazione} (46-78\%) e il \textbf{congelamento} (35-57\%) di verdure.\\
Per la maggior parte la vitamina B5 si trova sotto forma di \textbf{coenzima A}, ma si può trovare anche come acido fosfopantotenico, panteteina e fosfopanteteina. Eccezioni sono il \textbf{latte} umano e bovino in cui l’acido pantotenico è in forma \textbf{libera} (non legata) e costituisce il 90\% del totale di questo composto.\\
L’assunzione adeguata di vitamina B5 varia da età ad età, generalmente aumenta crescendo. Si passa infatti dai 1.7 mg/giorno dei neonati ai 2-4 dei bambini fino ai 13 anni, per arrivare ai \textbf{5 mg/giorno} per gli adolescenti e adulti. Durante la gravidanza e il periodo puerperale per la donna i valori raccomandati aumentano, arrivando a 6-7 mg/giorno.

\subsection{Assorbimento}
Dato che \textbf{l’organismo umano non può sintetizzare da sé la vitamina B5}, esso deve essere \textbf{assorbito dalla dieta}.\\ Tuttavia ci sono prove che esso venga sintetizzato dalla \textbf{flora batterica intestinale} e che poi venga assorbito nel \textbf{colon}.\\
Per quanto riguarda l’assorbimento dal cibo, si deve anticipare che non è regolato dai suoi livelli di assunzione dalla dieta, questo perché il pantotenato è molto importante per il metabolismo, e anche se venisse assorbito \textbf{ad alte dosi non risulterebbe tossico}.\\
Nel lume intestinale il \textbf{CoA}, che è appunto la forma in cui più frequentemente si trova l’acido pantotenico negli alimenti, viene \textbf{idrolizzato a panteteina} da due idrolasi non specifiche che sono la pirofosfatasi e la fosfatasi.\\
A questo punto la panteteina può essere assorbita così com’è oppure può essere ulteriormente scissa in \textbf{acido pantotenico} e \textbf{$\beta$-mercaptoetilammina} dall’azione della \textbf{panteteinasi}.\\
La vitamina B5 entra poi negli enterociti, grazie ad un \textbf{trasporto accoppiato al sodio} mediato dalla proteina \textbf{SMVT} (\textit{sodium-dependent multivitamin transporter}). Si tratta di un trasportatore transmembrana che non è specifico per l’acido pantotenico, ma come dice il nome, funziona per altre vitamine, come la biotina e derivati della biotina. SMVT infatti interagisce principalmente, ma non esclusivamente, con la lunga catena laterale contenente il gruppo carbossilico, che è presente appunto in questi composti.\\
È stato calcolato inoltre che per ogni molecola di pantotenato (o di biotina) vengano \textbf{co-trasportati 2 ioni \chemfig{Na^{+}}}, quindi, dato che a pH fisiologico sia l’acido pantotenico che la biotina sono anioni monovalenti, il trasporto è \textbf{elettrogenico}.\\
Una volta che la vitamina B5 è entrata nell’enterocita deve uscire dalla membrana basolaterale per entrare nel torrente sanguigno, ma il meccanismo attraverso cui questo avviene non è ancora stato stabilito.\\
Dai vasi sanguigni poi il pantotenato viene assorbito dalle cellule che lo necessitano, in modo simile a quello che avviene nell’intestino.\\
Sono stati individuati \textbf{mRNA di SMVT in tutti i tessuti} (intestino, fegato, rene, cuore, polmone, muscolo scheletrico, cervello e placenta) a testimoniare il fatto che questo composto è molto importante per l’organismo.\\
Questo tipo di trasporto avviene \textbf{contro gradiente}, infatti la concentrazione di acido pantotenico nel fegato e nel cuore è rispettivamente di 10-15 {\textmu}M e di 100 {\textmu}M, rispetto ad una del plasma di 1-5 {\textmu}M. Ne segue che al trasporto dell'acido pantotenico è accoppiato quello del sodio. La \textit{K\textsubscript{m}} apparente di SMVT è 15-20 {\textmu}M, quindi rispetto alla concentrazione plasmatica è molto alta, e quindi il trasportatore sarà \textbf{difficilmente saturabile}.\\
L’uptake di vitamina B5 è \textbf{stimolato da un’alta concentrazione di sodio} e inibito da potassio, ouabaina, gramicidina D, cianide, azide, o semplicemente se la concentrazione di sodio scende sotto i 40 mM.\\
\textbf{La funzione di SMVT sembra essere connessa con quella di PKC}, infatti il trasportatore comprende 2 siti fosforilabili da questa proteina chinasi. Se si inibisce la PKC il trasporto di acido pantotenico e di biotina verrà infatti inibito.\\
Sappiamo inoltre che la vitamina B5 è sintetizzata dalla microflora batterica, anche se non se ne conosce la quantità precisa.

\subsection{Escrezione}
L’acido pantotenico è escreto tramite le urine in una \textbf{quantità proporzionale all’apporto} attraverso la dieta, infatti quando si introducono quantità maggiori di \textbf{4 mg/giorno} la sua escrezione supera il livello basale.\\
Questo avviene dopo che è stato liberato dal \textbf{coenzima A} da una serie di reazioni di idrolisi che liberano \textbf{$\beta$-mercaptoetilammina} e \textbf{fosfato}.

\section{Coenzima A}
\begin{center}\CoA\end{center}
Il coenzima A è coinvolto in reazioni di trasferimento di gruppi acetilici e acilici, relativi al metabolismo ossidativo e al catabolismo.\\
Grazie al dominio adenosinico, CoA è in grado di legarsi agli enzimi che lo richiedono, mentre quello tioetanolamminico agisce nel legame dei substrati carboniosi e nel loro spostamento da un centro catalitico all'altro.\\
Il legame fra un gruppo acilico od acetilico, e quello tiolico del coenzima, porta alla formazione di un tioestere, rispettivamente \textbf{acil-CoA} o \textbf{acetil-CoA}. Esso è un \textbf{composto ad alta energia}, a causa della natura instabile del legame tioestereo, il cui $\Delta G'\degree$ di idrolisi permette lo svolgimento di numerose reazioni biochimiche.

\subsection{Sintesi}
L'acido pantotenico è essenziale per la sintesi dell'omonimo dominio presente in CoA.\\
La prima fase della sintesi prevede la fosforilazione ad \textbf{acido 4'-fosfopantotenico}, mediata dalla \textbf{chinasi dell'acido pantotenico} (PanK). Gli eucarioti possiedono PanK-II, mentre le varianti I e III sono proprie dei procarioti.\\
PanK agisce in un ampio intervallo di pH (fra 6 e 9) e lega: \begin{itemize}
\item \textbf{acido pantotenico} con \textit{K\textsubscript{m}} $\simeq$ 20 {\textmu}M, in qualità di accettore di fosfato
\item  \textbf{Mg-ATP} con \textit{K\textsubscript{m}} $\simeq$ 0.6 {\textmu}M, in qualità di donatore di fosfato
\end{itemize}
La fosforilazione dell'acido pantotenico è un fondamentale punto di controllo della sintesi del coenzima A. Infatti PanK sono regolate da:
\begin{itemize}
\item \textbf{vari anioni}, che attivano o inibiscono non specificamente l'enzima
\item \textbf{CoA e suoi derivati}, che inibiscono la sintesi di nuovo coenzima con un meccanismo a \textbf{feedback negativo}
\item \textbf{carnitina}, amminoacido trasportatore di acidi grassi nel mitocondrio, che indirettamente attiva l'enzima bloccando l'inibizione da parte dei derivati di CoA
\end{itemize}
Per la compartecipazione di questi fattori è stato ad esempio verificato che digiuno e diabete di tipo I (da ipoinsulinemia) incrementano l'attività di PanK e dunque la quantità di CoA. Al contrario, eccesso di glucosio e di acidi grassi all'interno della cellula riducono l'attività di PanK, per sottrazione di carnitina e maggiori concentrazioni di acetil-CoA.\\
Le successive fasi della sintesi del coenzima sono condotte su un complesso proteico dotato di vari siti catalitici e sono illustrate nello schema sottostante.\\
La \textbf{coenzima A idrolasi} catalizza l'idrolisi di CoA a 3'-5'-ADP e 4'-fosfopanteteina; quest'ultima può essere riutilizzata per la sintesi di nuovo coenzima. Si parla perciò di \textbf{ciclo del CoA/4'-fosfopanteteina}. Ogni esecuzione del ciclo richiede due molecole di ATP e ne produce una di ADP, una di PP\textsubscript{i} e una di 3',5'-ADP.
\begin{center}
\begin{tikzpicture}[node distance = 2cm, trim left = (1), trim right = (1)]

    \node (1) [block]{Acido pantotenico};
    \node (2) [block, below of=1]{Acido 4'-fosfopantotenico};
    \node (3) [block, below of=2]{4'-fosfopantotenilcisteina};
    \node (4) [block, below of=3]{4'-fosfopanteteina};
    \node (5) [block, below of=4]{4'-defosfo-CoA};
    \node (6) [block, below of=5]{Coenzima A};

    \path [line] (1) -- (2) node (7) [midway, left, xshift = -2mm] {\textit{PanK}};
    \path [line] (2) -- (3) node [midway, left, xshift = -2mm, align = right, text width = 5cm] {\textit{4'-fosfopantotenilcistina sintetasi}};
    \path [line] (3) -- (4) node [midway, left, xshift = -2mm, align = right, text width = 5cm] {\textit{4'-fosfopantotenilcistina decarbossilasi}};
    \path [line] (4) -- (5) node [midway, left, xshift = -2mm, align = right, text width = 5cm] {\textit{Defosfo-CoA pirofosforilasi}};
    \path [line] (5) -- (6) node [midway, left, xshift = -2mm, align = right, text width = 5cm] {\textit{Defosfo-CoA chinasi}};
    \path [dashed, line] (6) --++ (-6,-0) |- (7) node [midway, left] {$\ominus$};
    \path [line] (6) --++ (3,0) |- (4) node[midway, right]{\textit{CoA idrolasi}}

\end{tikzpicture}
\end{center}

\subsection{Acilazione}
\begin{center}
\setatomsep{30pt}
\footnotesize
\schemestart
ATP \+
\chemname{\chemfig{R-C(=[:90]O)-O^{-}}}{Gruppo acile} \+
\chemname{\chemfig{CoA-SH}}{Tiolo}
\arrow(.mid east--.mid west){<=>} \chemname{\chemfig{R-C(=[:90]O)-S-CoA}}{Tioestere} \+
AMP \+
PP\textsubscript{i}
\schemestop
\end{center}
La coniugazione di un gruppo acilico a CoA richiede consumo di energia, necessaria ad \textit{attivare} l'acido grasso per renderlo disponibile all'attacco sul coenzima.

\begin{center}
\setatomsep{20pt}
\tiny
\schemestart
\chemname{\chemfig{PO_3^{2-}-O-PO_2^{-}-O-PO_2^{-}-O-[:0, 1.5]{Ribosio}-[:0, 2]{Adenina}}}{ATP} \+
\chemname{\chemfig{R-C(=[:90]O)-O^{-}}}{Gruppo acile}
\arrow{-U>[][*{0}\schemestart PP\textsubscript{i}\arrow 2P\textsubscript{i}\schemestop]}[-90]
\chemname{\chemfig{R-C(=[:90]O)-O-PO_2^{-}-[:0, 1.5]{Ribosio}-[:0, 2]{Adenina}}}{Acil-adenilato}
\arrow{-U>[*{0}CoA-SH]}[-90]
\chemname{\chemfig{PO_3^{2-}-O-[:0, 1.5]{Ribosio}-[:0, 2]{Adenina}}}{AMP} \+
\chemname{\chemfig{R-C(=[:90]O)-S-CoA}}{Acil-CoA}
\schemestop
\end{center}

Il tioestere ottenuto è un composto \textbf{ad alta energia} poichè, a differenza dei normali esteri, non può stabilizzarsi per risonanza del doppio legame fra i due atomi di ossigeno. Infatti l'atomo di zolfo, appartenente al terzo periodo, è più ingombrante e meno elettronegativo dell'ossigeno, e per questo il legame \chemfig{C-S} è più lungo di quello \chemfig{C-O} e la sovrapposizione con gli orbitali dell'ossigeno peggiore.\\
A causa di tale instabilità, l'idrolisi del tiostere fornisce una variazione di energia libera significativamente negativa. Ad esempio, dall'idrolisi del tioestere dell'acido palmitico:
\begin{center}
\tiny
$\Delta G'\degree = -32.5$ kJ/mol di palmitoil-CoA
\end{center}

\subsection{Metabolismo dei carboidrati}
\subsubsection{Anabolismo}
CoA ha un ruolo fondamentale nel ciclo degli acidi tricarbossilici. Esso infatti partecipa a:
\begin{itemize}
\item \textbf{decarbossilazione del piruvato}, proveniente dal metabolismo glicolitico dei carboidrati, con la formazione di \textbf{acetil-CoA}.\\ Quest'ultimo è il punto di ingresso del ciclo, in quanto reagisce con ossalacetato per formare \textbf{acido citrico}. La reazione è catalizzata dall'enzima piruvato-deidrogenasi (\textbf{PDH}).
\begin{center}
\tiny
\setatomsep{15pt}
\setcompoundsep{10em}
\schemestart
\chemname{\chemfig{CH_3(-[:90]C(-[:90]C(-[:45]O^{-})(=[:135]O))=O)}}{Piruvato}
\arrow(.mid east--.mid west){-U>[CoA-SH + NAD\textsuperscript{+}][NADH][][0.2]}[,2]
\chemname{\chemfig{CH_3((-[:90]C(-[:45]S-CoA)(=[:135]O)))}}{Acetil-CoA}
\+
\chemfig{CO_2}
\schemestop
\end{center}
\begin{center}\tiny$\Delta G'\degree = -33.4$ kJ/mol\end{center}
\item \textbf{decarbossilazione di $\alpha$-chetoglutarato}, con formazione di \textbf{succinil-CoA}.\\ Esso, oltre ad essere convertito in succinato nella successiva tappa del ciclo, può reagire con la glicina per formare acido $\delta$-aminolevulinico, precursore del \textbf{gruppo eme}. Da ciò deriva l'importanza della vitamina B5 per la corretta sintesi di \textbf{emoglobina}, e dunque per il trasporto di ossigeno, e dei \textbf{citocromi}, per quello di elettroni.
\begin{center}
\tiny
\setatomsep{15pt}
\setcompoundsep{10em}
\schemestart
\chemname{\chemfig{COO^{-}(-[:90]C(-[:90]CH_2(-[:90]CH_2-COO^{-}))=O)}}{$\alpha$-chetoglutarato}
\arrow(.mid east--.mid west){-U>[CoA-SH + NAD\textsuperscript{+}][NADH][][0.2]}[,2]
\chemname{\chemfig{O(=[:90]C((-S-CoA)(-[:90]CH_2(-[:90]CH_2-COO^{-}))))}}{Succinil-CoA}
\+
\chemfig{CO_2}
\schemestop
\end{center}
\begin{center}\tiny$\Delta G'\degree = -33.5$ kJ/mol\end{center}
\end{itemize}

\subsection{Metabolismo dei lipidi}
\subsubsection{Catabolismo}
Il coenzima A è richiesto per due reazione del ciclo della \textbf{$\beta$-ossidazione} degli acidi grassi, durante cui due unità carboniose sono rimosse per ciascun ciclo, formando \textbf{acetil-CoA}.
\subsubsection{Anabolismo}
CoA partecipa alla \textbf{via metabolica dell'acido mevalonico}, che inizia con la condensazione di due molecole di acetil-CoA formando acetoacetil-CoA. Esso reagisce poi con una terza unità di acetil-CoA, dando luogo all'\textbf{acido mevalonico}.\\
L'acido mevalonico è il precursore degli \textbf{isoprenoidi}, e dunque anche degli \textbf{steroidi} attraverso lo squalene.
È quindi chiara l'importanza del coenzima per la sintesi di colesterolo, ormoni steroidei e altri lipidi, e per la modificazione di proteine mediante isoprenilazione. \\
La produzione di acido mevalonico è controllata mediante un meccanismo di \textbf{feedback negativo}: un eccesso di colesterolo all'interno della cellula, infatti, oltre a ridurre l'espressione del recettore per LDL, è in grado di inibire due enzimi impiegati nella via, \textbf{HMG-CoA sintasi} e \textbf{HMG-CoA reduttasi}.

\begin{center}
\begin{tikzpicture}[node distance = 2cm, auto, trim left = (1), trim right = (1)]

    \node (1) [block]{acetil-CoA + acetoacetil-CoA};
    \node (2) [block, below of=1]{HMG-CoA};
    \node (3) [block, below of=2]{acido mevalonico};
    \node (4) [block, below of=3]{colesterolo};
    \node (5) [below of=4]{LDL plasmatiche};
    \node (6) [right of=4, node distance = 5cm]{recettore LDL};

    \path [line] (1) -- (2) node (7) [midway, left, xshift = -2mm] {\textit{HMG-CoA sintasi}};
    \path [line] (2) -- (3) node (8) [midway, left, xshift = -2mm] {\textit{HMG-CoA reduttasi}};
    \path [dashed, thickline] (3) -- (4);
    \path [reverseline] (4) -- (5);
    \path [dashed, line] (4) --++ (-5,-0) |- (7) node [midway, left] {$\ominus$};
    \path [dashed, line] (4) --++ (-5,0) |- (8) node [midway, left] {$\ominus$};
    \path [dashed, line] (4) -- (6) node [very near end]{$\ominus$};

\end{tikzpicture}
\end{center}

\subsection{Metabolismo degli amminoacidi}
CoA rientra nel processamento della \textbf{leucina}, in quanto il suo chetoacido, ottenuto per deaminazione, reagisce con il coenzima formando acido acetoacetico e acetil-CoA.

\section {Proteina trasportatrice di acili}
\begin{center}\ACP\end{center}
\textbf{ACP} fa parte del complesso della \textbf{sintasi degli acidi grassi}, ed è quindi coinvolta nella biosintesi di tali composti.\\
Oltre ad introdurli con la dieta, infatti, buona parte dei lipidi derivano dal metabolismo dei carboidrati, convertiti in \textbf{piruvato} durante la glicolisi.\\
Il piruvato, prodotto nel citoplasma, diffonde passivamente nella matrice mitocondriale. In essa viene ossidato ad acetil-CoA, materiale di partenza per la sintesi di acidi grassi.\\
Acetil-CoA viene esportato dal mitocondrio come citrato, e si riforma nel citoplasma, dove avviene la sintesi lipidica.\\
Il complesso enzimatico della sintasi prevede due subunità identiche, ciascuna contenente gli enzimi necessari alla biosintesi. Esse sono attive solamente quando si combinano, con orientamento antiparallelo, nel formare un \textbf{omodimero}.

\subsection{Metabolismo dei lipidi}
\subsubsection{Anabolismo}
Acetil-CoA viene carbossilato, in una reazione dipendente da biotina come donatore, in \textbf{malonil-CoA}.\\
Il gruppo acilico di malonil-CoA viene successivamente ceduto all'\textbf{estremità tiolica} di ACP, con formazione di malonil-ACP. Esso reagisce con acetil-ACP, analogamente derivato da acetil-CoA, per formare \textbf{acetoacetil-ACP}.\\
Questo composto, dopo riduzione e deidratazione a butirril-ACP, reagisce nuovamente con malonil-ACP. Il composto così ottenuto possiede sei atomi di carbonio: quattro dati da butirril-ACP, due da malonil-ACP. Successive reazioni con malonil-ACP permettono l'elongazione del composto di due atomi di carbonio ad ogni ciclo, fino a produrre \textbf{acido palmitico} (16:0).\\
L'acido palmitico può essere ulteriormente allungato o desaturato, per produrre gli altri acidi grassi sintetizzabili dall'organismo.

\begin{center}
\begin{tikzpicture}[auto, trim left = (5), trim right = (5)]

    \node (1) [block]{acetil-CoA};
    \node (2) [block, right = 6cm of 1]{malonil-CoA};
    \node (3) [block, below = 1cm of 1]{acetil-ACP};
    \node (4) [block, below = 1cm of 2]{malonil-ACP};
    \node (5) [block, below = 1cm of {$(3)!0.5!(4)$}]{acetoacetil-ACP};
    \node (6) [block, below = 1cm of 5]{butirril-ACP};
    \node (7) [block, below = 1cm of 6]{C6-ACP};
    \node (8) [block, below = 1cm of 7]{C8-, C10-, Cn-ACP};
    \node (9) [block, below = 1cm of 8]{palmitil-ACP (C16)};

    \path [line] (1) -- (2);
    \path [line] (2) -- (4);
    \path [line] (1) -- (3);
    \path [line] (3) -- (5);
    \path [line] (4) -- (5);
    \path [line] (5) -- (6);
    \path [line] (6) -- (7) node (11) [midway,right];
    \path [line] (7) -- (8) node (12) [midway, right];
    \path [dashed, line] (8) -- (9);

    \path [line] (4) --++ (2,-0) |- (11);
    \path [line] (4) --++ (2,-0) |- (12);

\end{tikzpicture}
\end{center}

\section{Modificazione delle proteine}
Oltre agli effetti sul metabolismo finora descritti, un vasto campo di attività del coenzima A, e dunque dell'acido pantotenico, è l'alterazione delle proteine allo scopo di modificarne attività e altre proprietà.\\
Le variazioni strutturali delle proteine possono ad esempio determinarne la \textbf{collocazione nelle membrane} plasmatiche o in quelle intracellulari, le \textbf{interazioni} con altre proteine e l'\textbf{indirizzamento} a specifici organelli o strutture.\\
Molte proteine subiscono l'\textbf{aggiunta covalente di unità carboniose fornite da CoA}, in qualità di donatore, oppure grazie ad esso sintetizzate.\\
Le tre principali categorie di modificazioni richiedenti CoA sono:
\begin{itemize}
\item \textbf{acilazione}: aggiunta di radicali acilici di lunghezza variabile.\\ I due acidi grassi a lunga catena più comunemente addizionati come gruppi acilici sono quello \textbf{miristico} (14:0), su residui di glicina, e quello \textbf{palmitico}, sulla catena laterale di residui di cisteina.\\
La \textbf{palmitoilazione} si realizza mediante un legame tioestereo \textbf{instabile}, di facile idrolisi. Sono perciò possibili cicli di palmitoilazione e depalmitoilazione adatti a regolare la funzionalità di una proteina a seconda delle esigenze. Interessa ad esempio la subunità $\alpha$ delle proteine G, recettori di membrana, proteina del citoscheletro, di gap junction e neuronali, ed enzimi come l'acetilcolina-esterasi. È inoltre necessaria al distacco delle vescicole dalle cisterne del Golgi.\\
La \textbf{miristoilazione} invece prevede la formazione di un legame ammidico \textbf{stabile}. Anch'essa è applicata alle subunità $\alpha$ e ad enzimi, oltre che a fattori di ribosilazione di ADP, chinasi e proteine del sistema immunitario e alla recuperina, coinvolta nel ripristino dell'eccitabilità della visione.
\item \textbf{acetilazione}: particolare acilazione in cui la catena R è un semplice metile.\\ È frequente all'estremità amminica di proteine solubili al fine di alterarne l'affinità per recettori o altre proteine.
\item \textbf{prenilazione}: aggiunta di catene isopreniche.
Avviene ad esempio su residui di cisteina di motivi CAAX (cisteina, amminoacido alifatico e residuo C-terminale), per aggiunta di \textbf{farnesile} o \textbf{geranilgeranile}.\\
Sono soggette a prenilazione le proteine \textbf{Ras}, coinvolte nella trasduzione del segnale, le \textbf{Rab}, che regolano il traffico vescicolare, le lamìne nucleari, la subunità $\gamma$ delle proteine G e varie chinasi.
\end{itemize}
Esse possono essere \textbf{co-traduzionali}, ovvero attuate durante la sintesi del peptide, oppure \textbf{post-traduzionali}.

\subsection{Acetilazione di $\boldsymbol\beta$-endorfina}
Il \textbf{neurotrasmettitore} peptidico cerebrale $\beta$-endorfina determina effetto \textbf{analgesico} e influenza apprendimento, attitudine e attività sessuale. Subisce acetilazione post-traduzionale all'estremità ammino-terminale, che lo disattiva per impossibilità di legare i propri recettori.

\subsection{Acetilazione degli istoni}
Gli istoni \textbf{legano il DNA} determinandone la configurazione eu- o eterocromatinica. L'acetilazione dei residui lisinici di queste proteine in specifiche regioni di cromatina \textbf{riduce la loro affinità per il DNA}, permettendo l'assunzione di una forma maggiormente rilassata e dunque favorendo la trascrizione del tratto di genoma corrispondente.\\
Il processo è catalizzato dall'enzima \textbf{istone-acetiltransferasi} (HAC), che sfrutta acetil-CoA come donatore di acetile, ed è reversibile grazie all'intervento di \textbf{istone-deacetilasi} (HDAC).\\
L'acetilazione istonica reversibile permette alla cellula di trascrivere all'occorenza i geni richiesti per la propria attività, e di \textit{spegnerli} quando non più necessari.

\subsection{Acetilazione di $\boldsymbol\alpha$-tubulina}
I microtubuli, costituenti del citoscheletro, sono eterodimeri di $\alpha$ e $\beta$-tubulina.\\
L'acetilazione del gruppo amminico di specifici residui di lisina di $\alpha$-tubilina \textbf{stabilizza il microtubulo}; al contrario, la deacetilazione ne favorisce la depolimerizzazione.

\subsection{Acilazione di eNOS}
La \textbf{ossido nitrico-sintasi endoteliale} (NOSIII) è presente nella membrana di cellule endoteliali dei vasi sanguigni e dei cardiomiociti, dove sintetizza NO in risposta ad aumento della concentrazione intracellulare di calcio.\\
NO agisce in qualità di \textbf{vasodilatatore} grazie al suo effetto di rilassamento della muscolatura liscia perivasale. Media anche azioni \textbf{antitrombotiche} ed \textbf{antinfiammatorie}, ed inibisce la stimolazione vasocostrittoria promossa da angiotensina II.\\
L'enzima è mantenuto in membrana grazie ad una doppia acilazione. Subisce infatti:
\begin{itemize}
\item \textbf{miristoilazione} co-traduzionale
\item \textbf{palmitoilazione} post-traduzionale
\end{itemize}
La particolare estremità N-terminale di eNOS lega perciò acido miristico con un suo sito, e acido palmitico con altri due.\\
Ciò è di fondamentale importanza per l'inserimento in membrana plasmatica, in particolare nelle sue invaginazioni dette \textbf{caveole}, dove è in rapporto con altre proteine e canali utili alla sua attività.\\
eNOS ha inoltre la particolarità di legarsi a \textbf{caveolina-1}, che esercita su di essa un tono inibitorio. Quando la concentrazione citosolica di calcio aumenta, tale legame viene rimpiazzato da quello con la \textbf{calmodulina}, che determina l'avvio della sintesi di NO.\\
Se la palmitoilazione non avviene, eNOS non è presente nelle caveole e la cellula endoteliale non può produrre ossido nitrico.

\subsection{Acilazione di proteine G}
Le \textbf{proteine G monomeriche} sono lipidate con acidi grassi o con \textbf{farnesile}, e ciò ne garantisce la permanenza in membrana. L'unica eccezione è rappresentata da \textbf{Ran}, che gestisce il traffico bidirezionale di composti dal citoplasma al nucleo.\\
La \textbf{subunità} $\boldsymbol\alpha$ delle proteine G eterotrimeriche è soggetta a \textbf{palmitoilazione} reversibile catalizzata da palmitoil-transferasi, che preleva un palmitato fornito da palmitoil-CoA. Tale modifica media la \textbf{traslocazione} della subunità fra membrana plasmatica e citoplasma.\\
$\alpha$ di \textbf{G\textsubscript{i}} subisce inoltre l'aggiunta permanente di \textbf{acido miristico}. Esso consolida la localizzazione della subunità in membrana, accrescendo l'affinità per la componente $\boldsymbol{\beta\gamma}$ della proteina, e promuove l'interazione, di significato inibitorio, con l'\textbf{adenilato ciclasi}.
L'acilazione reversibile delle subunità $\alpha$ potrebbe quindi rappresentare un'ulteriore modalità di controllo per le vie di trasduzione del segnale mediate da proteine G.

\section{Carenze e implicazioni cliniche}
È ragionevole aspettarsi che carenze di acido pantotenico possano causare un ampio spettro di patologie, dato il suo coinvolgimento nella formazione di \textbf{coenzima A} e nel \textbf{metabolismo dei lipidi}.\\
Deficienze nell'approvvigionamento della vitamina causano inoltre \textbf{disfunzioni a livello mitocondriale}.\\
A sua volta, alterata l’omeostasi del CoA, la carenza di acido pantotenico è associata al \textbf{diabete}, all’\textbf{alcolismo} e alla \textbf{sindrome di Reye}.\\
Si verificano quindi profonde variazioni in risposta ad ormoni fondamentali per il metabolismo lipidico (ad esempio, glucorticoidi, insulina, glucagone e agonisti PPAR come il Clofibrato) per la carenza di acido pantotenico o in risposta ad inibitori della sua chinasi.\\
È \textbf{difficile che si verifichino gravi carenze di pantotenato}, infatti anche la caseina commerciale, \textit{vitamin-free}, può contenerne fino a 3 mg/kg.\\
Tuttavia, in condizioni di lieve carenza di pantotenato, pur non osservando variazioni di peso nei gruppi interessati, il livello di \textbf{trigliceridi sierici} e quello di \textbf{acidi grassi liberi} è \textbf{elevato}, un \textbf{riflesso della riduzione dei livelli di CoA}.\\
La carenza di pantotenato si rivela \textbf{meno preoccupante all’avanzare dell’età} perché è ragionevolmente \textbf{conservato} in seguito alle esposizioni precedenti.

\subsection{Animali}
Negli animali i sintomi classici di carenza di acido pantotenico sono il \textbf{ritardo della crescita} e la \textbf{dermatite}, come conseguenza dell’\textbf{alterazione del metabolismo lipidico}.\\
Altre manifestazioni comprendono difetti nel piumaggio, degenerazione del midollo spinale e del fegato, sterilità, indebolimento delle risposte immunitarie, spasticità, anemia, neuropatie ed altre compromissioni.

\subsection{Uomo}
Ciò che si conosce della carenza di acido pantotenico nella specie umana proviene principalmente da due fonti.\\
In primo luogo, durante la seconda guerra mondiale, i prigionieri di guerra \textbf{malnutriti} in Giappone, Birmania e Filippine provarono \textbf{intorpidimento} e \textbf{sensazioni di bruciore nei loro piedi}.\\
Un secondo studio invece riguarda la carenza di acido pantotenico \textbf{indotta sperimentalmente}, sia negli animali, sia nell’uomo, attraverso la somministrazione di un \textbf{inibitore della chinasi dell’acido pantotenico}, il \textbf{v-metilmantotenato}, in combinazione con una dieta a basso contenuto di acido pantotenico.\\
I sintomi osservati nell’uomo comprendevano anche alcuni dei quadri descritti nei campioni animali.\\
È stato dimostrato che un altro antagonista dell'acido pantotenico, il \textbf{calcio Hopantenato}, induce \textbf{encefalopatia} con \textbf{steatosi} epatica e una \textbf{sindrome di tipo Reye} sia nei cani che negli umani.\\ Per quanto riguarda l’espressione temporale dei sintomi per la carenza dell’acido pantotenico nell’uomo, il cui fabbisogno giornaliero è di circa 5 mg, sono richieste approssimativamente \textbf{6 settimane} prima che si osservino chiari sintomi di carenza. Infatti una perdita giornaliera di 4-6 mg di acido pantotenico rappresenta una perdita dell’1\% - 2\% del pool corporeo nell’uomo.

\subsection{Patologie genetiche}
Esistono \textbf{polimorfismi} o \textbf{difetti genetici} negli enzimi coinvolti nella \textbf{via di sintesi del CoA} e che provocano stati patologici come la \textbf{sindrome di Hallervorden-Spatz} o la \textbf{neurodegenerazione associata a pantotenato chinasi}.\\
Quest'ultima patologia deriva da mutazioni in \textbf{PanK-II}, che è la forma più abbondantemente espressa nel \textbf{cervello} e localizzata nei mitocondri. Si tratta di un disturbo neurodegenerativo autosomico recessivo, caratterizzato clinicamente da \textbf{distonia} e \textbf{atrofia} ottica o retinopatia pigmentaria con \textbf{depositi di ferro} nei gangli della base e nel globus pallidus.

\subsection{Tossicità}
L’assunzione di acido pantotenico è \textbf{generalmente sicura e priva di rischi} anche a dosi consistenti. Gli eccessi sono infatti per lo più \textbf{escreti nelle urine}.\\
A dosi orali molto elevate (\textgreater 1 g/giorno) di acido pantotenico possono presentarsi \textbf{diarrea} e \textbf{disturbi gastrointestinali}.\\ Tuttavia, non sono note segnalazioni di tossicità acuta nell'uomo. In effetti, non sono disponibili dati che suggeriscano neurotossicità, cancerogenicità, genotossicità o tossicità riproduttiva.

\subsection{Possibili applicazioni terapeutiche}
Le richieste di acido pantotenico vanno dalla prevenzione e trattamento dei \textbf{capelli grigi} (basato sull’osservazione che la mancanza di acido pantotenico nei roditori provoca ingrigimento della pelliccia) all’utilizzo per migliorare le \textbf{prestazioni atletiche}.\\
Diversi studi inoltre hanno indicato che la pantetheina, in dosi che vanno da 500 a 1200 mg/giorno, può \textbf{abbassare il livello di colesterolo} nel siero.\\
La somministrazione orale di acido pantotenico e l’applicazione di un unguento al panthenolo alla pelle sembra \textbf{accelerare la cicatrizzazione delle ferite} della pelle e aumenti la forza del tessuto cicatriziale nei modelli animali.

\begin{thebibliography}{9}

\bibitem{erdman}
  John W. Erdman, Jr., Ian A. MacDonald, Steven H. Zeisel,
  \textit{Present Knowledge in Nutrition},
  Wiley-Blackwell,
  10\textsuperscript{th} edition,
  2012.
\bibitem{ball}
  George F. M. Ball,
  \textit{Vitamins. Their role in the human body.},
  Wiley-Blackwell,
  1\textsuperscript{st} edition,
  2004.
\bibitem{ball2}
  George F. M. Ball,
  \textit{Vitamins in foods. Analysis, bioavailability, and stability},
  Taylor & Francis,
  1\textsuperscript{st} edition,
  2006.
\bibitem{Vadlapudi}
  Vadlapudi, A. D., Vadlapatla, R. K., & Mitra, A. K.,
  \textit{Sodium Dependent Multivitamin Transporter (SMVT): A Potential Target for Drug Delivery. Current Drug Targets},
  13(7),
  994–1003,
  2012.
\bibitem{lehninger}
  David L. Nelson, Michael M. Cox,
  \textit{Lehninger principles of biochemistry},
  Freeman, W. H. & Company,
  6\textsuperscript{th} edition,
  2012.
\bibitem{devlin}
  Thomas M. Devlin,
  \textit{Biochimica con aspetti chimico-farmaceutici},
  EdiSES,
  7\textsuperscript{a} edizione,
  2011.

\bibitem{Linder}
  Maurine E. Linder et al.,
  \textit{Lipid modifications of G proteins: $\alpha$ subunits are palmitoylated},
  Department of Pharmacology, University of Texas Southwestern Medical Center, Dallas,
  1993.

\end{thebibliography}

\end{document}
