\documentclass[a4paper, 12pt]{article}

\usepackage[italian]{babel}
\usepackage[version=3]{mhchem}
\usepackage{textgreek}
\usepackage{gensymb}
\usepackage{chemfig}
\usepackage{tikz}
\usepackage{fullpage}
\usetikzlibrary{shapes,arrows, positioning,calc}

\tikzstyle{block} = [rectangle, draw, text centered, minimum height=1.5em]
\tikzstyle{line} = [draw, -latex']
\tikzstyle{linewithouttip} = [draw, -]
\tikzstyle{reverseline} = [draw, latex'-]
\tikzstyle{thickline} = [draw, -latex', line width = 0.4mm]

\usepackage[utf8]{inputenc}

\makeatletter
\definearrow5{-n>}{
	\CF@arrow@shift@nodes{#3}%
	\expandafter\draw\expandafter[\CF@arrow@current@style,-CF@full](\CF@arrow@start@node)--(\CF@arrow@end@node)node[pos=.2](narrow@arctangent@i){} node[pos=.5](narrow@arctangent@ii){};%
	\edef\CF@tmp@str{[\ifx\@empty#1\@empty draw=none,\fi-CF@full]}%
	\expandafter\draw\CF@tmp@str (narrow@arctangent@i)%
		arc[radius=\CF@compound@sep*\CF@current@arrow@length*\ifx\@empty#4\@empty0.333\else#4\fi,start angle=\CF@arrow@current@angle+90,%
		delta angle=-\ifx\@empty#5\@empty60\else#5\fi]node(narrow@start){};
	\edef\CF@tmp@str{[\ifx\@empty#2\@empty draw=none,\fi-CF@full]}%
	\expandafter\draw\CF@tmp@str (narrow@arctangent@ii)%
		arc[radius=\CF@compound@sep*\CF@current@arrow@length*\ifx\@empty#4\@empty0.333\else#4\fi,start angle=\CF@arrow@current@angle+90,%
		delta angle=-\ifx\@empty#5\@empty60\else#5\fi]node(narrow@end){};
	\edef\CF@tmp@str{\if\string-\expandafter\@car\detokenize{#4.}\@nil+\else-\fi}%
	\CF@arrow@display@label{#1}{0}\CF@tmp@str{narrow@start}{#2}{1}\CF@tmp@str{narrow@end}%
}
\makeatother

\newcommand*\derivesubmol[4]{%
    \saveexpandmode\saveexploremode\expandarg\exploregroups
    \csname @\ifcat\relax\noexpand#2first\else second\fi oftwo\endcsname
        {\expandafter\StrSubstitute\@car#2\@nil}
        {\expandafter\StrSubstitute\csname CF@@#2\endcsname}
    {\@empty#3}{\@empty#4}[\temp@]%
    \csname @\ifcat\relax\noexpand#1first\else second\fi oftwo\endcsname
        {\expandafter\let\@car#1\@nil}
        {\expandafter\let\csname CF@@#1\endcsname}\temp@
    \restoreexpandmode\restoreexploremode
}

\setcrambond{2pt}{}{}

\definesubmol{adenina}{N*5([::-18]-*6(-N=-N=(-NH_2)-)=-N=-)}

\definesubmol{s1}{-[2]!{adenina}}
\definesubmol{s2}{-[6,.6]O-PO_3^{2-}}
\definesubmol{s3}{-[6,.6]OH}

\definesubmol{ribosiofosfatoadenina}{%
    -[::-90,2]%
        (%
            -[::115,1.176]O%
            -[::-50,1.176]%
        )%
    <[::45,0.8]%
        (%
          (!{s3})
          -[::45,,,,line width=2pt,shorten <=-.5pt,shorten >=-.5pt]%
              (!{s2})%
          >[::45,0.8]%
              (!{s1})%
         )%
}

\definesubmol{ribosioadenina}{%
    -[::-90,2]%
        (%
            -[::115,1.176]O%
            -[::-50,1.176]%
        )%
    <[::45,0.8]%
        (%
          (!{s3})
          -[::45,,,,line width=2pt,shorten <=-.5pt,shorten >=-.5pt]%
          >[::45,0.8]%
              (!{s1})%
         )%
}

\def \CoA {
\tiny
\schemestart
$\underbrace{\chemfig{HS-CH_2-CH_2-NH-}}_{\beta\textnormal{-mercapto-etilammina}}$
$\underbrace{\chemfig{C(=[:90]O)-CH_2-CH_2-NH-C(=[:90]O)-CH(-[:270]OH)-C(-[:90]CH_3)(-[:270]CH_3)-CH_2-O}}_{\textnormal{acido pantotenico}}$
$\underbrace{\chemfig{PO_2^{-}-O-PO_2^{-}-O-CH_2!{ribosiofosfatoadenina}}}_{\textnormal{3'-fosfoadenosina difosfato}}$
\schemestop
}

\def \ACP {
\tiny
\schemestart
$\underbrace{\chemfig{HS-CH_2-CH_2-NH-C(=[:90]O)-CH_2-CH_2-NH-C(=[:90]O)-CH(-[:270]OH)-C(-[:90]CH_3)(-[:270]CH_3)-CH_2-O-PO_2^{-}-}}_{\textnormal{4'-fosfopanteteina}}$
$\underbrace{\chemleft. \chemfig{O-PO_2^{-}-O-CH_2-CH(-[:90]C(-[:90]...)=O)(-[:270]NH(-[:270]...))}\chemright\}}_{\textnormal{serina}}$
proteina
\schemestop
}

\date{}

\setatomsep{15pt}

\title{%
  Vitamina B5 \\
  \large Funzioni biologiche}

\begin{document}

\begin{titlepage}

\maketitle

\tableofcontents

\section{Introduzione}
La quinta vitamina del gruppo B, corrispondente all'acido pantonenico, ricopre numerose funzioni biologiche di fondamentale importanza in tutti gli esseri viventi.\\
Essa espleta il suo ruolo principale in quanto componente essenziale del \textbf{coenzima A}, molecola centrale del metabolismo.\\
Rientra inoltre nella \textbf{proteina di trasporto di acili (ACP)}, componente importante nell'anabolismo degli acidi grassi.

\end{titlepage}

\section{Coenzima A}
\begin{center}\CoA\end{center}
Il coenzima A è coinvolto in reazioni di trasferimento di gruppi acetilici e acilici, relativi al metabolismo ossidativo e al catabolismo.\\
Grazie al dominio adenosinico, CoA è in grado di legarsi agli enzimi che lo richiedono, mentre quello tioetanolamminico agisce nel legame dei substrati carboniosi e nel loro spostamento da un centro catalitico all'altro.\\
Il legame fra un gruppo acilico od acetilico, e quello tiolico del coenzima, porta alla formazione di un tioestere, rispettivamente \textbf{acil-CoA} o \textbf{acetil-CoA}. Esso è un \textbf{composto ad alta energia}, a causa della natura instabile del legame tioestereo, il cui $\Delta G'\degree$ di idrolisi permette lo svolgimento di numerose reazioni biochimiche.

\subsection{Sintesi}
L'acido pantotenico è essenziale per la sintesi dell'omonimo dominio presente in CoA.\\
La prima fase della sintesi prevede la fosforilazione ad \textbf{acido 4'-fosfopantotenico}, mediata dalla \textbf{chinasi dell'acido pantotenico} (PanK). Gli eucarioti possiedono PanK-II, mentre le varianti I e III sono proprie dei procarioti.\\
PanK agisce in un ampio intervallo di pH (fra 6 e 9) e lega: \begin{itemize}
\item \textbf{acido pantotenico} con \textit{K\textsubscript{m}} $\simeq$ 20 {\textmu}M, in qualità di accettore di fosfato
\item  \textbf{Mg-ATP} con \textit{K\textsubscript{m}} $\simeq$ 0.6 {\textmu}M, in qualità di donatore di fosfato
\end{itemize}
La fosforilazione dell'acido pantotenico è un fondamentale punto di controllo della sintesi del coenzima A. Infatti PanK sono regolate da:
\begin{itemize}
\item \textbf{vari anioni}, che attivano o inibiscono non specificamente l'enzima
\item \textbf{CoA e suoi derivati}, che inibiscono la sintesi di nuovo coenzima con un meccanismo a \textbf{feedback negativo}
\item \textbf{carnitina}, amminoacido trasportatore di acidi grassi nel mitocondrio, che indirettamente attiva l'enzima bloccando l'inibizione da parte dei derivati di CoA
\end{itemize}
Per la compartecipazione di questi fattori è stato ad esempio verificato che digiuno e diabete di tipo I (da ipoinsulinemia) incrementano l'attività di PanK e dunque la quantità di CoA. Al contrario, eccesso di glucosio e di acidi grassi all'interno della cellula riducono l'attività di PanK, per sottrazione di carnitina e maggiori concentrazioni di acetil-CoA.\\
Le successive fasi della sintesi del coenzima sono condotte su un complesso proteico dotato di vari siti catalitici e sono illustrate nello schema sottostante.\\
La \textbf{coenzima A idrolasi} catalizza l'idrolisi di CoA a 3'-5'-ADP e 4'-fosfopanteteina; quest'ultima può essere riutilizzata per la sintesi di nuovo coenzima. Si parla perciò di \textbf{ciclo del CoA/4'-fosfopanteteina}. Ogni esecuzione del ciclo richiede due molecole di ATP e ne produce una di ADP, una di PP\textsubscript{i} e una di 3',5'-ADP.
\begin{center}
\begin{tikzpicture}[node distance = 2cm, trim left = (1), trim right = (1)]

    \node (1) [block]{Acido pantotenico};
    \node (2) [block, below of=1]{Acido 4'-fosfopantotenico};
    \node (3) [block, below of=2]{4'-fosfopantotenilcisteina};
    \node (4) [block, below of=3]{4'-fosfopantoteina};
    \node (5) [block, below of=4]{4'-defosfo-CoA};
    \node (6) [block, below of=5]{Coenzima A};

    \path [line] (1) -- (2) node (7) [midway, left, xshift = -2mm] {\textit{PanK}};
    \path [line] (2) -- (3) node [midway, left, xshift = -2mm, align = right, text width = 5cm] {\textit{4'-fosfopantotenilcistina sintetasi}};
    \path [line] (3) -- (4) node [midway, left, xshift = -2mm, align = right, text width = 5cm] {\textit{4'-fosfopantotenilcistina decarbossilasi}};
    \path [line] (4) -- (5) node [midway, left, xshift = -2mm, align = right, text width = 5cm] {\textit{Defosfo-CoA pirofosforilasi}};
    \path [line] (5) -- (6) node [midway, left, xshift = -2mm, align = right, text width = 5cm] {\textit{Defosfo-CoA chinasi}};
    \path [dashed, line] (6) --++ (-6,-0) |- (7) node [midway, left] {$\ominus$};
    \path [line] (6) --++ (3,0) |- (4) node[midway, right]{\textit{CoA idrolasi}}

\end{tikzpicture}
\end{center}

\subsection{Acilazione}
\begin{center}
\setatomsep{30pt}
\footnotesize
\schemestart
ATP \+
\chemname{\chemfig{R-C(=[:90]O)-O^{-}}}{Gruppo acile} \+
\chemname{\chemfig{CoA-SH}}{Tiolo}
\arrow(.mid east--.mid west){<=>} \chemname{\chemfig{R-C(=[:90]O)-S-CoA}}{Tioestere} \+
AMP \+
PP\textsubscript{i}
\schemestop
\end{center}
La coniugazione di un gruppo acilico a CoA richiede consumo di energia, necessaria ad \textit{attivare} l'acido grasso per renderlo disponibile all'attacco nucleofilo sul coenzima.

\begin{center}
\setatomsep{20pt}
\tiny
\schemestart
\chemname{\chemfig{PO_3^{2-}-O-PO_2^{-}-O-PO_2^{-}-O-[:0, 1.5]{Ribosio}-[:0, 2]{Adenina}}}{ATP} \+
\chemname{\chemfig{R-C(=[:90]O)-O^{-}}}{Gruppo acile}
\arrow{-U>[][*{0}\schemestart PP\textsubscript{i}\arrow 2P\textsubscript{i}\schemestop]}[-90]
\chemname{\chemfig{R-C(=[:90]O)-O-PO_2^{-}-[:0, 1.5]{Ribosio}-[:0, 2]{Adenina}}}{Acil-adenilato}
\arrow{-U>[*{0}CoA-SH]}[-90]
\chemname{\chemfig{PO_3^{2-}-O-[:0, 1.5]{Ribosio}-[:0, 2]{Adenina}}}{AMP} \+
\chemname{\chemfig{R-C(=[:90]O)-S-CoA}}{Acil-CoA}
\schemestop
\end{center}

Il tioestere ottenuto è un composto \textbf{ad alta energia} poichè, a differenza dei normali esteri, non può stabilizzarsi per risonanza del doppio legame fra i due atomi di ossigeno. Infatti l'atomo di zolfo, appartenente al terzo periodo, è più ingombrante e meno elettronegativo dell'ossigeno, e per questo il legame \chemfig{C-S} è più lungo di quello \chemfig{C-O} e la sovrapposizione con gli orbitali dell'ossigeno peggiore.\\
A causa di tale instabilità, l'idrolisi del tiostere fornisce una variazione di energia libera significativamente negativa. Ad esempio, dall'idrolisi del tioestere dell'acido palmitico:
\begin{center}
$\Delta G'\degree = -32.5$ kJ/mol di palmitoil-CoA
\end{center}

\subsection{Metabolismo dei carboidrati}
\subsubsection{Anabolismo}
CoA ha un ruolo fondamentale nel ciclo degli acidi tricarbossilici. Esso infatti partecipa a:
\begin{itemize}
\item \textbf{decarbossilazione del piruvato}, proveniente dal metabolismo glicolitico dei carboidrati, con la formazione di \textbf{acetil-CoA}.\\ Quest'ultimo è il punto di ingresso del ciclo, in quanto reagisce con ossalacetato per formare \textbf{acido citrico}. La reazione è catalizzata dall'enzima piruvato-deidrogenasi (\textbf{PDH}).
\begin{center}
\tiny
\setatomsep{15pt}
\setcompoundsep{10em}
\schemestart
\chemname{\chemfig{CH_3(-[:90]C(-[:90]C(-[:45]O^{-})(=[:135]O))=O)}}{Piruvato}
\arrow(.mid east--.mid west){-U>[CoA-SH + NAD\textsuperscript{+}][NADH][][0.2]}[,2]
\chemname{\chemfig{CH_3(-[:90]C(-[:90]C(-[:45]S-CoA)(=[:135]O))=O)}}{Acetil-CoA}
\+
\chemfig{CO_2}
\schemestop
\end{center}
\begin{center}\tiny$\Delta G'\degree = -33.4$ kJ/mol\end{center}
\item \textbf{decarbossilazione di $\alpha$-chetoglutarato}, con formazione di \textbf{succinil-CoA}.\\ Esso, oltre ad essere convertito in succinato nella successiva tappa del ciclo, può reagire con la glicina per formare acido $\delta$-aminolevulinico, precursore del \textbf{gruppo eme}. Da ciò deriva l'importanza della vitamina B5 per la corretta sintesi di \textbf{emoglobina}, e dunque per il trasporto di ossigeno, e dei \textbf{citocromi}, per quello di elettroni.
\begin{center}
\tiny
\setatomsep{15pt}
\setcompoundsep{10em}
\schemestart
\chemname{\chemfig{COO^{-}(-[:90]C(-[:90]CH_2(-[:90]CH_2-COO^{-}))=O)}}{$\alpha$-chetoglutarato}
\arrow(.mid east--.mid west){-U>[CoA-SH + NAD\textsuperscript{+}][NADH][][0.2]}[,2]
\chemname{\chemfig{O(=[:90]C((-S-CoA)(-[:90]CH_2(-[:90]CH_2-COO^{-}))))}}{Succinil-CoA}
\+
\chemfig{CO_2}
\schemestop
\end{center}
\begin{center}\tiny$\Delta G'\degree = -33.5$ kJ/mol\end{center}
\end{itemize}

\subsection{Metabolismo dei lipidi}
\subsubsection{Catabolismo}
Il coenzima A è richiesto per due reazione del ciclo della \textbf{$\beta$-ossidazione} degli acidi grassi, durante cui due unità carboniose sono rimosse per ciascun ciclo, formando \textbf{acetil-CoA}.
\subsubsection{Anabolismo}
CoA partecipa alla \textbf{via metabolica dell'acido mevalonico}, che inizia con la condensazione di due molecole di acetil-CoA formando acetoacetil-CoA. Esso reagisce poi con una terza unità di acetil-CoA, dando luogo all'\textbf{acido mevalonico}.\\
L'acido mevalonico è il precursore degli \textbf{isoprenoidi}, e dunque anche degli \textbf{steroidi} attraverso lo squalene.
È quindi chiara l'importanza del coenzima per la sintesi di colesterolo, ormoni steroidei e altri lipidi, e per la modificazione di proteine mediante isoprenilazione. \\
La produzione di acido mevalonico è controllata mediante un meccanismo di \textbf{feedback negativo}: un eccesso di colesterolo all'interno della cellula, infatti, oltre a ridurre l'espressione del recettore per LDL, è in grado di inibire due enzimi impiegati nella via, \textbf{HMG-CoA sintasi} e \textbf{HMG-CoA reduttasi}.

\begin{center}
\begin{tikzpicture}[node distance = 2cm, auto, trim left = (1), trim right = (1)]

    \node (1) [block]{acetil-CoA + acetoacetil-CoA};
    \node (2) [block, below of=1]{HMG-CoA};
    \node (3) [block, below of=2]{acido mevalonico};
    \node (4) [block, below of=3]{colesterolo};
    \node (5) [below of=4]{LDL plasmatiche};
    \node (6) [right of=4, node distance = 5cm]{recettore LDL};

    \path [line] (1) -- (2) node (7) [midway, left, xshift = -2mm] {\textit{HMG-CoA sintasi}};
    \path [line] (2) -- (3) node (8) [midway, left, xshift = -2mm] {\textit{HMG-CoA reduttasi}};
    \path [dashed, thickline] (3) -- (4);
    \path [reverseline] (4) -- (5);
    \path [dashed, line] (4) --++ (-5,-0) |- (7) node [midway, left] {$\ominus$};
    \path [dashed, line] (4) --++ (-5,0) |- (8) node [midway, left] {$\ominus$};
    \path [dashed, line] (4) -- (6) node [very near end]{$\ominus$};

\end{tikzpicture}
\end{center}

\subsection{Metabolismo degli amminoacidi}
CoA rientra nel processamento della \textbf{leucina}, in quanto il suo chetoacido, ottenuto per deaminazione, reagisce con il coenzima formando acido acetoacetico e acetil-CoA.

\section {Proteina trasportatrice di acili}
\begin{center}\ACP\end{center}
\textbf{ACP} fa parte del complesso della \textbf{sintasi degli acidi grassi}, ed è quindi coinvolta nella biosintesi di tali composti.\\
Oltre ad introdurli con la dieta, infatti, buona parte dei lipidi derivano dal metabolismo dei carboidrati, convertiti in \textbf{piruvato} durante la glicolisi.\\
Il piruvato, prodotto nel citoplasma, diffonde passivamente nella matrice mitocondriale. In essa viene ossidato ad acetil-CoA, materiale di partenza per la sintesi di acidi grassi.\\
Acetil-CoA viene esportato dal mitocondrio come citrato, e si riforma nel citoplasma, dove avviene la sintesi lipidica.\\
Il complesso enzimatico della sintasi prevede due subunità identiche, ciascuna contenente gli enzimi necessari alla biosintesi. Esse sono attive solamente quando si combinano, con orientamento antiparallelo, nel formare un \textbf{omodimero}.

\subsection{Metabolismo dei lipidi}
\subsubsection{Anabolismo}
Acetil-CoA viene carbossilato, in una reazione dipendente da biotina come donatore, in \textbf{malonil-CoA}.\\
Il gruppo acilico di malonil-CoA viene successivamente ceduto all'\textbf{estremità tiolica} di ACP, con formazione di malonil-ACP. Esso reagisce con acetil-ACP, analogamente derivato da acetil-CoA, per formare \textbf{acetoacetil-ACP}.\\
Questo composto, dopo riduzione e deidratazione a butirril-ACP, reagisce nuovamente con malonil-ACP. Il composto così ottenuto possiede sei atomi di carbonio: quattro dati da butirril-ACP, due da malonil-ACP. Successive reazioni con malonil-ACP permettono l'elongazione del composto di due atomi di carbonio ad ogni ciclo, fino a produrre \textbf{acido palmitico} (16:0).\\
L'acido palmitico può essere ulteriormente allungato o desaturato, per produrre gli altri acidi grassi sintetizzabili dall'organismo.

\begin{center}
\begin{tikzpicture}[auto, trim left = (5), trim right = (5)]

    \node (1) [block]{acetil-CoA};
    \node (2) [block, right = 6cm of 1]{malonil-CoA};
    \node (3) [block, below = 1cm of 1]{acetil-ACP};
    \node (4) [block, below = 1cm of 2]{malonil-ACP};
    \node (5) [block, below = 1cm of {$(3)!0.5!(4)$}]{acetoacetil-ACP};
    \node (6) [block, below = 1cm of 5]{butirril-ACP};
    \node (7) [block, below = 1cm of 6]{C6-ACP};
    \node (8) [block, below = 1cm of 7]{C8-, C10-, Cn-ACP};
    \node (9) [block, below = 1cm of 8]{palmitil-ACP (C16)};

    \path [line] (1) -- (2);
    \path [line] (2) -- (4);
    \path [line] (1) -- (3);
    \path [line] (3) -- (5);
    \path [line] (4) -- (5);
    \path [line] (5) -- (6);
    \path [line] (6) -- (7) node (11) [midway,right];
    \path [line] (7) -- (8) node (12) [midway, right];
    \path [dashed, line] (8) -- (9);

    \path [line] (4) --++ (2,-0) |- (11);
    \path [line] (4) --++ (2,-0) |- (12);

\end{tikzpicture}
\end{center}

\section{Modificazione delle proteine}
Oltre agli effetti sul metabolismo finora descritti, un vasto campo di attività del coenzima A, e dunque dell'acido pantotenico, è l'alterazione delle proteine allo scopo di modificarne attività e altre proprietà.\\
Le variazioni strutturali delle proteine possono ad esempio determinarne la \textbf{collocazione nelle membrane} plasmatiche o in quelle intracellulari, le \textbf{interazioni} con altre proteine e l'\textbf{indirizzamento} a specifici organelli o strutture.\\
Molte proteine subiscono l'\textbf{aggiunta covalente di unità carboniose fornite da CoA}, in qualità di donatore, oppure grazie ad esso sintetizzate.\\
Le tre principali categorie di modificazioni richiedenti CoA sono:
\begin{itemize}
\item \textbf{acilazione}: aggiunta di radicali acilici di lunghezza variabile.\\ I due acidi grassi a lunga catena più comunemente addizionati come gruppi acilici sono quello \textbf{miristico} (14:0), su residui di glicina, e quello \textbf{palmitico}, sulla catena laterale di residui di cisteina.\\
La \textbf{palmitoilazione} si realizza mediante un legame tioestereo \textbf{instabile}, di facile idrolisi. Sono perciò possibili cicli di palmitoilazione e depalmitoilazione adatti a regolare la funzionalità di una proteina a seconda delle esigenze. Interessa ad esempio la subunità $\alpha$ delle proteine G, recettori di membrana, proteina del citoscheletro, di gap junction e neuronali, ed enzimi come l'acetilcolina-esterasi. È inoltre necessaria al distacco delle vescicole dalle cisterne del Golgi.\\
La \textbf{miristoilazione} invece prevede la formazione di un legame ammidico \textbf{stabile}. Anch'essa è applicata alle subunità $\alpha$ e ad enzimi, oltre che a fattori di ribosilazione di ADP, chinasi e proteine del sistema immunitario e alla recuperina, coinvolta nel ripristino dell'eccitabilità della visione.
\item \textbf{acetilazione}: particolare acilazione in cui la catena R è un semplice metile.\\ È frequente all'estremità amminica di proteine solubili al fine di alterarne l'affinità per recettori o altre proteine.
\item \textbf{prenilazione}: aggiunta di catene isopreniche.
Avviene ad esempio su residui di cisteina di motivi CAAX (cisteina, amminoacido alifatico e residuo C-terminale), per aggiunta di \textbf{farnesile} o \textbf{geranilgeranile}.\\
Sono soggette a prenilazione le proteine \textbf{Ras}, coinvolte nella trasduzione del segnale, le \textbf{Rab}, che regolano il traffico vescicolare, le lamìne nucleari, la subunità $\gamma$ delle proteine G e varie chinasi.
\end{itemize}
Esse possono essere \textbf{co-traduzionali}, ovvero attuate durante la sintesi del peptide, oppure \textbf{post-traduzionali}.

\subsection{Acetilazione di $\boldsymbol\beta$-endorfina}
Il \textbf{neurotrasmettitore} peptidico cerebrale $\beta$-endorfina determina effetto \textbf{analgesico} e influenza apprendimento, attitudine e attività sessuale. Subisce acetilazione post-traduzionale all'estremità ammino-terminale, che lo disattiva per impossibilità di legare i propri recettori.

\subsection{Acetilazione degli istoni}
Gli istoni \textbf{legano il DNA} determinandone la configurazione eu- o eterocromatinica. L'acetilazione dei residui lisinici di queste proteine in specifiche regioni di cromatina \textbf{riduce la loro affinità per il DNA}, permettendo l'assunzione di una forma maggiormente rilassata e dunque favorendo la trascrizione del tratto di genoma corrispondente.\\
Il processo è catalizzato dall'enzima \textbf{istone-acetiltransferasi} (HAC), che sfrutta acetil-CoA come donatore di acetile, ed è reversibile grazie all'intervento di \textbf{istone-deacetilasi} (HDAC).\\
L'acetilazione istonica reversibile permette alla cellula di trascrivere all'occorenza i geni richiesti per la propria attività, e di \textit{spegnerli} quando non più necessari.

\subsection{Acetilazione di $\boldsymbol\alpha$-tubulina}
I microtubuli, costituenti del citoscheletro, sono eterodimeri di $\alpha$ e $\beta$-tubulina.\\
L'acetilazione del gruppo amminico di specifici residui di lisina di $\alpha$-tubilina \textbf{stabilizza il microtubulo}; al contrario, la deacetilazione ne favorisce la depolimerizzazione.

\subsection{Acilazione di proteine G}
La \textbf{subunità} $\boldsymbol\alpha$ delle proteine G è soggetta a \textbf{palmitoilazione} reversibile catalizzata da palmitoil-transferasi, che preleva un palmitato fornito da palmitoil-CoA. Tale modifica media la \textbf{traslocazione della subunità nella membrana plasmatica}, mentre la reversione ne determina il ritorno nel citoplasma.\\
$\alpha$ di \textbf{G\textsubscript{i}} subisce inoltre l'aggiunta permanente di \textbf{acido miristico}. Esso consolida la localizzazione della subunità in membrana, accrescendo l'affinità per la componente $\boldsymbol{\beta\gamma}$ della proteina, e promuove l'interazione, di significato inibitorio, con l'\textbf{adenilato ciclasi}.

\begin{thebibliography}{9}

\bibitem{erdman}
  John W. Erdman, Jr., Ian A. MacDonald, Steven H. Zeisel,
  \textit{Present Knowledge in Nutrition},
  Wiley-Blackwell,
  10\textsuperscript{th} edition,
  2012.
\bibitem{ball}
  George F. M. Ball,
  \textit{Vitamins. Their role in the human body.},
  Wiley-Blackwell,
  1\textsuperscript{st} edition,
  2004.
\bibitem{ball2}
  George F. M. Ball,
  \textit{Vitamins in foods. Analysis, bioavailability, and stability},
  Taylor & Francis,
  1\textsuperscript{st} edition,
  2006.
\bibitem{lehninger}
  David L Nelson, Michael M Cox,
  \textit{Lehninger principles of biochemistry},
  Freeman, W. H. & Company,
  6\textsuperscript{th} edition,
  2012.
\bibitem{devlin}
  Thomas M. Devlin,
  \textit{Biochimica con aspetti chimico-farmaceutici},
  EdiSES,
  7\textsuperscript{a} edizione,
  2011.

\end{thebibliography}

\end{document}
