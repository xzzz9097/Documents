\documentclass[a4paper, 12pt]{article}

\usepackage[italian]{babel}
\usepackage[version=3]{mhchem}
\usepackage{textgreek}
\usepackage{gensymb}
\usepackage{chemfig}
\usepackage{tikz}
\usepackage{fullpage}
\usetikzlibrary{shapes,arrows, positioning,calc}

\tikzstyle{block} = [rectangle, draw, text centered, minimum height=1.5em]
\tikzstyle{line} = [draw, -latex']
\tikzstyle{linewithouttip} = [draw, -]
\tikzstyle{reverseline} = [draw, latex'-]
\tikzstyle{thickline} = [draw, -latex', line width = 0.4mm]

\usepackage[utf8]{inputenc}

\makeatletter
\definearrow5{-n>}{
	\CF@arrow@shift@nodes{#3}%
	\expandafter\draw\expandafter[\CF@arrow@current@style,-CF@full](\CF@arrow@start@node)--(\CF@arrow@end@node)node[pos=.2](narrow@arctangent@i){} node[pos=.5](narrow@arctangent@ii){};%
	\edef\CF@tmp@str{[\ifx\@empty#1\@empty draw=none,\fi-CF@full]}%
	\expandafter\draw\CF@tmp@str (narrow@arctangent@i)%
		arc[radius=\CF@compound@sep*\CF@current@arrow@length*\ifx\@empty#4\@empty0.333\else#4\fi,start angle=\CF@arrow@current@angle+90,%
		delta angle=-\ifx\@empty#5\@empty60\else#5\fi]node(narrow@start){};
	\edef\CF@tmp@str{[\ifx\@empty#2\@empty draw=none,\fi-CF@full]}%
	\expandafter\draw\CF@tmp@str (narrow@arctangent@ii)%
		arc[radius=\CF@compound@sep*\CF@current@arrow@length*\ifx\@empty#4\@empty0.333\else#4\fi,start angle=\CF@arrow@current@angle+90,%
		delta angle=-\ifx\@empty#5\@empty60\else#5\fi]node(narrow@end){};
	\edef\CF@tmp@str{\if\string-\expandafter\@car\detokenize{#4.}\@nil+\else-\fi}%
	\CF@arrow@display@label{#1}{0}\CF@tmp@str{narrow@start}{#2}{1}\CF@tmp@str{narrow@end}%
}
\makeatother

\newcommand*\derivesubmol[4]{%
    \saveexpandmode\saveexploremode\expandarg\exploregroups
    \csname @\ifcat\relax\noexpand#2first\else second\fi oftwo\endcsname
        {\expandafter\StrSubstitute\@car#2\@nil}
        {\expandafter\StrSubstitute\csname CF@@#2\endcsname}
    {\@empty#3}{\@empty#4}[\temp@]%
    \csname @\ifcat\relax\noexpand#1first\else second\fi oftwo\endcsname
        {\expandafter\let\@car#1\@nil}
        {\expandafter\let\csname CF@@#1\endcsname}\temp@
    \restoreexpandmode\restoreexploremode
}

\setcrambond{2pt}{}{}

\definesubmol{adenina}{N*5([::-18]-*6(-N=-N=(-NH_2)-)=-N=-)}

\definesubmol{s1}{-[2]!{adenina}}
\definesubmol{s2}{-[6,.6]O-PO_3^{2-}}
\definesubmol{s3}{-[6,.6]OH}

\definesubmol{ribosiofosfatoadenina}{%
    -[::-90,2]%
        (%
            -[::115,1.176]O%
            -[::-50,1.176]%
        )%
    <[::45,0.8]%
        (%
          (!{s3})
          -[::45,,,,line width=2pt,shorten <=-.5pt,shorten >=-.5pt]%
              (!{s2})%
          >[::45,0.8]%
              (!{s1})%
         )%
}

\definesubmol{ribosioadenina}{%
    -[::-90,2]%
        (%
            -[::115,1.176]O%
            -[::-50,1.176]%
        )%
    <[::45,0.8]%
        (%
          (!{s3})
          -[::45,,,,line width=2pt,shorten <=-.5pt,shorten >=-.5pt]%
          >[::45,0.8]%
              (!{s1})%
         )%
}

\def \Pantotenato {
  \tiny
  \chemfig{HO-C(=[:90]O)-CH_2-CH_2-NH-C(=[:90]O)-CH(-[:270]OH)-C(-[:90]CH_3)(-[:270]CH_3)-CH_2-OH}
}

\def \Colesterolo {
  \chemfig{[::30]HO-*6(--*6(=--*6(-*5(---(-[::-36](-[::+60]) -[::-60]-[::-60]-[::+60]-[::-60](-[::-60])-[::+60])-)-(-[::+0])---)--)-(-[::+0])---)}
}

\def \Statina {
  \chemname{\chemfig{[:-30]-[::60]-(-[::-60])(-[::+120]R_1)-[::+60](=[::+60]O)-[::-60]O-[::-60]*6(--(-R_2)-=*6(-=-(-)-*6(-*6(--*6(-OH)(-[::+60]-[::-60](-[::+60]HO)(-[::-60]-[::-60]COO^{-}))))-)--)}}{Statina}
}

\def \Mevalonato {
  \chemname{\chemfig{[:-180]OH*6(--(-H_{2}C)(-[:-150]HO)---COO^{-})}}{Mevalonato}
}

\def \CoA {
\tiny
\schemestart
$\underbrace{\chemfig{HS-CH_2-CH_2-NH-}}_{\beta\textnormal{-mercapto-etilammina}}$
$\underbrace{\chemfig{C(=[:90]O)-CH_2-CH_2-NH-C(=[:90]O)-CH(-[:270]OH)-C(-[:90]CH_3)(-[:270]CH_3)-CH_2-O}}_{\textnormal{acido pantotenico}}$
$\underbrace{\chemfig{PO_2^{-}-O-PO_2^{-}-O-CH_2!{ribosiofosfatoadenina}}}_{\textnormal{3'-fosfoadenosina difosfato}}$
\schemestop
}

\def \ACP {
\tiny
\schemestart
$\underbrace{\chemfig{HS-CH_2-CH_2-NH-C(=[:90]O)-CH_2-CH_2-NH-C(=[:90]O)-CH(-[:270]OH)-C(-[:90]CH_3)(-[:270]CH_3)-CH_2-O-PO_2^{-}-}}_{\textnormal{4'-fosfopanteteina}}$
$\underbrace{\chemleft. \chemfig{O-CH_2-CH(-[:90]C(-[:90]...)=O)(-[:270]NH(-[:270]...))}\chemright\}}_{\textnormal{serina}}$
proteina
\schemestop
}

\date{}

\setatomsep{15pt}

\title{%
  Ipercolesterolemia \\
  \large Alterazioni del metabolismo del colesterolo
}

\begin{document}

\begin{titlepage}

\maketitle

\begin{center}{\setatomsep{20pt}\Colesterolo}\end{center}

\tableofcontents

\section{Introduzione}
L'\textit{ipercolesterolemia} è genericamente definita come un \textbf{eccesso di colesterolo nel sangue}.\\
Quella a base genetica si distingue in:
\begin{itemize}
\item \textbf{poligenica} (85\% dei casi)
\item \textbf{monogenica}, detta anche ipercolesterolemia familiare
\item \textbf{primitiva}
\end{itemize}

\section{Descrizione}

\subsection{Ipercolesterolemia familiare}
L’ipercolesterolemia \textit{familiare} è una malattia genetica \textbf{autosomica dominante} dovuta ad alterazioni (se ne conoscono oltre 620) del gene codificante per il \textbf{recettore delle LDL}.\\
Si tratta di una proteina localizzata sulla \textbf{superficie} delle cellule, che è in grado di \textbf{captare le LDL del sangue} e di farle entrare nella cellula, dove vengono \textbf{scomposte}.\\
Le \textbf{LDL} sono lipoproteine a \textbf{bassa densità} che originano dalle \textbf{IDL} (lipoproteine a densità intermedia, dette anche \textit{rimanenze di VLDL}) con funzione di \textbf{trasporto di lipidi ai tessuti periferici}.\\
Le lipoproteine sono i \textbf{trasportatori plasmatici dei lipidi}, aggregati macromolecolari dotati di specifiche proteine trasportatrici, chiamate \textbf{apolipoproteine}.\\ Le componenti proteiche agiscono come segnali che \textbf{indirizzano le lipoproteine} verso tessuti specifici oppure che \textbf{attivano enzimi} che poi agiscono sulle stesse lipoproteine.\\
In base al genotipo, si distinguono due forme di ipercolesterolemia:
\begin{itemize}
\item \textbf{eterozigote}, caratterizzata dalla presenza di \textbf{una copia normale} del gene ereditata dal genitore sano, e di \textbf{una sola malata}, proveniente dal genitore affetto. Il recettore è quindi presente ma è prodotto in \textbf{quantità insufficiente} (50\%).
\item \textbf{omozigote}, nella quale \textbf{entrambe le copie del gene sono alterate}, e il recettore è del tutto assente o \textbf{molto ridotto} (da 0 a 25\%).
\end{itemize}

La frequenza degli individui \textbf{eterozigoti} è piuttosto \textbf{alta} (1/200-1/250), dato che identifica l’ipercolesterolemia familiare come uno dei più frequenti errori congeniti del metabolismo.
\\
La frequenza degli \textbf{omozigoti} è invece molto più \textbf{bassa} (1/300.000-1/1.000.000).\\
In Italia si stima che esistano circa 50-60 individui affetti dalla forma omozigote, dei quali solo una quindicina conosciuti e attualmente curati.

\subsection{Sintomi}
Un sintomo tipico è la comparsa (dopo i trenta-quarant’anni negli eterozigoti e entro i primi quattro anni di vita negli omozigoti) di xantomi a livello dei tendini negli eterozigoti (xantomi tendinei) e a livello della cute dei gomiti, delle ginocchia e tra le dita negli omozigoti (xantomi cutanei). Negli eterozigoti gli xantomi tendinei possono essere assenti per tutta la vita (circa 40\% dei casi). Altri sintomi spesso presenti negli eterozigoti sono, inoltre,  gli xantelasmi e l’arco corneale. Tuttavia questo sintomo viene considerato solo per individui relativamente giovani(dai 45 anni in giù), dal momento che l’arco corneale compare nella popolazione generale in relazione all’invecchiamento.
Molti eterozigoti mostrano un livello di colesterolo nel sangue all’incirca doppio del normale (350-550 mg/dl). Questo aumento è riscontrabile sin dalla nascita. Verso i 35 anni nel maschi e i 45 nelle femmine, in assenza di trattamento, possono comparire problemi cardiovascolari. Infatti elevati livelli di colsterolo LDL (LDL-C) già in giovane età causano la deposizione di placche aterosclerotiche nelle arterie coronarie e nell’aorta prossimale, aumentando il rischio di sviluppare malattie cardiovascolari. Spesso l’ipercolesterolemia familiare è correlata alla CAD (coronary artery disease), che può manifestarsi con angina, infarto del miocardio e più raramente con ictus. È importante però notare che questi sintomi possono variare molto da un individuo all’altro. Alcuni individui eterozigoti possono anche non accorgersi della loro condizione e rendersene conto dopo un controllo casuale. Altri presentano sintomi talmente lievi da non essere praticamente riconosciuti. 
Gli omozigoti hanno un quadro clinico più grave rispetto agli eterozigoti. L’ipercolesterolemia è molto grave (550 - 1000 mg/dl) e anch’essi hanno xantomi, che però sono visibili a livello cutaneo e sono riscontrabili entro i primi 4 anni di vita. I problemi cardiovascolari iniziano già nell’infanzia, e in assenza di trattamento, la maggior parte degli individui colpiti presenta una grave CAD( e talvolta anche una seria stenosi aortica) entro i 20 anni, muore prematuramente o riesce a sopravvivere grazie al bypass coronarico. Anche per gli omozigoti esiste però una variabilità di sintomi. Esistono casi di omozigoti \textit{lievi}, i cui sintomi somigliano molto più a quelli degli eterozigoti, e che sopravvivono fino ad età avanzata.

\section{Biochimica}
Il colesterolo è un importante lipide steroideo del nostro organismo.
E’ un componente fondamentale delle membrane biologiche nonché precursore di tutti gli ormoni steroidei, dei sali biliari e della vitamina D.
Esso è essenziale per molti animali, compreso l’uomo, ma non è richiesto nella dieta dei mammiferi, perché tutte le cellule possono sintetizzarlo a partire da precursori semplici. I 27 atomi di carbonio che compongono il colesterolo derivano da unità bicarboniose dell’acetato, substrato di partenza per la sintesi isoprenica  o terpenica.
La sintesi isoprenica rappresenta uno dei meccanismi più importanti ed efficaci per sintetizzare scheletri carboniosi. La prima fase ha luogo nel citosol e produce unità C5 isopreniche attivate partendo dall’acetato e  passando per il mevalonato. La seconda fase conduce allo squalene (terpene a 30 atomi di carbonio) mediante la sintesi isoprenica vera e propria che avviene sulle membrane del reticolo endoplasmatico. La terza fase prevede la ciclizzazione dello squalene per originare il nucleo steroideo.
Tappa 1): due molecole di acetil-CoA condensano formando acetoacetil-CoA per generare il composto a sei atomi di carbonio beta-idrossi-beta-metilglutaril-CoA (HMG-CoA). Queste due reazioni sono catalizzate dalla acetil-CoA acetiltransferasi e dalla HMG-CoA sintasi. Avviene poi la riduzione dell’HMG CoA a mevalonato, in cui due molecole di NADPH donano due eletroni ciascuna.
2) tre gruppi fosfato vengono trasferiti da tre molecole di ATP al mevalonato. Il gruppo fosfato  e il gruppo carbossilico vicino vengono liberati generando un doppio legame nel prodotto a 5 atomi di carbonio, delta3-isopentenil pirofosfato. L’isomerizzazione di quest’ultima genera una seconda unità isoprenica attivata, il dimetilallil pirofosfato.
3) l’isopentenil pirofosfato e il dimetilallil pirofosfato vanno incontro ad una condensazione del tipo “testa coda”, in cui si forma una catena di 10 atomi di carbonio, il geranil pirofosfato. Il geranil pirofosfato subisce un’altra condensazione testa coda in cui viene prodotto un intermedio  a 15C , il farnesil pirofosfato. Infine due molecole di farnesil pirofosfato si uniscono testa testa formando lo squalene
4) la squalene monossigenasi aggiunge un atomo di ossigeno prelevandolo dall’O2 all’estremità del la molecola di squalene, formando un epossido. I doppi legami di questa reazione (squalene 2,3-epossido)sono posizionati in modo tale che si formi una struttura ciclica.

\section{Processo sano}
Le LDL fisiologicamente entrano nelle cellule bersaglio attraverso l’endocitosi mediata da recettori. LDLr è una glicoproteina a singola catena dotata di:
\begin{itemize}
  \item una sola elica \textbf{transmembrana}
  \item un'estremità \textbf{C-terminale} citoplasmatica
  \item un'estremità \textbf{N-terminale} protesa nell'ambiente cellulare
\end{itemize}
L'estremità N-terminale presenta i siti di legame per \textbf{apoB-100} e \textbf{apoE-100}, le due apolipoproteine espresse dalle \textbf{LDL} e VLDL che ne mediano il riconoscimento e la \textbf{internalizzazione} da parte delle cellule epatiche e periferiche dotate del recettore.\\
I recettori delle LDL vengono sintetizzati nel reticolo endoplasmatico rugoso, successivamente trasferiti all’apparato di Golgi e poi trasportati nella membrana plasmatica. Essi non sono distribuiti in modo uniforme sulla membrana cellulare ma concentrati in zone dette fossette rivestite (invaginazioni della membrana contenenti grosse quantità di clatrina). Le LDL sono in grado di legare i loro recettori grazie alla presenza dell’apolipoproteina B100 nella struttura della lipoproteina stessa. Si forma così un complesso LDL-recettore che viene avvolto dalla clatrina (presente nelle fossette rivestite) e successivamente internalizzato nella cellula sotto forma di vescicola rivestita. All’interno della cellula un enzima depolarizza la clatrina causando la perdita del rivestimento della vescicola e quindi il complesso LDL-recettore si dissocia. I recettori liberi nel citoplasma ritornano nella membrana a livello delle fossette rivestite e sono così riciclati. Quello che rimane nella vescicola invece viene inglobato da lisosomi dove idrolasi lisosomiali idrolizzano le componenti rimaste, inclusa l’apolipoproteina B100, gli esteri del colesterolo e i trigliceridi. Le molecole ricavate possono essere così utilizzate dalla cellula per vari processi in base alle necessità metaboliche della cellula stessa. La sintesi dei recettori per le LDL da parte della cellula viene inibita (e di conseguenza anche l’internalizzazione di nuove LDL) quando la concentrazione del colesterolo cellulare è elevata. Si riduce così in modo fisiologico la capacità delle cellule di captare LDL dal sangue. Quando il contenuto delle LDL subisce processi di perossidazione, il metabolismo delle LDL viene compromesso con conseguenti accumuli patologici di colesterolo.

\section{Processo malato}

\subsection{Difetti}
Nelle persone affette da ipercolesterolemia i recettori per le LDL possono essere assenti, incapaci di legare LDL o distribuiti su tutta la superficie impedendo la successiva endocitosi mediata da clatrina nelle fossette rivestite.\\
Vari meccanismi agiscono come fattori predisponenti dell'ipercolesterolemia mediante \textbf{riduzione dell'azione di LDLr}:
\begin{itemize}
\item il difetto più frequente è il \textbf{calo del numero} di recettori funzionanti esposti. Nel quadro più comune, esso è dovuto al \textit{feedback negativo} sulla \textbf{sintesi di LDLr} dato dall'abbondante colesterolo già presente all'interno dell'epatocita.\\ L'\textbf{ipercolesterolemia familiare} è una patologia autosomica dominante dovuta a \textbf{mutazione non-senso} del gene per esso codificante. L'eterozigote, pur avendo un solo allele mutato, non produce quantità sufficiente di LDLr funzionante per catabolizzare il colesterolo in circolo.
\item in altri casi la patologia è dovuta a \textbf{mutazioni senso} che producono LDLr con \textbf{difetti nel legame} con le lipoproteine
\item talvolta il difetto risiede nel \textbf{meccanismo di trasporto} della glicoproteina che, seppur correttamente sintetizzata, non raggiunge la \textbf{membrana cellulare}
\item infine, mutazioni di LDLr possono compromettere la regione \textbf{C-terminale}, fondamentale per l'\textbf{internalizzazione del complesso LDL-recettore}
\end{itemize}

\section{Correlazione fra processi alterati e sintomi}
L'esito del difetto di LDLr è una significativa \textbf{difficoltà nella ricaptazione delle LDL} plasmatiche da parte degli epatociti.\\
Per questo motivo, tali lipoproteine cariche di colesterolo \textbf{restano nel sangue} invece che venire internalizzate e degradate. Ciò ha fondamentalmente due conseguenze sul metabolismo del colesterolo:
\begin{itemize}
\item \textbf{calo della degradazione}
\item \textbf{aumento della sintesi}
\end{itemize}
I due fenomeni concorrono ad \textbf{aumentare il livello di colesterolo plasmatico}.

\subsection{Epatocita}
La normale internalizzazione delle LDL porta ad aumento della quota di colesterolo intracellulare, che viene in gran parte \textbf{smaltito} grazie alla degradazione ad \textbf{acidi biliari} secreti nella \textbf{bile}.\\
L'accumulo di colesterolo nel citoplasma ha inoltre l'importante significato di \textbf{prevenire ulteriore sintesi del composto}, mediante inibizione dell'enzima regolatore \textbf{HMG-CoA reduttasi}.

\begin{center}
\begin{tikzpicture}[node distance = 2cm, auto, trim left = (1), trim right = (1)]

    \node (1) [block]{acetil-CoA + acetoacetil-CoA};
    \node (2) [block, below of=1]{HMG-CoA};
    \node (3) [block, below of=2]{acido mevalonico};
    \node (4) [block, below of=3]{colesterolo};
    \node (5) [below of=4]{LDL plasmatiche};
    \node (6) [right of=4, node distance = 5cm]{recettore LDL};

    \path [line] (1) -- (2) node (7) [midway, left, xshift = -2mm] {\textit{HMG-CoA sintasi}};
    \path [line] (2) -- (3) node (8) [midway, left, xshift = -2mm] {\textit{HMG-CoA reduttasi}};
    \path [dashed, thickline] (3) -- (4);
    \path [reverseline] (4) -- (5);
    \path [dashed, line] (4) --++ (-5,-0) |- (7) node [midway, left] {$\ominus$};
    \path [dashed, line] (4) --++ (-5,0) |- (8) node [midway, left] {$\ominus$};
    \path [dashed, line] (4) -- (6) node [very near end]{$\ominus$};

\end{tikzpicture}
\end{center}

L'inibizione è mediata dagli intermedi \textbf{ossisteroli}, che stimolano la \textbf{proteolisi} di HMG-CoA reduttasi e mantengono presso il reticolo endoplasmico il \textbf{fattore di trascrizione SREBP}.\\
Durante le \textbf{fasi iniziali} della patologia, fattori come l'elevato introito dietetico di steroli mediano \textbf{retroattivamente} la \textbf{riduzione del numero} di recettori per LDL.\\
Nell'ipercolesterolemia conclamata \textbf{l'assunzione del colesterolo plasmatico è fortemente compromessa} per il calo di LDLr, e quindi la concentrazione intracellulare rimane bassa.\\
Di conseguenza, manca l'inibizione di HMG-CoA reduttasi e di SREBP, il quale migra nel nucleo \textbf{attivando la trascrizione della reduttasi stessa} e di altri enzimi correlati alla sintesi di colesterolo.\\
In sintesi, dato il \textbf{mancato equilibrarsi del colesterolo intracellulare con quello plasmatico}, gli epatociti \textbf{continuano a sintetizzarlo} anche in presenza di eccesso nel sangue.


\subsection{Sangue}
Il risultato di un’assunzione difettosa delle LDL è un elevato livello di queste lipoproteine nel sangue (e del colesterolo che esse trasportano). La sintesi di colesterolo endogeno continua anche quando nell’organismo è presente un eccesso di colesterolo, in quanto il colesterolo extracellulare non può entrare nel citosol delle cellule e regolare la propria sintesi intracellulare.
L’accumulo patologico di colesterolo (placche), in particolare di LDL (colesterolo “cattivo”), può ostruire i vasi sanguigni, una condizione nota come aterosclerosi.  La formazione di placche nei vasi sanguigni ha inizio quando le LDL contenenti acidi grassi parzialmente ossidati aderiscono alla matrice extracellulare delle cellule endoteliali che rivestono le arterie e vi si accumulano. Le cellule del sistema immunitario (monociti) vengono attratte nelle regioni in cui sono presenti questi accumuli di LDL e si differenziano in macrofagi, che possono internalizzare le LDL e il colesterolo che esse contengono. Con l’aumento crescente di esteri del colesterolo e di colesterolo libero, queste cellule si trasformano in cellule schiumose (hanno un aspetto schiumoso quando sono osservate al microscopio). In seguito all’accumulo dell’eccesso di colesterolo libero nelle cellule schiumose e nelle loro membrane, queste vanno incontro ad apoptosi. In un lungo arco di tempo, le arterie tendono progressivamente ad occludersi, man mano che le placche costituite da materiale della matrice extracellulare, tessuto cicatriziale formatosi dalla muscolatura liscia e dalle rimanenze delle cellule schiumose diventano sempre più spesse. Inoltre si ha aggregazione piastrinica, con liberazione di PDGF che induce proliferazione
delle cellule muscolari lisce e rigonfamento della parete dei vasi.
Talvolta capita che una placca si stacchi dal sito della sua formazione e venga trasportata attraverso il torrente circolatorio in una regione più stretta di un’arteria localizzata nel cervello o nel cuore, causando un ictus o un attacco cardiaco.

\subsection{Altre sedi}
Oltre che nei vasi sanguigni, il colesterolo in eccesso può presentare altre sedi caratteristiche di accumulo. Si parla di:
\begin{itemize}
\item \textbf{xantoma}: alterazione cutanea dovuta a infiltrato di cellule istiocitarie cariche di sostanze lipidiche, come colesterolo e trigliceridi, nella cute, con formazione di chiazze piane o rilevate di un colorito giallastro caratteristico.
\item \textbf{xantelasmi}: accumuli di grasso che si formano sulle palpebre
\item \textbf{arco corneale}: piccola lunetta bianca, grigia o bluastra che si forma all’interno dell’occhio, alla periferia della cornea.
\end{itemize}

\section{Terapia}

\subsection{Dietetica}
È possibile ottenere un abbassamento della colesterolemia mediante \textbf{riduzione dell'apporto dietetico} del lipide.\\
L'introduzione giornaliera di \textbf{non più di 300 mg} di colesterolo, unita alla riduzione dell'introito calorico assoluto e della \textbf{quota relativa di lipidi} al \textbf{30\%}, consente di controllare parte delle problematiche ponderali e cardiocircolatorie connesse alla patologia.\\
Inoltre è consigliato che circa \textbf{due terzi} dell'apporto di grassi sia dato da lipidi \textbf{mono- o polinsaturi}.

\subsection{Farmacologica}

\textbf{Colestiramina} e \textbf{colestipolo} sono due farmaci che sequestrano i \textbf{sali biliari}, promuovendone l'escrezione e quindi la \textbf{risintesi epatica}. Si ottiene quindi un aumento di:
\begin{itemize}
\item \textbf{assunzione delle LDL} da parte del fegato
\item \textbf{escrezione fecale} di colesterolo
\end{itemize}
Le \textbf{statine} sono una classe di agenti \textbf{inibitori competitivi di HMG-CoA reduttasi}, grazie alla loro somiglianza al mevalonato, e quindi della sintesi endogena di colesterolo. Sono particolarmente efficaci, permettendo la riduzione fino a \textbf{40-50\%} di colesterolo LDL.\\

\begin{center}
\setatomsep{20pt}
\schemestart
\Statina
\Mevalonato
\schemestop
\end{center}

La bassa concentrazione di colesterolo generato nell'epatocita favorisce la \textbf{riespressione del recettore per le LDL}, e quindi l'aumento della loro ricaptazione che riduce la colesterolemia.\\
Nuove strategie sono mirate all'\textbf{inibizione dell'assunzione intestinale} del composto, mediante inibitori del trasportatore \textbf{NPC1L1}, e alla \textbf{rimozione della modulazione negativa di LDLr} attuata da \textbf{PCSK9}.\\
Per i casi di ipercolesterolemia a base genetica sono in studio \textbf{vettori del gene corretto del recettore} delle LDL.

\begin{thebibliography}{4}

\bibitem{lehninger}
  David L. Nelson, Michael M. Cox,
  \textit{Lehninger principles of biochemistry},
  Freeman, W. H. & Company,
  6\textsuperscript{th} edition,
  2012.
\bibitem{devlin}
  Thomas M. Devlin,
  \textit{Biochimica con aspetti chimico-farmaceutici},
  EdiSES,
  7\textsuperscript{a} edizione,
  2011.
\bibitem{Bertoni}
  Alessandra Bertoni,
  \textit{Corso di Biochimica II},
  CdLM in Medicina e Chirurgia,
  Università del Piemonte Orientale,
  Anno accademico 2017-2018.
\bibitem{Wikipedia}
  Wikipedia,
  \textit{Ipercolesterolemia},
  2018

\end{thebibliography}

\end{document}
