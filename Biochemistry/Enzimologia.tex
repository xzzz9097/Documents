
\documentclass[a4paper,12pt]{article}

\usepackage[italian]{babel}
\usepackage[version=3]{mhchem}
\usepackage{textgreek}
\usepackage{gensymb}

\date{}

\title{Enzimologia}

\begin{document}

\maketitle

\textbf{1. Qual \'e la funzione degli enzimi?}\\

Gli enzimi sono catalizzatori biologici di reazioni chimiche.
Molte reazioni necessarie alla sopravvivenza degli organismi viventi, pur essendo spontanee, avverrebbero soltanto in tempi eccessivamente lunghi, e non compatibili con le esigenze fisiologiche dell'organismo.\\
Gli enzimi permettono di aumentare, anche di svariati ordini di grandezza (da $10^5$ a $10^{17}$), la velocit\`a di svolgimento di processi catabolici e anabolici fondamentali, in quanto forniscono ai reagenti un ambiente adeguato e selettivo in cui interagire con maggior produttivit\`a.\\
In ogni caso, un enzima non pu\'o rendere spontanea una reazione che non lo sia gi\`a.\\

\textbf{2. Qual \'e la nomenclatura e funzione degli enzimi?}\\

Gli enzimi, generalmente caratterizzati dal suffisso ``-asi'', sono classificati in prima instanza in base al tipo di reazione che catalizzano. Distinguiamo:
\begin{enumerate}
\item \textbf{Ossidoreduttasi}: trasferimento di elettroni
\item \textbf{Transferasi}: trasferimento di gruppi
\item \textbf{Idrolasi}: trasferimento di gruppi a \ce{H2O}
\item \textbf{Liasi}: addizione di gruppi a doppi legami, o formazioni di doppi legami per rimozione di gruppi
\item \textbf{Isomerasi}: trasferimento di gruppi all'interno di una stessa molecola
\item \textbf{Ligasi}: formazione di legami \ce{C\bond{-}C}, \ce{C\bond{-}S}, \ce{C\bond{-}O} e \ce{C\bond{-}N} per reazioni di condenzasione accoppiate a idrolisi di ATP (processo attivo). Sono anche dette \textbf{sintetasi}
\end{enumerate}
La nomenclatura prevede sia un nome sistematico, indicante classe dell'enzima e substrati e gruppi su cui agisce, sia un codice a quattro cifre definito \textbf{EC} (Commissione degli Enzimi), il cui primo numero corrisponde alla classe e gli altri a sottoclasse e gruppi coinvolti nella catalisi. Spesso vi \'e poi anche un nome comune.\\
Esempio: 2.4.1.1 = 1,4-$\alpha$-D-glucano:fosfato $\alpha$-D-glicosiltransferasi = (Glicogeno) Fosforilasi\\

\textbf{3. Qual \'e il meccanismo d'azione di un enzima?}\\

Una reazione di trasformazione di reagenti in prodotti comporta una variazione di energia, da uno stato stabile di partenza a uno finale. Tale energia, definita \textit{$\Delta G\degree$} o variazione di energia libera standard, \'e correlata alla costante di equilibrio della reazione secondo la legge:
\begin{center}$\Delta G\degree = -RTlnK\textsubscript{e}$\end{center}
L'enzima non altera \textit{$\Delta G\degree$}, quindi non modifica in alcun modo la spontaneit\`a della reazione e la posizione del suo equilibrio.
\\Piuttosto, esso velocizza il raggiungimento di tale equilibrio.\\

Una generica reazione enzimatica pu\`o essere cos\`i riassunta:
\begin{center}\ce{E + S <=> ES <=> EP <=> E + P}\end{center}
dove E \'e l'enzima, S il substrato e P il prodotto di reazione.\\
Per passare dallo stato stabile iniziale a quello finale, la reazione deve obbligatoriamente attraverstare una condizione instabile e temporanea, detta \textbf{stato di transizione}, alla quale essa ha uguale probabilit\`a di procedere verso i reagenti o verso i prodotti.\\
La differenza di energia fra lo stato stabile di partenza e quello di transizione, \'e definita \textit{$\Delta G$\textsuperscript{$\ddagger$}} o \textbf{energia di attivazione}. Tanto \'e maggiore l'energia di attivazione, tanto pi\`u lenta la reazione.\\
\textbf{I catalizzatori intervengono abbassando l'energia di attivazione, quindi velocizzando il raggiungimento dell'equilibrio di reazione ma non la sua posizione}. I catalizzatori aumentano allo stesso modo anche la velocit\`a della reazione inversa, che per\`o genericamente non avviene in quanto termodinamicamente sfavorita.
L'aumento di velocit\`a \'e spiegato per una reazione di primo ordine dalle relazioni:
\begin{center}$V = k[S]$ con $k = \frac{\mathrm{k}T}{h}e^\frac{-\Delta G\textsuperscript{\ddagger}}{RT}$\end{center}
dove \textbf{k} \'e la costante di Boltzmann e \textit{h} quella di Planck. Al diminuire di \textit{$\Delta G$\textsuperscript{$\ddagger$}}, \textit{k} aumenta (esponenziale negativo) e quindi anche \textit{V}.\\

\textbf{4. Che cos'\'e il sito attivo?}\\

Il sito attivo \`e una specifica zona della struttura proteica dell'enzima presso cui avviene la reazione catalizzata. Il reagente, o substrato, si posiziona nel sito attivo interagendo con dei suoi opportuni residui amminoacidici.
L'interazione fra sito attivo dell'enzima e substrato porta all'instaurarsi di legami, e quindi ad una liberazione di \textbf{energia di legame}, o \textit{$\Delta G$\textsubscript{B}}.\\
\textbf{L'energia di legame enzima-substrato \'e la principale responsabile dell'abbassamento di energia di attivazione da parte degli enzimi.}
Essa \'e in particolare dovuta alla formazione di numerosi legami deboli fra substrato e sito attivo dell'enzima.\\
Il sito attivo non \'e specificatamente affine al substrato, perch\'e in caso contrario esso lo stabilizerebbe rendendo pi\`u difficile lo svolgimento della reazione, n\'e al prodotto, che altrimenti non si distaccherebbe dall'enzima.\\ La teoria della complementariet\`a fra enzima e substrato, secondo il \textbf{modello chiave-serratura}, è stata per questo superata da quella dell'\textbf{adattamento indotto}: il sito attivo non \'e rigido, ma si rimodella interagendo con il substrato, divenendo preferenzialmente affine allo stato di transizione. Ci\`o crea un ambiente ottimale per l'instaurazione di legami deboli, e quindi per la conversione dei substrati in prodotti.\\
Dato che un grande numero di possibili legami deboli libera una maggiore energia di legame, e dunque velocizza pi\`u efficamente la reazione, gli enzimi hanno notevoli dimensioni e il loro sito attivo rimuove l'acqua circostante il substrato concentrando le interazioni su di s\'e piuttosto che sul solvente.\\
I legami che il sito attivo pu\`o formare sono solitamente ottimizzati per un particolare substrato, grazie a una precisa selezione di amminoacidi carichi, idrofobici o polari che garantisce \textbf{specificit\`a}.\\

\textbf{5. Qual \'e l'effetto della concentrazione del substrato sull'attivit\`a enzimatica?}\\

La cinetica enzimatica \'e descritta dall'equazione di Michaelis-Menten:
\begin{center}$V\textsubscript{0} = \frac{V\textsubscript{max}[S]}{K\textsubscript{m} + [S]}$ con $V\textsubscript{max} = k\textsubscript{cat}[E\textsubscript{t}]$\end{center}\\
L'effetto della concentrazione di substrato sulla velocit\`a varia a seconda della concentrazione stessa. Infatti:
\begin{center}$V\textsubscript{0} = \frac{V\textsubscript{max}}{K\textsubscript{m}}[S]$ con $[S] << K\textsubscript{m}$ e $V\textsubscript{0} = V\textsubscript{max}$ con $[S] >> K\textsubscript{m}$\end{center}\\
Quindi in presenza di ridotte quantit\`a di substrato, la velocit\`a di reazione \'e direttamente proporzionale alla sua concentrazione, mentre in eccesso di substrato la reazione raggiunge una velocit\'a massima oltre la quale l'attivit\'a catalitica non aumenta ulteriormente.\\
Concettualmente ci\`o \'e prevedibile dal fatto che con poco substrato, vi sar\`a molto enzima libero \textit{E} disponibile a legarsi ad un'eventuale aggiunta, mentre con substrato in eccesso l'enzima sar\`a presente quasi totalmente in forma legata \textit{ES} e non potr\`a accogliere nuovo substrato prima di completare il processamento di quello legato. Tale condizione \'e definita \textbf{saturazione}.\\
Inoltre si ha:
\begin{center}$V\textsubscript{0} = \frac{1}{2}V\textsubscript{max}$ con $[S] = K\textsubscript{m}$\end{center}
Pertanto \textit{K\textsubscript{m}}, o \textbf{costante di Michaelis-Menten}, \'e la quantit\`a di substrato alla quale la velocit\`a di reazione \'e pari alla met\`a di quella massima.\\

\textbf{6. Qual \'e l'effetto della concentrazione dell'enzima?}\\

Riconsiderando le relazioni:
\begin{center}$V\textsubscript{0} = \frac{V\textsubscript{max}[S]}{K\textsubscript{m} + [S]}$ e $V\textsubscript{max} = k\textsubscript{cat}[E\textsubscript{t}]$, con $[E\textsubscript{t}] = [E\textsubscript{f}] + [ES]$\end{center}\\
si nota che la velocit\'a massima, e dunque anche quella di reazione che ad essa \'e direttamente proporzionale, aumentano all'aumentare della concentrazione totale di enzima \textit{[E\textsubscript{t}]}.\\

\textbf{7. Qual \'e l'effetto della concentrazione di cofattori?}\\

I cofattori sono composti chimici aggiuntivi necessari all'attivit\`a di alcuni enzimi. Si tratta prevalentemente di ioni inorganici come \ce{Fe^{2+}}, \ce{Mg^{2+}}, \ce{Mn^{2+}}, \ce{Cu^{2+}}, usati ad esempio come accettori di elettroni in reazioni di ossidoriduzione. Si parla invece di \textbf{cofattori} quando il composto addizionale \'e di natura organica e pi\'u complessa, agente come trasportatore di gruppi funzionali. Molti di essi derivano da vitamine.\\
Se il coenzima o ione si lega covalentemente al suo enzima, si parla di \textbf{gruppo prostetico}.\\
I cofattori, se richiesti da un enzima, sono fondamentali al corretto svolgimento della catalisi, infatti solo l'\textbf{oloenzima}, ovvero il complesso enzima-cofattori, \'e funzionalmente attivo, a differenza della pura parte proteica definita \textbf{apoenzima} o \textbf{apoproteina}.\\
Per questo motivo carenze di cofattori e coenzimi, dovute ad esempio a quadri di ipovitaminosi, alterano l'efficacia di reazioni fisiologiche catalizzate da enzimi.\\

\textbf{8. Qual \'e l'effetto della temperatura?}\\

In generale un aumento di temperatura velocizza le reazioni chimiche, poich\'e aumenta l'energia cinetica delle molecole dei reagenti, e dunque il numero di quelle che hanno energia sufficiente a raggiungere lo stato di transizione superando \textit{$\Delta G$\textsuperscript{$\ddagger$}}.\\
Tuttavia, data la natura proteica della maggior parte degli enzimi, aumenti o cali di temperatura eccessivi possono portare a loro denaturazione, e quindi perdita della catalisi.\\
Nei mammiferi, gli enzimi hanno un intervallo di temperatura ottimale fra 40\degree C e 45\degree C, e fra 0 e 40\degree C l'attivit\'a dell'enzima raddoppia ogni 10\degree C di incremento.\\

\textbf{9. Qual \'e l'effetto del pH?}\\

Analogamente alla temperatura, gli enzimi presentano un intervallo di pH ottimale, al di fuori del quale la loro attivit\'a viene compromessa.\\
Ci\`o \'e dovuto al fatto che, a seconda della loro p\textit{K\textsubscript{a}}, gli amminoacidi carichi che formano la catena peptidica dell'enzima possono presentarsi in diverso stato di ionizzazione: la carica diventa positiva a $pH < p\textit{K\textsubscript{a}}$, mentre \'e negativa a $pH > p\textit{K\textsubscript{a}}$.\\
La corretta carica dei residui amminoacidici \'e necessaria a realizzare le interazioni ioniche con lo specifico substrato, e quindi a permettere l'abbassamento dell'energia di attivazione per la catalisi.\\

\textbf{10. Cosa sono gli isoenzimi?}\\

Gli isoenzimi sono forme strutturali alternative di un enzima che catalizzano la stessa reazione, nonostante lievi differenze nella sequenza amminoacidica. Sono codificati da geni differenti.\\ Spesso si tratta di isoforme tessuto-specifiche di uno stesso enzima.\\

\textbf{11. Cosa sono gli enzimi allosterici?}\\

Gli enzimi allosterici sono degli enzimi, la cui attivit\`a pu\`o essere regolata dall'organismo grazie ad appositi \tevtbf{modulatori allosterici}, cio\'e piccoli metaboliti o cofattori che si legano ad esso non covalentemente.\\
Il legame dei modulatori allosterici induce nell'enzima un cambiamento conformazionale, che pu\`o portare alla sua stimolazione o inibizione.\\
Gli enzimi allosterici si dividono in:
\begin{itemize}
\item \textbf{Omotropici}: un substrato \'e il modulatore allosterico
\item \textbf{Eteretropici}: il modulatore non \'e un substrato
\end{itemize}
Il modulatore si lega pertanto ad uno specifico sito regolatore dell'enzima, che nel caso degli omotropici coincide con il sito attivo, e provoca cambiamenti conformazionali che aumentano o diminuiscono l'attivit\`a di altri siti attivi dell'enzima.\\
Dal punto di vista cinetico, gli enzimi allosterici non seguono la legge di Michaelis-Menten. Gli enzimi eterotropici presentano curve di saturazione sigmoidali, e non iperboliche, a causa dei fenomeni cooperativi positivi tra i siti attivi, analoghi a quelli fra le subunit\`a di emoglobina in seguito al legame con \ce{O\textsubscript{2}}. Pi\`u variabile \'e l'effetto delle interazioni allosteriche negli enzimi eterotropici.
Gli enzimi allosterici, cos\`i come altri enzimi regolati con modalit\`a differenti, sono tipicamente catalizzatori della prima di una serie di reazioni svolte in sequenza. In questo modo, una loro disattivazione comporta il mancato svolgimento dell'intera catena. In vari casi il modulatore allosterico \'e il prodotto dell'ultima reazione della catena, permettendo cos\`i la realizzazione di un \textbf{feedback negativo}: il prodotto finale inibisce la sua stessa sintesi interrompendo la sua catena di formazione.\\
Il feedback negativo \'e tipico delle vie metaboliche, che vengono perci\`o  avviate dall'organismo quando il loro prodotto finale manca, e sospese quando esso \'e presente in quantit\'a sufficiente.\\
La regolazione enzimatica permette uno sfruttamento ottimale delle risorse energetiche e metaboliche dell'organismo, risparmiandole quando opportuno, e il mantenimento di macromolecole complesse in stato ridotto evitando la loro spontanea degradazione a complessi pi\'u semplici.

\end{document}
