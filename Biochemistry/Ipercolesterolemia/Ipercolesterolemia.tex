\documentclass[a4paper, 12pt]{article}

\usepackage[italian]{babel}
\usepackage[version=3]{mhchem}
\usepackage{textgreek}
\usepackage{gensymb}
\usepackage{chemfig}
\usepackage{tikz}
\usepackage{fullpage}
\usetikzlibrary{shapes,arrows, positioning,calc}

\tikzstyle{block} = [rectangle, draw, text centered, minimum height=1.5em]
\tikzstyle{line} = [draw, -latex']
\tikzstyle{linewithouttip} = [draw, -]
\tikzstyle{reverseline} = [draw, latex'-]
\tikzstyle{thickline} = [draw, -latex', line width = 0.4mm]

\usepackage[utf8]{inputenc}

\makeatletter
\definearrow5{-n>}{
	\CF@arrow@shift@nodes{#3}%
	\expandafter\draw\expandafter[\CF@arrow@current@style,-CF@full](\CF@arrow@start@node)--(\CF@arrow@end@node)node[pos=.2](narrow@arctangent@i){} node[pos=.5](narrow@arctangent@ii){};%
	\edef\CF@tmp@str{[\ifx\@empty#1\@empty draw=none,\fi-CF@full]}%
	\expandafter\draw\CF@tmp@str (narrow@arctangent@i)%
		arc[radius=\CF@compound@sep*\CF@current@arrow@length*\ifx\@empty#4\@empty0.333\else#4\fi,start angle=\CF@arrow@current@angle+90,%
		delta angle=-\ifx\@empty#5\@empty60\else#5\fi]node(narrow@start){};
	\edef\CF@tmp@str{[\ifx\@empty#2\@empty draw=none,\fi-CF@full]}%
	\expandafter\draw\CF@tmp@str (narrow@arctangent@ii)%
		arc[radius=\CF@compound@sep*\CF@current@arrow@length*\ifx\@empty#4\@empty0.333\else#4\fi,start angle=\CF@arrow@current@angle+90,%
		delta angle=-\ifx\@empty#5\@empty60\else#5\fi]node(narrow@end){};
	\edef\CF@tmp@str{\if\string-\expandafter\@car\detokenize{#4.}\@nil+\else-\fi}%
	\CF@arrow@display@label{#1}{0}\CF@tmp@str{narrow@start}{#2}{1}\CF@tmp@str{narrow@end}%
}
\makeatother

\newcommand*\derivesubmol[4]{%
    \saveexpandmode\saveexploremode\expandarg\exploregroups
    \csname @\ifcat\relax\noexpand#2first\else second\fi oftwo\endcsname
        {\expandafter\StrSubstitute\@car#2\@nil}
        {\expandafter\StrSubstitute\csname CF@@#2\endcsname}
    {\@empty#3}{\@empty#4}[\temp@]%
    \csname @\ifcat\relax\noexpand#1first\else second\fi oftwo\endcsname
        {\expandafter\let\@car#1\@nil}
        {\expandafter\let\csname CF@@#1\endcsname}\temp@
    \restoreexpandmode\restoreexploremode
}

\setcrambond{2pt}{}{}

\definesubmol{adenina}{N*5([::-18]-*6(-N=-N=(-NH_2)-)=-N=-)}

\definesubmol{s1}{-[2]!{adenina}}
\definesubmol{s2}{-[6,.6]O-PO_3^{2-}}
\definesubmol{s3}{-[6,.6]OH}

\definesubmol{ribosiofosfatoadenina}{%
    -[::-90,2]%
        (%
            -[::115,1.176]O%
            -[::-50,1.176]%
        )%
    <[::45,0.8]%
        (%
          (!{s3})
          -[::45,,,,line width=2pt,shorten <=-.5pt,shorten >=-.5pt]%
              (!{s2})%
          >[::45,0.8]%
              (!{s1})%
         )%
}

\definesubmol{ribosioadenina}{%
    -[::-90,2]%
        (%
            -[::115,1.176]O%
            -[::-50,1.176]%
        )%
    <[::45,0.8]%
        (%
          (!{s3})
          -[::45,,,,line width=2pt,shorten <=-.5pt,shorten >=-.5pt]%
          >[::45,0.8]%
              (!{s1})%
         )%
}

\def \Pantotenato {
  \tiny
  \chemfig{HO-C(=[:90]O)-CH_2-CH_2-NH-C(=[:90]O)-CH(-[:270]OH)-C(-[:90]CH_3)(-[:270]CH_3)-CH_2-OH}
}

\def \Colesterolo {
  \chemfig{[::30]HO-*6(--*6(=--*6(-*5(---(-[::-36](-[::+60]) -[::-60]-[::-60]-[::+60]-[::-60](-[::-60])-[::+60])-)-(-[::+0])---)--)-(-[::+0])---)}
}

\def \Statina {
  \chemfig{[:-30]-[::60]-(-[::-60])(-[::+120]R_1)-[::+60](=[::+60]O)-[::-60]O-[::-60]*6(---=*6(-=--*6(-*6(--*6(-----)))-)--)}
}

\def \CoA {
\tiny
\schemestart
$\underbrace{\chemfig{HS-CH_2-CH_2-NH-}}_{\beta\textnormal{-mercapto-etilammina}}$
$\underbrace{\chemfig{C(=[:90]O)-CH_2-CH_2-NH-C(=[:90]O)-CH(-[:270]OH)-C(-[:90]CH_3)(-[:270]CH_3)-CH_2-O}}_{\textnormal{acido pantotenico}}$
$\underbrace{\chemfig{PO_2^{-}-O-PO_2^{-}-O-CH_2!{ribosiofosfatoadenina}}}_{\textnormal{3'-fosfoadenosina difosfato}}$
\schemestop
}

\def \ACP {
\tiny
\schemestart
$\underbrace{\chemfig{HS-CH_2-CH_2-NH-C(=[:90]O)-CH_2-CH_2-NH-C(=[:90]O)-CH(-[:270]OH)-C(-[:90]CH_3)(-[:270]CH_3)-CH_2-O-PO_2^{-}-}}_{\textnormal{4'-fosfopanteteina}}$
$\underbrace{\chemleft. \chemfig{O-CH_2-CH(-[:90]C(-[:90]...)=O)(-[:270]NH(-[:270]...))}\chemright\}}_{\textnormal{serina}}$
proteina
\schemestop
}

\date{}

\setatomsep{15pt}

\title{%
  Ipercolesterolemia \\
  \large Alterazioni del metabolismo del colesterolo
}

\begin{document}

\begin{titlepage}

\maketitle

\begin{center}{\setatomsep{20pt}\Colesterolo}\end{center}

\tableofcontents

\section{Processo malato}
L'\textit{ipercolesterolemia} è genericamente definita come un \textbf{eccesso di colesterolo} nel sangue, in particolare di quello trasportato dalle \textbf{lipoproteine a bassa densità (LDL)}.\\
Le cause principali di tale incremento sono:
\begin{itemize}
\item \textbf{aumento della produzione epatica di VLDL}, lipoproteine precursori delle LDL plasmatiche che trasportano i \textbf{trigliceridi}, anch'essi elevati, e il colesterolo endogeni \textbf{verso i tessuti}
\item \textbf{insufficiente ricaptazione delle LDL} per deficit di LDLr o delle apolipoproteine
\end{itemize}
La forma più frequente (85\% dei casi) di ipercolesterolemia è quella \textit{multifattoriale}, indotta dalla coesistenza di \textbf{fattori ambientali}, quali dieta iperlipidica e sedentarietà, e di \textbf{fattori genetici predisponenti} che riducono l'efficacia dei sistemi di \textbf{controllo metabolico dei lipidi}.\\
La forma \textit{familiare} è invece una patologia propriamente genetica, dovuta a difetti del \textbf{gene per LDLr}.
\subsection{Recettore per le LDL}
LDLr è una glicoproteina a singola catena dotata di:
\begin{itemize}
  \item una sola elica \textbf{transmembrana}
  \item un'estremità \textbf{C-terminale} citoplasmatica
  \item un'estremità \textbf{N-terminale} protesa nell'ambiente cellulare
\end{itemize}
L'estremità N-terminale presenta i siti di legame per \textbf{apoB-100} e \textbf{apoE-100}, le due apolipoproteine espresse dalle \textbf{LDL} e VLDL che ne mediano il riconoscimento e la \textbf{internalizzazione} da parte delle cellule epatiche e periferiche dotate del recettore.\\
In particolare, è il \textbf{fegato} a svolgere il catabolismo della maggior parte delle LDL circolanti (75\%).
\subsection{Difetti}
Vari meccanismi agiscono come fattori predisponenti dell'ipercolesterolemia mediante \textbf{riduzione dell'azione di LDLr}:
\begin{itemize}
\item il difetto più frequente è il \textbf{calo del numero} di recettori funzionanti esposti. Nel quadro più comune, esso è dovuto al \textit{feedback negativo} sulla \textbf{sintesi di LDLr} dato dall'abbondante colesterolo già presente all'interno dell'epatocita.\\ L'\textbf{ipercolesterolemia familiare} è una patologia autosomica dominante dovuta a \textbf{mutazione non-senso} del gene per esso codificante. L'eterozigote, pur avendo un solo allele mutato, non produce quantità sufficiente di LDLr funzionante per catabolizzare il colesterolo in circolo.
\item in altri casi la patologia è dovuta a \textbf{mutazioni senso} che producono LDLr con \textbf{difetti nel legame} con le lipoproteine
\item talvolta il difetto risiede nel \textbf{meccanismo di trasporto} della glicoproteina che, seppur correttamente sintetizzata, non raggiunge la \textbf{membrana cellulare}
\item infine, mutazioni di LDLr possono compromettere la regione \textbf{C-terminale}, fondamentale per l'\textbf{internalizzazione del complesso LDL-recettore}
\end{itemize}

\section{Alterazioni metaboliche}
L'esito del difetto di LDLr è una significativa \textbf{difficoltà nella ricaptazione delle LDL} plasmatiche da parte degli epatociti.\\
Per questo motivo, tali lipoproteine cariche di colesterolo \textbf{restano nel sangue} invece che venire internalizzate e degradate. Ciò ha fondamentalmente due conseguenze sul metabolismo del colesterolo:
\begin{itemize}
\item \textbf{calo della degradazione}
\item \textbf{aumento della sintesi}
\end{itemize}
I due fenomeni concorrono ad \textbf{aumentare il livello di colesterolo plasmatico}.

\subsection{Epatocita}
La normale internalizzazione delle LDL porta ad aumento della quota di colesterolo intracellulare, che viene in gran parte \textbf{smaltito} grazie alla degradazione ad \textbf{acidi biliari} secreti nella \textbf{bile}.\\
L'accumulo di colesterolo nel citoplasma ha inoltre l'importante significato di \textbf{prevenire ulteriore sintesi del composto}, mediante inibizione dell'enzima regolatore \textbf{HMG-CoA reduttasi}.

\begin{center}
\begin{tikzpicture}[node distance = 2cm, auto, trim left = (1), trim right = (1)]

    \node (1) [block]{acetil-CoA + acetoacetil-CoA};
    \node (2) [block, below of=1]{HMG-CoA};
    \node (3) [block, below of=2]{acido mevalonico};
    \node (4) [block, below of=3]{colesterolo};
    \node (5) [below of=4]{LDL plasmatiche};
    \node (6) [right of=4, node distance = 5cm]{recettore LDL};

    \path [line] (1) -- (2) node (7) [midway, left, xshift = -2mm] {\textit{HMG-CoA sintasi}};
    \path [line] (2) -- (3) node (8) [midway, left, xshift = -2mm] {\textit{HMG-CoA reduttasi}};
    \path [dashed, thickline] (3) -- (4);
    \path [reverseline] (4) -- (5);
    \path [dashed, line] (4) --++ (-5,-0) |- (7) node [midway, left] {$\ominus$};
    \path [dashed, line] (4) --++ (-5,0) |- (8) node [midway, left] {$\ominus$};
    \path [dashed, line] (4) -- (6) node [very near end]{$\ominus$};

\end{tikzpicture}
\end{center}

L'inibizione è mediata dagli intermedi \textbf{ossisteroli}, che stimolano la \textbf{proteolisi} di HMG-CoA reduttasi e mantengono presso il reticolo endoplasmico il \textbf{fattore di trascrizione SREBP}.\\
Durante le \textbf{fasi iniziali} della patologia, fattori come l'elevato introito dietetico di steroli mediano \textbf{retroattivamente} la \textbf{riduzione del numero} di recettori per LDL.\\
Nell'ipercolesterolemia conclamata \textbf{l'assunzione del colesterolo plasmatico è fortemente compromessa} per il calo di LDLr, e quindi la concentrazione intracellulare rimane bassa.\\
Di conseguenza, manca l'inibizione di HMG-CoA reduttasi e di SREBP, il quale migra nel nucleo \textbf{attivando la trascrizione della reduttasi stessa} e di altri enzimi correlati alla sintesi di colesterolo.\\
In sintesi, dato il \textbf{mancato equilibrarsi del colesterolo intracellulare con quello plasmatico}, gli epatociti \textbf{continuano a sintetizzarlo} anche in presenza di eccesso nel sangue.


\subsection{Sangue}
La diminuzione della frazione di colesterolo degradata, unita all'aumento della sua sintesi per \textbf{assenza di regolazione negativa}, causa l'\textit{ipercolesterolemia} propriamente detta.\\
Essa è genericamente definita come un \textbf{tasso di colesterolo plasmatico superiore a 240 mg/dL}, rispetto ad un valore consigliato inferiore a \textbf{200 mg/dL}.\\
L'eccesso di colesterolo ematico, trasportato dalle LDL, tende ad essere \textbf{ceduto a macrofagi} collocati nello \textbf{strato subendoteliale delle arterie}.\\
Il processo di internalizzazione ed esterificazione da parte dei macrofagi è efficace in quanto mediato anche da \textbf{vie LDLr-indipendenti}, e \textbf{non regolate negativamente} dal contenuto cellulare di colesterolo.\\
I macrofagi carichi di lipidi, denominati \textit{cellule schiumose}, possono provocare:
\begin{itemize}
\item \textbf{rigonfiamento} della parete dei vasi
\item \textbf{aggregazione piastrinica}, con liberazione di \textbf{PDGF} che induce proliferazione delle cellule muscolari lisce
\item \textbf{accumulo di lipidi} e \textbf{fibrosi} dei vasi conseguente a morte dei macrofagi
\end{itemize}
Tale quadro clinico, definito \textit{aterosclerosi}, è una condizione predisponente ad \textbf{infarti} ed \textbf{ischemie}, data la restrizione del lume vasale e la possibilità di distacco di \textbf{trombi} dalle placche.

\section{Terapia}

\subsection{Dietetica}
È possibile ottenere un abbassamento della colesterolemia mediante \textbf{riduzione dell'apporto dietetico} del lipide.\\
L'introduzione giornaliera di \textbf{non più di 300 mg} di colesterolo, unita alla riduzione dell'introito calorico assoluto e della \textbf{quota relativa di lipidi} al \textbf{30\%}, consente di controllare parte delle problematiche ponderali e cardiocircolatorie connesse alla patologia.\\
Inoltre è consigliato che circa \textbf{due terzi} dell'apporto di grassi sia dato da lipidi \textbf{mono- o polinsaturi}.

\subsection{Farmacologica}

\textbf{Colestiramina} e \textbf{colestipolo} sono due farmaci che sequestrano i \textbf{sali biliari}, promuovendone l'escrezione e quindi la \textbf{risintesi epatica}. Si ottiene quindi un aumento di:
\begin{itemize}
\item \textbf{assunzione delle LDL} da parte del fegato
\item \textbf{escrezione fecale} di colesterolo
\end{itemize}
Le \textbf{statine} sono una classe di agenti \textbf{inibitori competitivi di HMG-CoA reduttasi}, e quindi della sintesi endogena di colesterolo. Sono particolarmente efficaci, perettendo la riduzione fino a \textbf{40-50\%} di colesterolo LDL. La bassa concentrazione di colesterolo generato nell'epatocita favorisce la \textbf{riespressione del recettore per le LDL}, e quindi l'aumento della loro ricaptazione che riduce la colesterolemia.\\
Nuove strategie sono mirate all'\textbf{inibizione dell'assunzione intestinale} del composto, mediante inibitori del trasportatore \textbf{NPC1L1}, e alla \textbf{rimozione della modulazione negativa di LDLr} attuata da \textbf{PCSK9}.\\
Per i casi di ipercolesterolemia a base genetica sono in studio \textbf{vettori del gene corretto del recettore} delle LDL.

\begin{thebibliography}{2}

\bibitem{lehninger}
  David L. Nelson, Michael M. Cox,
  \textit{Lehninger principles of biochemistry},
  Freeman, W. H. & Company,
  6\textsuperscript{th} edition,
  2012.
\bibitem{devlin}
  Thomas M. Devlin,
  \textit{Biochimica con aspetti chimico-farmaceutici},
  EdiSES,
  7\textsuperscript{a} edizione,
  2011.
\bibitem{Bertoni}
  Alessandra Bertoni,
  \textit{Corso di Biochimica II},
  CdLM in Medicina e Chirurgia,
  Università del Piemonte Orientale,
  Anno accademico 2017-2018.

\end{thebibliography}

\end{document}
