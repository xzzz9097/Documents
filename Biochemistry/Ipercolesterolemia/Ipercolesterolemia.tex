\documentclass[a4paper, 12pt]{article}

\usepackage[italian]{babel}
\usepackage[version=3]{mhchem}
\usepackage{textgreek}
\usepackage{gensymb}
\usepackage{chemfig}
\usepackage{tikz}
\usepackage{fullpage}
\usetikzlibrary{shapes,arrows, positioning,calc}

\tikzstyle{block} = [rectangle, draw, text centered, minimum height=1.5em]
\tikzstyle{line} = [draw, -latex']
\tikzstyle{linewithouttip} = [draw, -]
\tikzstyle{reverseline} = [draw, latex'-]
\tikzstyle{thickline} = [draw, -latex', line width = 0.4mm]

\usepackage[utf8]{inputenc}

\makeatletter
\definearrow5{-n>}{
	\CF@arrow@shift@nodes{#3}%
	\expandafter\draw\expandafter[\CF@arrow@current@style,-CF@full](\CF@arrow@start@node)--(\CF@arrow@end@node)node[pos=.2](narrow@arctangent@i){} node[pos=.5](narrow@arctangent@ii){};%
	\edef\CF@tmp@str{[\ifx\@empty#1\@empty draw=none,\fi-CF@full]}%
	\expandafter\draw\CF@tmp@str (narrow@arctangent@i)%
		arc[radius=\CF@compound@sep*\CF@current@arrow@length*\ifx\@empty#4\@empty0.333\else#4\fi,start angle=\CF@arrow@current@angle+90,%
		delta angle=-\ifx\@empty#5\@empty60\else#5\fi]node(narrow@start){};
	\edef\CF@tmp@str{[\ifx\@empty#2\@empty draw=none,\fi-CF@full]}%
	\expandafter\draw\CF@tmp@str (narrow@arctangent@ii)%
		arc[radius=\CF@compound@sep*\CF@current@arrow@length*\ifx\@empty#4\@empty0.333\else#4\fi,start angle=\CF@arrow@current@angle+90,%
		delta angle=-\ifx\@empty#5\@empty60\else#5\fi]node(narrow@end){};
	\edef\CF@tmp@str{\if\string-\expandafter\@car\detokenize{#4.}\@nil+\else-\fi}%
	\CF@arrow@display@label{#1}{0}\CF@tmp@str{narrow@start}{#2}{1}\CF@tmp@str{narrow@end}%
}
\makeatother

\newcommand*\derivesubmol[4]{%
    \saveexpandmode\saveexploremode\expandarg\exploregroups
    \csname @\ifcat\relax\noexpand#2first\else second\fi oftwo\endcsname
        {\expandafter\StrSubstitute\@car#2\@nil}
        {\expandafter\StrSubstitute\csname CF@@#2\endcsname}
    {\@empty#3}{\@empty#4}[\temp@]%
    \csname @\ifcat\relax\noexpand#1first\else second\fi oftwo\endcsname
        {\expandafter\let\@car#1\@nil}
        {\expandafter\let\csname CF@@#1\endcsname}\temp@
    \restoreexpandmode\restoreexploremode
}

\setcrambond{2pt}{}{}

\definesubmol{adenina}{N*5([::-18]-*6(-N=-N=(-NH_2)-)=-N=-)}

\definesubmol{s1}{-[2]!{adenina}}
\definesubmol{s2}{-[6,.6]O-PO_3^{2-}}
\definesubmol{s3}{-[6,.6]OH}

\definesubmol{ribosiofosfatoadenina}{%
    -[::-90,2]%
        (%
            -[::115,1.176]O%
            -[::-50,1.176]%
        )%
    <[::45,0.8]%
        (%
          (!{s3})
          -[::45,,,,line width=2pt,shorten <=-.5pt,shorten >=-.5pt]%
              (!{s2})%
          >[::45,0.8]%
              (!{s1})%
         )%
}

\definesubmol{ribosioadenina}{%
    -[::-90,2]%
        (%
            -[::115,1.176]O%
            -[::-50,1.176]%
        )%
    <[::45,0.8]%
        (%
          (!{s3})
          -[::45,,,,line width=2pt,shorten <=-.5pt,shorten >=-.5pt]%
          >[::45,0.8]%
              (!{s1})%
         )%
}

\def \Pantotenato {
  \tiny
  \chemfig{HO-C(=[:90]O)-CH_2-CH_2-NH-C(=[:90]O)-CH(-[:270]OH)-C(-[:90]CH_3)(-[:270]CH_3)-CH_2-OH}
}

\def \Colesterolo {
  \chemfig{[::30]HO-*6(--*6(=--*6(-*5(---(-[::-36](-[::+60]) -[::-60]-[::-60]-[::+60]-[::-60](-[::-60])-[::+60])-)-(-[::+0])---)--)-(-[::+0])---)}
}

\def \CoA {
\tiny
\schemestart
$\underbrace{\chemfig{HS-CH_2-CH_2-NH-}}_{\beta\textnormal{-mercapto-etilammina}}$
$\underbrace{\chemfig{C(=[:90]O)-CH_2-CH_2-NH-C(=[:90]O)-CH(-[:270]OH)-C(-[:90]CH_3)(-[:270]CH_3)-CH_2-O}}_{\textnormal{acido pantotenico}}$
$\underbrace{\chemfig{PO_2^{-}-O-PO_2^{-}-O-CH_2!{ribosiofosfatoadenina}}}_{\textnormal{3'-fosfoadenosina difosfato}}$
\schemestop
}

\def \ACP {
\tiny
\schemestart
$\underbrace{\chemfig{HS-CH_2-CH_2-NH-C(=[:90]O)-CH_2-CH_2-NH-C(=[:90]O)-CH(-[:270]OH)-C(-[:90]CH_3)(-[:270]CH_3)-CH_2-O-PO_2^{-}-}}_{\textnormal{4'-fosfopanteteina}}$
$\underbrace{\chemleft. \chemfig{O-CH_2-CH(-[:90]C(-[:90]...)=O)(-[:270]NH(-[:270]...))}\chemright\}}_{\textnormal{serina}}$
proteina
\schemestop
}

\date{}

\setatomsep{15pt}

\title{%
  Ipercolesterolemia \\
  \large Alterazioni del metabolismo del colesterolo
}

\begin{document}

\begin{titlepage}

\maketitle

\begin{center}{\setatomsep{20pt}\Colesterolo}\end{center}

\tableofcontents

\section{Processo malato}
L'ipercolesterolemia è una grave patologia metabolica dovuta a difetti del recettore delle \textbf{lipoproteine a bassa densità (LDL)}.
\subsection{Recettore per le LDL}
LDLr è una glicoproteina a singola catena dotata di:
\begin{itemize}
  \item una sola elica \textbf{transmembrana}
  \item un'estremità \textbf{C-terminale} citoplasmatica
  \item un'estremità \textbf{N-terminale} protesa nell'ambiente cellulare
\end{itemize}
L'estremità N-terminale presenta i siti di legame per \textbf{apoB-100} e \textbf{apoE-100}, le due apolipoproteine espresse dalle \textbf{LDL} e VLDL che ne mediano il riconoscimento e la \textbf{internalizzazione} da parte delle cellule epatiche e periferiche dotate del recettore.\\
In particolare, è il \textbf{fegato} a svolgere il catabolismo della maggior parte delle LDL circolanti (75\%).
\subsection{Difetti}
La disfunzione che causa l'ipercolesterolemia può interessare vari aspetti della costituzione e attività di LDLr:
\begin{itemize}
\item il difetto più frequente è il \textbf{calo del numero} di recettori funzionanti, dovuto a una \textbf{mutazione non-senso} del gene per esso codificante. L'\textbf{ipercolesterolemia familiare} è una patologia autosomica dominante, in quanto l'eterozigote, pur avendo un solo allele mutato, non produce quantità sufficiente di LDLr funzionante per catabolizzare il colesterolo in circolo.
\item in altri casi la patologia è dovuta a \textbf{mutazioni senso} che producono LDLr con \textbf{difetti nel legame} con le lipoproteine
\item talvolta il difetto risiede nel \textbf{meccanismo di trasporto} della glicoproteina che, seppur correttamente sintetizzata, non raggiunge la \textbf{membrana cellulare}
\item infine, mutazioni di LDLr possono compromettere la regione \textbf{C-terminale}, fondamentale per l'\textbf{internalizzazione del complesso LDL-recettore}
\end{itemize}

\section{Alterazioni metaboliche}
L'esito del difetto di LDLr è una significativa \textbf{difficoltà nella ricaptazione delle LDL} plasmatiche da parte degli epatociti.\\
Per questo motivo, tali lipoproteine cariche di colesterolo restano nel sangue invece che venire internalizzate e degradate. Ciò ha fondamentalmente due conseguenze sul metabolismo del colesterolo:
\begin{itemize}
\item \textbf{calo della degradazione}
\item \textbf{aumento della sintesi}
\end{itemize}
I due fenomeni concorrono ad \textbf{aumentare il livello di colesterolo plasmatico}.

\subsection{Epatocita}
La normale internalizzazione delle LDL porta ad aumento della quota di colesterolo intracellulare, che viene in gran parte \textbf{smaltito} grazie alla degradazione ad \textbf{acidi biliari} secreti nella \textbf{bile}.\\
L'accumulo di colesterolo nel citoplasma ha inoltre l'importante significato di \textbf{prevenire ulteriore sintesi del composto}, mediante inibizione dell'enzima regolatore \textbf{HMG-CoA reduttasi}.

\begin{center}
\begin{tikzpicture}[node distance = 2cm, auto, trim left = (1), trim right = (1)]

    \node (1) [block]{acetil-CoA + acetoacetil-CoA};
    \node (2) [block, below of=1]{HMG-CoA};
    \node (3) [block, below of=2]{acido mevalonico};
    \node (4) [block, below of=3]{colesterolo};
    \node (5) [below of=4]{LDL plasmatiche};
    \node (6) [right of=4, node distance = 5cm]{recettore LDL};

    \path [line] (1) -- (2) node (7) [midway, left, xshift = -2mm] {\textit{HMG-CoA sintasi}};
    \path [line] (2) -- (3) node (8) [midway, left, xshift = -2mm] {\textit{HMG-CoA reduttasi}};
    \path [dashed, thickline] (3) -- (4);
    \path [reverseline] (4) -- (5);
    \path [dashed, line] (4) --++ (-5,-0) |- (7) node [midway, left] {$\ominus$};
    \path [dashed, line] (4) --++ (-5,0) |- (8) node [midway, left] {$\ominus$};
    \path [dashed, line] (4) -- (6) node [very near end]{$\ominus$};

\end{tikzpicture}
\end{center}

L'inibizione è mediata dagli intermedi \textbf{ossisteroli}, che stimolano la \textbf{proteolisi} di HMG-CoA reduttasi e mantengono presso il reticolo endoplasmico il \textbf{fattore di trascrizione SREBP}.\\
Nell'ipercolesterolemia \textbf{l'assunzione del colesterolo plasmatico non può avvenire}, e quindi la concentrazione intracellulare rimane bassa.\\
Di conseguenza, manca l'inibizione di HMG-CoA reduttasi e di SREBP, il quale migra nel nucleo \textbf{attivando la trascrizione della reduttasi stessa} e di altri enzimi correlati alla sintesi di colesterolo.\\
In sintesi, dato il \textbf{mancato equilibrarsi del colesterolo intracellulare con quello plasmatico}, gli epatociti \textbf{continuano a sintetizzarlo} anche in presenza di eccesso nel sangue.


\subsection{Sangue}

\section{Terapia}

\begin{thebibliography}{2}

\bibitem{lehninger}
  David L. Nelson, Michael M. Cox,
  \textit{Lehninger principles of biochemistry},
  Freeman, W. H. & Company,
  6\textsuperscript{th} edition,
  2012.
\bibitem{devlin}
  Thomas M. Devlin,
  \textit{Biochimica con aspetti chimico-farmaceutici},
  EdiSES,
  7\textsuperscript{a} edizione,
  2011.
\bibitem{Bertoni}
  Alessandra Bertoni,
  \textit{Corso di Biochimica II},
  CdLM in Medicina e Chirurgia,
  Università del Piemonte Orientale,
  Anno accademico 2017-2018.

\end{thebibliography}

\end{document}
