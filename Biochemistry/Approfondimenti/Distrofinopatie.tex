\documentclass[a4paper, 12pt]{article}

\usepackage[italian]{babel}
\usepackage{textgreek}
\usepackage{gensymb}
\usepackage{fullpage}
\usepackage[utf8]{inputenc}

\date{}

\title{%
  Distrofinopatie \\
  \large Basi molecolari
}

\begin{document}

\maketitle

\section{Introduzione}
La distrofina è una \textbf{proteina di adesione} coinvolta nella connessione del citoscheletro actinico alla matrice extracellulare.\\
L'adempimento di tale funzione è garantito dall'interazione con altre proteine nel complesso \textbf{DAPC}. La distrofina ha quindi un ruolo strutturale, permettendo il mantenimento del tono muscolare e la contrattilità.\\
La distrofina è codificata dal gene \textbf{DND}, il più lungo (2.4 Mb) del genoma umano, la cui trascrizione e splicing richiedono oltre 16 ore.

\section{Isoforme}
Tre varianti \textit{full-length} si trovano nel sarcolemma, nel cervelletto e nei neuroni di Purkinje.\\
Isoforme più brevi sono presenti in rene, polmone, cuore e fegato.\\
Quella più studiata è la \textbf{4}, una \textit{full-length} presente in sarcolemma, guinzioni neuro-muscolari e \textbf{costameri}. La sua corretta localizzazione richiede anchirina \textbf{ANK2} e \textbf{ANK3}.

\section{Struttura}
La distrofina ha 3 domini:
\begin{itemize}
\item \textbf{N-terminale}: deputato al legame con F-actina, grazie a regioni \textbf{CH} omologhe alla calponina, in singola copia o in tandem.
\item \textbf{Rod-domain}: contiene 24 ripetizioni di regioni \textbf{SPEC}, leganti spectrina, ciascuna spiraliforme con tre eliche; presenta inoltre un ulteriore sito di legame per F-actina e una regione \textbf{WW} (40 amminoacidi di triptofano) che lega proteine ricche in prolina.\\
È dotata infine di un motivo \textbf{zinc-finger} che grazie a due ioni zinco coordinati partecipa all'interazione con i ligandi o fa da scaffold.\\
Il \textit{rod-domain} presenta quindi siti di legame per molecole cariche, come i \textbf{distroglicani}, o idrofobe, come quelle di ancoraggio alla membrana.
\item \textbf{C-terminale}: presenta un sito ricco di cisteine per legare il \textbf{distroglicano} e la \textbf{sintrofina}.
\end{itemize}

\section{DAPC}
Il complesso delle proteine associate alla distrofina lega il citoscheletro actinico alla matrice extracellulare, stabilizzando il sarcolemma durante contrazione/rilassamento e trasmettendo la forza generata dai sarcomeri alla matrice.\\
Inoltre partecipa alla genesi del PdA e alla segnalazione interacendo con CaM, nNOS, Grb2 e canali per Na\textsuperscript{+}.

\subsection{Distroglicani}
In forma {\textalpha} e {\textbeta} partecipano al nucleo di DAPC. Sono derivati dal medesimo peptide e ampiamente glicosilati:
\begin{itemize}
  \item {\textalpha} è O-glicosilato in \textbf{serina}, è legato alla \textbf{laminina 2} della matrice e al sarcolemma grazie a {\textbeta}
  \item {\textbeta} è N-glicosilato in \textbf{asparagina}
\end{itemize}
Loro mutazioni causano \textbf{distroglicanopatie}.

\subsection{Sarcoglicani}
Sono 5 proteine TM, di cui {\textgamma} è legato alla distrofina e {\textdelta} ai distroglicani.\\
Interagiscono con la \textbf{distrobrevina}.

\subsection{Sintrofina}
Lega il C-ter di distrofina.\\
Possiede domini \textbf{PDZ}, grazie a cui ancora i recettori di membrana al citoscheletro e interagisce con i canali di Na\textsuperscript{+} (genesi di PdA) e le nNOS (aumento di flusso sanguigno).

\subsection{Altre}
La \textbf{distrobrevina} si lega a distrofina, sintrofina, sarcospano e sarcoglicani, stabilizzando DAPC.\\
La \textbf{sincoilina}, interagendo con la desmina, collega DAPC ai \textbf{filamenti intermedi}.\\
Il \textbf{sarcospano}, con 4 domini TM e C/N-ter intracellulari, stabilizza meccanicamente DAPC legandosi alla distrobrevina.

\section{Patologie}
Mutazioni a componenti di DAPC causano le \textbf{distrofie uscolari}.

\subsection{Di Duchenne e Becker}
Sono malattie neuromuscolari degenerative che portano a progressiva atrofia muscolare.\\
Sono causate da mutazioni recessive di DND sul cromosoma X, e si manifestano dunque prevalentemente nei maschi.\\
In DMD la mutazione è una frameshift con conseguente assenza totale di distrofina per NMD. Alcune terapie teorizzano l'uso di oligonucleotidi antisenso per riportare la mutazione di DMD in frame. I danni sono molto gravi e l'esordio precoce, con morte attorno a 20 anni.\\
In DMB la mutazione è in frame, quindi la distrofina è presente in minore quantità, ma non totalmente difettiva. Per questo la sintomatologia è più lieve e l'insorgenza più tardiva.\\
Dal punto di vista diagnostico queste patologie si caratterizzano con biopsie muscolari, diagnosi molecolari del genoma o dosaggi enzimatici del sangue (CPK elevata).

\end{document}
