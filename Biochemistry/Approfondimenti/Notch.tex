\documentclass[a4paper, 12pt]{article}

\usepackage[italian]{babel}
\usepackage{textgreek}
\usepackage{gensymb}
\usepackage{fullpage}
\usepackage[utf8]{inputenc}

\date{}

\title{%
  Via Notch \\
  \large Trasduzione del segnale e molecole correlate
}

\begin{document}

\maketitle

\section{Introduzione}
La segnalazione attraverso la proteina Notch è largamente usata nello \textbf{sviluppo} dei tessuti animali.\\
Essa possiede un ruolo di controllo nelle scelta del destino evolutivo cellulare e nella regolazione del pattern di formazione di molti tessuti, così come in quello del loro rinnovamento.

\section{Notch}
Notch è una proteina a singolo dominio transmembrana che richiede un processamento proteolitico per divenire funzionale.\\
Agisce in qualità di fattore di trascrizione latente e fornisce la più semplice e diretta via di segnalazione da un recettore di superficie al nucleo.\\
Quando attivata dal legame del ligando Delta di una cellula vicina, una proteasi di membrana taglia la coda citoplasmatica di Notch.\\
Quest'ultima, liberata, migra nel nucleo dove attiva la trascrizione di una serie di geni \textbf{Notch-responsivi} legandosi a una proteina legante DNA e convertendola da repressore ad attivatore.\\

\subsection{Attivazione}
Il recettore Notch subisce tre tagli proteolitici successivi, gli ultimi due dei quali dipendono dal legame di Delta. Il primo avviene invece normalmente nel Golgi e porta alla formazione di un eterodimero, trasportato in membrana in quanto recettore maturo.\\
Il legame di Delta induce un secondo taglio nel dominio extracellulare, mediato da una proteasi extracellulare; il taglio finale interessa appunto la coda citoplasmatica del recettore attivato.\\
A differenza di molti recettori, l'attivazione di Notch è irreversibile e la proteina attivata non può venire riciclata.

\subsection{Taglio finale}
Interessa la porzione più vicina al segmento transmembrana, ed è mediato da un complesso detto \textbf{\textgamma-secretasi}, responsabile anche del taglio di altre proteine intracellulari.\\
Una delle sue subinità è la \textit{presenilinina}, le mutazioni nel cui gene sono associate ad Alzheimer famigliare poichè i suoi prodotti di taglio si aggregano in placche amiloidi.

\section{Inibizione laterale}
Dagli studi su Drosophila, la via di Notch è nota per il coinvolgimento nella produzione delle cellule neuronali, che nell'insetto si formano come cellule singole isolate in foglietti epiteliali di precursori.\\
Durante tale processo, quando un precursore si impegna verso il destino neuronale, segnala ai suoi immediati vicini di non fare lo stesso. Questi ultimi divengono infatti cellule epidermiche.\\
Un'azione di questo tipo è denominata \textbf{inibizione laterale}, e dipende dalla proteina a singolo dominio transmembrana denominata \textbf{Delta}, espressa sulla superficie della futura cellula neuronale.\\
Il legame di Delta ai recettori Notch delle cellule vicine, trasmette il segnale di blocco dell'evoluzione neuronale.\\
Quando questo processo è difettoso, un eccesso di cellule nervose causa la letale mancanza di tessuto epidermico.

\section{Modificazioni dei componenti della via}
Notch e Delta sono glicoproteine, e la loro interazione è regolata dalla glicosilazione di Notch.\\
Le glicosil-transferasi della famiglia \textbf{Fringe} aggiungono ulteriori monosaccaridi agli oligosaccaridi O-legati, alterando la specificità di Notch per i suoi ligandi e fornendo una particolare via di modulazione dell'interazione ligando-recettore.

\end{document}
