\documentclass[a4paper, 12pt]{article}

\usepackage[italian]{babel}
\usepackage{textgreek}
\usepackage{gensymb}
\usepackage{fullpage}
\usepackage[utf8]{inputenc}

\date{}

\title{%
  Recettori tirosina fosfatasi \\
  \large Trasduzione del segnale
}

\begin{document}

\maketitle

\section{Introduzione}
La fosforilazione reversibile di alcuni residui amminoacidici è attuata dagli eucarioti e dai procarioti per regolare crescita, differenziamento, espressione genica, metabolismo e altri processi.\\
Le fosforilazioni sono applicate dalle \textbf{chinasi} e rimosse dalle \textbf{fosfatasi}.\\
Le \textbf{tirosina fosfatasi PTP} sono molto selettivi per tale amminoacido a causa della troppo limitata dimensione di serina e treonina, che non raggiungono la cisteina del sito attivo.

\section{Struttura}
Sono eterodimeri \textalpha\textbeta formati da 4 foglietti preceduti e seguiti da porzioni ad elica, che costituiiscono la tasca catalitica accogliente il substrato, grazie alla carica positiva che attira il fosfato.\\
La specificità di substrato è data dalla profondità di tale tasca.\\
Il sito catalitico comprende 200-300 aa. di cui il 20-30\% sono conservati. La sua azione non richiede la presenza di ioni metallici, e si esplica grazie a:
\begin{itemize}
  \item una sequenza \textbf{HCXXGXXRST}, la cui cisteina risulta indispensabile per la catalisi in quanto si comporta da \textbf{nucleofilo}
  \item un residuo di \textbf{aspartato} che si comporta da acido nel meccanismo catalitico
  \item un residuo di arginina rivolto verso la tasca, che assiste il legame del substrato e la catalisi
\end{itemize}
All'esterno del sito attivo le PTP posseggono domini \textbf{SH2}, \textbf{sequenze di target subcellulare} necessarie per la localizzazione e \textbf{ripetizioni Ig} che regolano il contatto e l'adesione cellula-cellula.\\
Altri domini hanno probabilmente funzione regolatoria.

\section{Classificazione}

\subsection{Classe 1}
Comprende 99 membri basati su cisteina divisi in:
\begin{itemize}
  \item \textbf{PTPs classiche}: 38, altamente tirosina-specifiche, si dividono in \begin{itemize}
    \item \textbf{RPTPs} simili a recettori, ovvero proteine integrali dipendenti da ligandi extracellulari e funzionanti da interfaccia esterno-interno
    \item \textbf{NRPTPs} non recettoriali, regolate da una sequenza fiancheggiante il dominio catalitico e localizzate nel citosol dove promouovono le cascate avviate da Ras
  \end{itemize}
  \item \textbf{VHI-like o DSP}: 61, con grande varietà di substrati, alcune specifiche per MAPK. Trasducono segnali mitogenici e legati al ciclo cellulare.
\end{itemize}

\subsection{Classe 2}
Anch'esse basate su cisteina catalitica, comprendono \textbf{LMPTP} con ampia varietà di substrati e correlata a varie patologie.

\subsection{Classe 3}
Sono \textbf{tirosina/treonina specifiche}, e le principali rappresentati sono le tre \textbf{p80\textsuperscript{CDC25}} regolatorie del ciclo cellulare, che defosforilano CDK guidando la progressione vitale della cellula.

\subsection{Classe 4}
Il loro meccanismo catalitico sfrutta \textbf{aspartato}, e ha attività \textbf{Ser/Tyr specifica}.\\
Controllano lo sviluppo di vari organi.

\section{Meccanismo catalitico}
Le prime tre famiglie sfruttano lo stesso meccanismo, conducente alla formazione dell'intermedio cisteinil-fosfato.\\
La tasca ha carica positiva e interagisce perciò con il fosfato negativo che si lega. Partecipa al complesso anche un atomo di \textbf{ossigeno} che si rapporta con specifici residui.\\
Il loop \textbf{WPD} si chiude dopo il legame del substrato, in modo da permettere la defosforilazione della tirosina.

\section{Esempi}
Le PTPs hanno ruolo fondamentale nel controllo di proliferazione, comunicazione, differenziazione, migrazione e adesione.

\subsection{CD45}
Regola la risposta immunitaria controllando l'attivazione dei TCR, la risposta delle citochine e la sopravvivenza dei linfociti.\\
È associato alla patogenesi di malattie autoimmuni e alle infezioni.

\subsection{LAR}
Coinvolta nelle adesioni focali, ha come substrato le \textbeta-catenine e p130cas.\\
La defosforilazione di tali target destabilizza la cellula e può spingerla verso la morte.\\
Inoltre regola la fosforilazione mediata dal recettore insulinico, e può quindi essere correlato al diabete da insulino-resistenza.

\end{document}
