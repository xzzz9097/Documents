\documentclass[a4paper, 12pt]{article}

\usepackage[italian]{babel}
\usepackage{textgreek}
\usepackage{gensymb}
\usepackage{fullpage}
\usepackage[utf8]{inputenc}

\date{}

\title{%
  Integrine \\
  \large Funzioni biologiche e trasduzione del segnale
}

\begin{document}

\maketitle

\section{Introduzione}
Le integrine sono proteine della membrana plasmatica coinvolte in funzione di \textbf{adesione cellula-cellula} e \textbf{cellula-matrice}, mediando la trasduzione di segnali bidirezionali.\\
Sono eterodimeri composti da subunità glicoproteiche {\textalpha}, di 120-170 kDa, e {\textbeta}, di 90-199 kDa.\\
Vi sono 8 diversi geni che codificano per 24 {\textalpha} e 9 {\textbeta} differenti.\\
{\textbeta} è ricca di cisteine e ponti disolfuro, mentre {\textalpha} lega cationi bivalenti indispensabili per l'interazione con i ligandi.\\
Sono presenti sono nei metazoi, e sono necessarie per l'adesione delle cellule alle matrici, per evoluzione e sviluppo e per il mantenimento delle lamine basali.

\section{Classificazione}
La specificità è data dal dominio extracellulare di {\textalpha} e {\textbeta}. Le 3 principali categorie di integrine legano:
\begin{itemize}
  \item collagene
  \item laminina
  \item fibronectina
\end{itemize}
Vi può essere ridondanza ed espressione differenziale nei tessuti.

\section{Struttura}
Le integrine comprendono:
\begin{itemize}
  \item un ampio \textbf{dominio extracellulare}, di 80-150 kDa, dato dalle estremità sporgenti di {\textalpha} e {\textbeta}. Esse formano una testa globulare N-ter che lega il ligando.\\
  {\textalpha} possiede un \textbf{\textalpha-I} (interazione con il ligando), un \textbf{\textbeta-propeller} (7 foglieti), una  \textbf{thigh} Ig-simile C-ter, e infine due \textbf{calf} a \textbeta-sandwich.\\
  {\textbeta} possiede un \textbf{\textbeta-I} (interazione con il ligando), un \textbf{dominio ibrido}, un \textbf{PSI} (pleckstrin-semaphor-integrin, importante per l'attivazione), quattro ripetizioni \textbf{EGF} e una coda beta-terminale.\\
  Presso le \textit{thigh} l'integrina si può flettere od estendere, per attivarsi o inibirsi. N-ter interagisce con la matrice.
  \item un \textbf{dominio transmembrana} a singola elica \textbf{coiled-coil}
  \item un \textbf{dominio intracellulare} di 10-70 residui, non strutturato.\\
  Le code di {\textbeta} sono corte, conservate, fosforilabili e reclutano le proteine di legame al citoscheletro, come la \textbf{talina}.
\end{itemize}

\section{Cambio conformazionale}
La conformazione ripiegata è inattiva, quella distesa è attiva ed espone il sito per il ligando extracellulare separando le due subunità, ovvero le \textit{gambe} nella zona di membrana.\\ L'attivazione è allostericamente regolata.\\
I cambiamenti conformazionali di un versante ne introducono altri su quello opposto:
\begin{itemize}
  \item attivazione \textbf{outside-in}, quando una proteina di ECM con sequenza \textbf{RGD} si lega al lato extracellulare. Ciò induce l'esposizione, su quello citoplasmatico, di siti di legame per proteine adattatrici, che collegano l'integrina all'actina citoscheletrica.
  \item attivazione \textbf{inside-out}, quando segnalatori intracellulari quali \textbf{PIP2} portano molecole quali la talina ad interagire con il versante interno dell'integrina.\\
  Il legame della talina alla coda {\textbeta} stimola il distacco di quest'ultima dalla {\textalpha}, e quindi l'attivazione.
\end{itemize}

\section{Interazione con i ligandi}
Le integrine riconoscono ligandi dotati della sequenza consenso \textbf{RGD}. Il legame:
\begin{itemize}
  \item richiede la presenza di \textbf{cationi bivalenti} legati da Glu e Asp
  \item è \textbf{specifico} grazie alla composizione della particolare integrina
  \item dipende dalla presenza di \textbf{elementi di riconoscimento} nei liganti
  \item richiede lo \textbf{switch conformazionale inside-out} dell'integrina alla forma attiva
\end{itemize}

\section{Funzioni}
Le integrine mediano il legame fra ECM e citoscheletro, permettendo \textbf{adesione}, e quindi forma, mobilità e polarità; \textbf{trasporto} e migrazione delle cellule fra i tessuti per rottura e ricostituzione dei legami con ECM, essenziale per diapedesi ed extravasione leucocitaria; \textbf{adesione extracellulare}.\\
Il legame integrina-actina, mediato da talina, actinina, filamina e vinculina, può creare \textbf{adesioni focali} di forte ancoraggio se coinvolge la GTPasi \textbf{Rho}.\\
Le integrine attivano infine vie di segnalazione associandosi a \textbf{chinasi} e altre proteine adattatrici, regolando espressione genica, crescita cellulare e differenziamento.

\end{document}
