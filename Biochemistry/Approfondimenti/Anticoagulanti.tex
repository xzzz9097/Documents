\documentclass[a4paper, 12pt]{article}

\usepackage[italian]{babel}
\usepackage{textgreek}
\usepackage{gensymb}
\usepackage{fullpage}
\usepackage[utf8]{inputenc}

\date{}

\title{%
  Anticoagulanti \\
  \large Modalità di azione
}

\begin{document}

\maketitle

\section{Introduzione}
Il controllo della coagulazione del sangue è un elemento clinico di notevole importanza, per il quale sono state elaborate numerose molecole agenti a più livelli del processo.\\
La cascata di coagulazione porta alla formazione di un \textbf{coagulo} atto ad interrompere o limitare una perdita emorragica.\\
I fattori della cascata sono proteasi che vengono attivate e attivano altri fattori per tagli proteolitici.\\
La funzionalità di alcuni di essi dipende dalla capacità di legare Ca\textsuperscript{2+}, permessa da modificazioni che richiedono la vitamina \textbf{K} come cofattore.\\
La coagulazione prevede:
\begin{itemize}
  \item una \textbf{via estrinseca}, mediata da un fattore tissutale rilasciato dall'endotelio lesionato, rapida ma poco amplificata
  \item una \textbf{via intrinseca}, che coinvolge fattori inattivi del plasma, lenta ma completa e molto amplificata
\end{itemize}
Entrambe portano all'attivazione di FX, che catalizza a sua volta la formazione della trombina e quindi della fibrina.

\section{Eparina}
Fa parte degli anticoagulanti ad assunzione parenterale.\\
È un \textbf{mucopolisaccaride}, presente in polmone, fegato ed intestino, formato da dimeri di \textbf{acido iduronico} e di \textbf{D-glucosamina}.\\
Interviene nell'emostasi inibendo la formazione di trombina, e abbassa i livelli di trigliceridi attivando le lipasi associate all'endotelio.\\
Come anticoagulante, l'eparina agisce legandosi alla serpina \textbf{AT3} su un residui lisinico, e inducendo in essa un cambio conformazionale che espone il sito attivo. AT3 inattiva la trombina e quindi la conversione del fibrinogeno in fibrina.\\
Tale effetto si riscontra con eparina ottadecasaccaridica; quella \textbf{pentasaccaridica} agisce invece con AT3 bloccando \textbf{FXa}.

\section{Warfarin}
Commercialmente noto come Coumadin, è un anticoagulante ad assunzione orale facente parte della famiglia delle \textbf{cumarine}.\\
Ha un ossidrile in posizione 4 e un carbonio chirale in 3; l'enantiomero levogiro è maggiormente attivo.\\
Esplica la sua azione impedento la \textgamma-carbossilazione dei residui di gluttammato dei fattori II, VII, IX e X, e precludendo loro quindi la facoltà di legarsi al calcio per interagire con i PL di membrana negativi.\\
La \textgamma-glutammil-carbossilasi richiede vitamina K ridotta come cofattore, ossidandola ad epossido nel corso della reazione.\\
K-epossido viene poi processata da una epossido reduttasi che ne ripristina la forma funzionale. Warfarin è un antagonista di K per la reduttasi, e quindi l'epossido non può più essere ridotto.\\
Il risultato finale è una diminuzione della carbossilazione, e quindi della funzionalità, di alcuni fattori chiave del processo di coagulazione.

\end{document}
