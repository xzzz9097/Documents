\documentclass[a4paper, 12pt]{article}

\usepackage[italian]{babel}
\usepackage{textgreek}
\usepackage{gensymb}
\usepackage{fullpage}
\usepackage[utf8]{inputenc}

\date{}

\title{%
  Matrice extracellulare \\
  \large Struttura e composizione
}

\begin{document}

\maketitle

\section{Introduzione}
L'ECM è una complessa rete di macromolecole che occupa lo spazio esterno alle cellule, formando una quota rilevante del volume dei tessuti.\\
Partecipa a morfogenesi, differenziazione ed omeostasi venendo continuamente rimodellata e garantendo forza tensile ed elasticità.
Si tratta di un \textbf{reticolo compatto} di proteine e polisaccaridi, in grado di stabilizzare la struttura fisica dei tessuti e di formare un punto di adesione, migrazione, comunicazione e proliferazione per le cellule.\\
Prevede:
\begin{itemize}
  \item una componente \textbf{fibro-proteica} con collagene, fibre elastiche e reticolari
  \item una componente \textbf{amorfa}, la \textbf{sostanza fondamentale}, con GAG, proteoglicani e glicoproteine multiadesive.
\end{itemize}

\section{Componente fibro-proteica}
\subsection{Collagene}
Il collagene forma il 25\% di ECM, è flessibile e resistente alla trazione.\\
Ha struttura omomerica o eteromerica a \textbf{tripla elica}, ricca di glicina, prolina, idrossilisina e idrossiprolina. Può essere:
\begin{itemize}
  \item \textbf{fibrillare}, più abbondante, come \textbf{I} (ossa, tendini, legamenti) o \textbf{II} (cartilagine)
  \item \textbf{associato a fibrille}, con interruzioni nella tripla elica che donano flessibilità, come \textbf{IX} e \textbf{XII}
  \item \textbf{laminare}, che si organizza in maglie reticolate, come \textbf{IV} della lamina basale
\end{itemize}

\subsection{Fibre reticolari}
Costituite da collagene \textbf{III}, si associano in fibre più sottili e ramificate.\\
Costituiscono una trama di supporto a magioa o rete, in particolare nel connettivo lasso o intorno ad adipociti, vasi sanguigni e cellule nervose.\\
Sono tipiche dei tessuti immaturi, venendo rimpiazzate dal tipo I. Rappresentano inoltre lo stroma degli organi emopoietici.

\subsection{Fibre elastiche}
Consentono ai tessuti di rispondere a stiramento e distensione.\\
Sono sottili e ramificate in una rete 3D, e sono formate da un core centrale di \textbf{elastina} e da una rete circostante di \textbf{fibrillina}.
\begin{itemize}
  \item \textbf{elastina}: ricca in prolina e glicina distribuita casaualmente, che rende la sostanza idrofobica e tendente ad aggregazione casuale in \textbf{random coil}. Contiene desmosina e isodesmosina che formano legami crociati.
  \item \textbf{emilina}: all'interfaccia elastina-fibrillina
  \item \textbf{fibrillina}: glicoproteina, substrato per l'elastogenesi
\end{itemize}

\section{Componente amorfa}

\subsection{Glicoproteine multiadesive}
Sono un piccolo ma importante gruppo di proteine, dotate di domini multipli che stabilizzano ECM ed intervengono nel suo legame alla superficie cellulare, in movimento, migrazione, proliferazione e differenziamento.

\subsubsection{Fibronectina}
Di 250-280 kDa, è la glicoproteina più abbondante nel connettivo. È un dimero con 2 subunità legate da ponte disolfuro.\\
Possiede domini atti al legame con eparansolfato, collagene, fibrina, acido ialuronico, altra fibronectina e con le \textbf{integrine} di membrana.\\ Il legame alle integrine induce la fibrillizzazione della fibronectina.

\subsubsection{Laminina}
È una glicoproteina adesiva di 140-400 kDa, abbondante nelle lamine basali, con tre grosse catene che formano una \textbf{croce} con un braccio lungo e tre corti.\\
Lega il collagene \textbf{IV}, eparansolfato, eparina e le integrine, formando una trama simile al feltro.\\
È fondamentale durante lo sviluppo embrionale e nervoso, per organizzare le cellule e indirizzarne la migrazione.

\subsubsection{Tenascina}
Presente solo in sviluppo, riparazione di ferite e tumori maligni, ed è in grado di legare le cellule alla matrice grazie ad appositi siti di legame.

\subsubsection{Osteopontina}
Un peptide glicosilato di 44 kDa caratteristico della matrice ossea.\\
Lega gli osteoclasti facendoli aderire alla superficie ossea, è coinvolta nel sequestro di calcio e promuove la calcificazione della matrice.

\subsubsection{Entactina/nidogeno}
Glicoproteina solforilata di 150 kDa che lega la laminina al collagene IV nella lamina basale.

\subsection{GAG}
Sono eteropolisaccaridi lineari con unità disaccaridiche di \textbf{GlcNAc} o \textbf{GalNAc} e \textbf{GlcA} o \textbf{IdoA}.\\
Sono fortemente negativi e per questo attraggono acqua formando un gel idratato, nel queale le molecole idrosolubili diffondono facilmente.

\subsection{Proteoglicani}
Formano il core proteico di complessi da cui si dipartono i GAG, che si legano mediante un trisaccaride (Gal-Gal-Xyl) O-glicosilato su residui di serina o treonina del core.\\
L'\textbf{aggrecano} si lega allo ialuronato e contiene 100-150 catene di cheratansolfato e condroitinsolfato, essendo responsabile dell'idratazione della cartilagine.
La \textbf{decorina} è una molecola formata da una sola catena di condroitinsolfato e dermatansolfato, presente in cartilagine e ossa.\\ Interagisce con TGF\textbeta.
Vi sono poi \textbf{versicano} e \textbf{sindecano}, in grado di legarsi a componenti di matrice e al citoscheletro.

\end{document}
