\documentclass[a4paper, 12pt]{article}

\usepackage[italian]{babel}
\usepackage{textgreek}
\usepackage{gensymb}
\usepackage{fullpage}
\usepackage[utf8]{inputenc}

\date{}

\title{%
  Recettori adrenergici \\
  \large Trasduzione del segnale
}

\begin{document}

\maketitle

\section{Introduzione}
Gli effetti di \textbf{adrenalina} e \textbf{noradrenalina} sono mediati da una specifica classe di recettori di membrana.\\
Si tratta di GPCR le cui proteine G accoppiate hanno come effettori \textbf{adenilato ciclasi} o \textbf{fosfolipasi C}.

\section{Classificazione}
Le famiglie note sono:
\begin{itemize}
  \item \textbf{\textalpha\textsubscript{1}}, con le sottofamiglie A, B e C
  \item \textbf{\textalpha\textsubscript{2}}, con le sottofamiglie A, B e C
  \item \textbf{\textbeta\textsubscript{1}}
  \item \textbf{\textbeta\textsubscript{2}}
  \item \textbf{\textbeta\textsubscript{3}}
\end{itemize}
\textalpha sono maggiormente affini a \textbf{noradrenalina}, \textbeta ad \textbf{adrenalina}.

\section{Struttura}
Sono costituiti da una singola catena polipeptidica di 400-500 residui, con tre loop intracellulari, tre extracellulari e 7TMS idrofobici.\\
Le regioni N- e C-ter variano per lunghezza e sequenza, conferendo specificità di ligando ed effettore.\\
I sette segmenti idrofobici formano inoltre una tasca che partecipa al legame con il ligando.\\
L'estremità N-ter contiene siti di glicosilazione, quella C-ter è fosforilabile da parte di PKA o altre chinasi.

\subsection{N-glicosilazione}
Interessa residui caratterizzati dalla sequenza consenso Asn-X-Ser presenti ad N-ter di tutti gli AR.\\
La mancata aggiunta del polisaccaride non parrebbe alterare il legame del ligando o la capacità trasduttiva, ma riduce la densità di espressione in membrana di alcune classi.

\subsection{Palmitoilazione}
Gli AR sono palmitoilati presso un residuo di \textbf{cisteina} posto immediatamente dopo il settimo dominio TMS.\\
Tale modificazione promuove l'interazione del complesso del recettore attivato con l'\textbf{adenilato ciclasi}, e la sua mancanza è associata a una maggiore fosforilazione.

\subsection{Formazione di ponti disolfuro}
Almeno un legame disolfuro è essenziale all'interazione con il ligando, che è infatti compromessa dalla sua riduzione.

\section{Legame del ligando e trasduzione}
Il ligando interagisce prevalentemente con i 7 TMS e solo in minima parte con le porzioni N-ter e C-ter.\\
Le regioni prossime al ligando possiedono residui coinvolti nell'associazione con esso e nell'interazione con la proteina G.

\subsection{Interazione con il ligando}
Fra i primi troviamo \textbf{Asp\textsuperscript{113}} nel terzo TMS, il cui carbossilato attrae il gruppo amminico della catecolammina. \\
Nell'interazione con essa sono coinvolti anche \textbf{Ser\textsuperscript{204}} e \textbf{Ser\textsuperscript{207}}, le cui posizioni sono conservate fra tutti gli AR.

\subsection{Trasduzione}
La trasduzione coinvolge \textbf{Asp\textsuperscript{79}}, \textbf{Tyr\textsuperscript{316}} \textbf{Asn\textsuperscript{312}}. L'attivazione della G\textsubscript{s} è probabilmente mediata dalla formazione di un legame H fra Tyr e Asn.\\
La proteina G accoppiata è trimerica e agisce in modalità analoga a quella associata agli altri GPCR. Molti AR interagiscono con \textbf{G\textsubscript{s}}; alcuni, come \textbf{\textalpha\textsubscript{1}} attivano invece \textbf{G\textsubscript{q}} e quindi una PLC, con produzione di IP3; altri infine, come gli \textbf{\textalpha\textsubscript{2}}, attivano \textbf{G\textsubscript{i}} bloccando AC.

\section{Regolazione}
La \textbf{fosforilazione} dei recettori attivati, in presenza di eccesso di agonista, porta ad una \textbf{desensibilizzazione} della via ed è mediata dalle \textbf{\textbeta-ARK} agenti su residui di serina e treonina posti sul C-ter di AR.\\
In alternativa AR può essere fosforilato da \textbf{RTK} presso un apposito sito C-terminale, portando a marcata desensibilizzazione del recettore.

\section{Effetti fisiologici}
La distribuzione delle varie classi di AR è diversa fra gli organi, ed è quindi diversa la risposta alle catecolammine.\\
L'esposizione prolungata a queste ultime porta a refrattarietà alle stesse per desensibilizzazione fosforilativa.\\
La risposta prepara a reazioni di \textbf{attacco-fuga}, aumentando vigilanza, vascolarizzazione, ossigenazione e glicemia.

\subsection{Sistemici}
Le catecolammine agiscono sul cuore aumentando la frequenza cardiaca, la velocità di conduzione e l'eccitabilità, conseguenti in aumento della gittata.\\
NA causa inoltre vasocostrizione generalizzata aumentando la pressione arteriosa e inducendo bradicardia; l'adrenalina provoca invece vasocostrizione periferica e vasodilatazione muscolare, coronarica ed epatica, senza alterare la pressione.\\
Determinano inoltre broncodilatazione, rilassamento GI con contrazione degli sfinteri e midriasi.\\
L'effetto complessivo è di tipo \textbf{antifatica} e potenzia la contrazione muscolare.

\subsection{Metabolici}
Le catecolammine hanno azione \textbf{iperglicemizzante}, attivando lipolisi, glicocenolisi e gluconeogenesi in fegato e muscolo e inibendo la captazione periferica del glucosio.\\
Aumentano inoltre il metabolismo basale con effetto calorigeno.

\end{document}
