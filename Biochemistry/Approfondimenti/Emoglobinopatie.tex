\documentclass[a4paper, 12pt]{article}

\usepackage[italian]{babel}
\usepackage{textgreek}
\usepackage{gensymb}
\usepackage{fullpage}
\usepackage[utf8]{inputenc}

\date{}

\title{%
  Emoglobinopatie \\
  \large Basi molecolari
}

\begin{document}

\maketitle

\section{Introduzione}
L'emoglobina può essere soggetta a disordini genetici che ne influenzino la sintesi, dal punto di vista quantitativo e qualitativo.\\
Quelle che interessano l'emoglobina sono le patologie monogeniche più comuni fra gli esseri umani, e sono diffuse soprattutto fra le popolazioni mediterranee, mediorentali, centroafricane e asiatiche.

\section{Struttura dell'emoglobina}
È un \textbf{tetramero} formato da
\begin{itemize}
  \item quattro catene \textbf{globiniche} uguali a due a due
  \item quattro gruppi prostetici, detti \textbf{eme}
\end{itemize}
La molecola è in grado di legare ossigeno con alta affinità nei polmoni, e di rilasciarlo nei tessuti a seconda delle loro esigenze.\\
L'affinità di Hb per O\textsubscript{2} è regolata da fattori quali temperatura, pH e concentrazione di 2,3 BPG.\\
L'eme è un complesso formato da un anello protoporfininico IX e da un atomo di Fe\textsuperscript{2+}, posizionato in una tasca idrofobica fra le eliche delle globine.

\section{Tipologie}
\begin{itemize}
  \item \textbf{HbA}: 96\% dell'emoglobina nell'adulto, ha struttura \textalpha2\textbeta2
  \item \textbf{HbA2}: 3\% dell'emoglobina nell'adulto, ha struttura \textalpha2\textdelta2
  \item \textbf{HbF}: <1\% dell'emoglobina nell'adulto, ha struttura \textalpha2\textgamma2 e forma il 60-90\% dell'emoglobina del feto
  \item \textbf{embrionali}
\end{itemize}
Nelle prime settimane di vita del feto, l'emopoiesi avviene nel saccp vitellino; dall'ottava settimana si svolge in fegato e milza, e dalla diciottesima passa a carico del midollo osseo.

\section{Patologie}
La notevole variabilità genica globinica si traduce in svariate patologie.

\subsection{Talassemie}
Si tratta di modificazioni di tipo quantitativo a carico delle globine, clinicamente rilevanti quando interessano le catene {\textalpha} o {\textbeta}.

\subsubsection{\textbeta-talassemie}
Le più diffuse in Italia, si trasmettono in via \textbf{autosomica recessiva}. In eterosigosi si riscontra la forma \textit{minor}, mentre in omozigosi si ha la \textit{maior} che può portare alla morte del bambino duranet l'infanzia.\\
I difetti più frequenti sono mutazioni puntiformi.\\
Forme intermedie possono portare a una sintesi ridotta, ma non assente, di \textbeta-globina, con anemia microcitica e ipocromica che porta a sintomi non gravi.\\
Nella variante \textit{maior}, la compensazione dell'anemia causa ipertrofia dei tessuti emopoietici con deformazioni scheletriche di cranio e ossa.\\
Si cura con trasfusioni o con il trapianto di midollo.

\subsubsection{\textalpha-talassemie}
Sono dovute a delezioni nei geni globinici.\\ Una delezione singola è frequente nei popoli africani e asintomatica; una doppia o tripla è invece diffusa in Asia e provoca minor volume globulare ma sintomi lievi.\\
Quando tutti e quattro i monomeri sono mutati si formano omotetrameri {\textgamma} che formano emoglobina \textbf{Bart}, incapace di trasportare ossigeno. Ciò causa anossia dei tessuti e scompensi cardiaci.

\subsection{Emoglobinopatie}
Sono causate da mutazioni missenso nei geni globinici, che donano ad Hb proprietà fisico-chimiche anomale.

\subsubsection{Anemia falciforme}
Detta anche \textbf{drepanocitica} è caratterizzata da globuli rossi a forma di falce. Si trasmette in via autosomica recessiva.\\
HbS tende a precipitare nei globuli, per sostituzione di un acido glutammico con una valina che abbassa la solubilità della proteina. HbS forma polimeri spiraliformi di 14 unità, che si aggregano ulteriormente in strutture più voluminose.
\\ I globuli falcemici aumentano la viscosità del sangue, deossigenano Hb e predispongono ad ostruzioni, infarti e problemi immunitari, oltre che emolisi ed anemia.\\
La terapia prevede splenectomia e continue trasfusioni, con eventuale ricorso a trapianti di midollo.

\subsection{Emoglobine C, D ed E}
Sono varianti diffuse in Medio Oriente e nel Sud-Est asiatico, caratterizzate dalla sostutuzione di residui delle catene {\textbeta}.\\
Le Hb mutate causano anemia emolitica, splenomegalie e microcitemia.

\subsection{Metaemoglibuna}
È causata da un eccesso di HbM non funzionale, dotata di Fe\textsuperscript{3+} a causa di mutazioni dei residui emici che coordinano lo ione.\\
I sintomi sono cianosi per colore blu del sangue, insufficiente ossigenazione di cuore e tessuti variabile a seconda del caso.\\
Si cura con ascorbato o blu di metilene, che riducono il ferro di HbM.

\subsection{Anemia microdrepanocitica}
Caratterizzata dalla compresenza del tratto falcemico e di quello anemico, per doppia mutazione ereditata dai genitori.\\ La sintomatologia è piuttosto severa.

\end{document}
