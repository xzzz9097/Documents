\documentclass[a4paper, 12pt]{article}

\usepackage[italian]{babel}
\usepackage{textgreek}
\usepackage{gensymb}
\usepackage{fullpage}
\usepackage[utf8]{inputenc}

\date{}

\title{%
  Misfolding \\
  \large Prioni e malattie correlate
}

\begin{document}

\maketitle

\section{Introduzione}
Le proteine prioniche sono in grado di agire come agenti infettivi in assenza di materiale genetico (DNA o RNA) che ne trasporti il messaggio.\\
Le patologie prioniche umane possono insorgere:
\begin{itemize}
\item spontaneamente
\item per trasmissione del gene mutato di una proteina prionica
\end{itemize}
I sintomi di tali malattie comprendono atassia, demenza e paralisi e il decorso è solitamente fatale.\\
I danni più gravi interessano l'encefalo, dove gli esami evidenziano placche amiloidi e degenerazione spongiforme.

\section{Struttura della proteina prionica}
Si tratta di una glicoproteina contenente carboidrati e fosfatidilinositolo, che la ancora al lato extracellulare della membrana citoplasmatica. Lì svolge funzioni recettoriali.
La proteina prionica può assumere due conformazioi:
\begin{itemize}
\item \textbf{PrP\textsuperscript{c}}, la variante normale ad alta solubilità. Contiene tre \textalpha-eliche e due piccoli segmenti a filamento \textbeta.
\item \textbf{PrP\textsuperscript{Sc}}, la variante mutata. È caratterizzata dalla conversione di una delle eliche in filamento \textbeta.
\end{itemize}

\section{Patogenesi}
La conformazione \textbf{PrP\textsuperscript{Sc}} spinge altri prioni a mutare verso questa struttura, e ad aggregare in amiloidi mediante interazioni fra i filamenti \textbeta.\\
L'assunzione della conformazione difettosa è irreversibile e gli aggregati fibrillari insolubili, simili fra le varie patologie, causano gravi danni nel sistema nervoso.\\
Il loro aumento è ad esempio correlato ad Alzheimer, Parkinson e SLA.\\
Sono stati teorizzati due meccanismi per la propagazione di \textbf{PrP\textsuperscript{Sc}}:
\begin{itemize}
\item \textbf{nucleazione-polimerizzazione}, secondo cui la forma mutata è normalmente in equilibrio con quella normale del prione, ma la polimerizzazione è lenta. PrP\textsuperscript{Sc} agisce iniziando il processo di nucleazione della polimerizzazioe, che continua poi attraverso i frammenti sintetizzati.
\item \textbf{stampo} la forma mutata agisce da stampo per la trasmissione della forma mutata alle proteine normali.
\end{itemize}

\section{Trasmissione}
Le patologie prioniche umane sono pleiotropiche, con fenotipo dipendente dalla causa di insorgenza e dai polimorfismi presenti nel gene prionico posseduto dall'individuo.\\
Le possibili vie di trasmissione sono:
\begin{itemize}
\item formazione sporadica e spontanea di \textbf{PrP\textsuperscript{Sc}}, come nella malattia di \textbf{Creutzfeldt-Jacob}
\item ingestione di \textbf{PrP\textsuperscript{Sc}} presente nelle carni di animali malati
\item difetto nel gene prionico, il cui prodotto assume preferenzialmente la conformazione \textbf{PrP\textsuperscript{Sc}}, come nella malattia di \textbf{Gerstmann} e nell'\textbf{insonnia fatale familiare}
\end{itemize}
CJD sporadica insorge attorno a 70 anni e ha decorso di circa sei mesi; una variante insorge invece a circa 30 anni e ha decorso di un anno.\\
Una possibile fonte di CJD è la via iatrogena, per somministrazione di sostanze contaminate provenienti da cadaveri o soggetti malati.

\section{Altre malattie da misfolding}
Le proteine misfolding non presentano la loro conformazione nativa e funzionale, e non sono state debitamente eliminate dai sistemi di controllo cellulare.\\
La conformazione mutata può causare perdita (LoF) o guadagno (GoF) di funzione, alterando i processi cellulari.

\subsection{BSE e CJD}
Già descritte, causate dall'attività di prioni mutati.

\subsection{Alzheimer}
Questa patologia è associata al \textbf{peptide \textbeta-amiloide A\textbeta}, di 40-42 amminoacidi e derivato dal precursore amiloide \textbf{APP} presente sulle membrane.\\
APP è processata da \textalpha- e \textgamma-secretasi in condizioni fisiologiche, ma quando intervengono le secretasi \textbeta e \textgamma si ha la formazione di A\textbeta.\\
Esso presenta una prevalenza di struttura secondaria \textbeta, e interagisce con altri A\textbeta nel formare aggregati fibrillari.\\
Nell'Alzheimer, la contemporanea presenza di aggregati A\textbeta sulle membrane cellulari e di \texttau all'interno delle cellule del tessuto nervoso, causa uno stato infiammatorio che danneggia gravemente e irreversibilmente l'encefalo.\\
In particolare, la degradazione inizia dalle aree ippocampali, coinvolte in memoria e apprendimento, e si propaga poi alle aree di pensiero e pianificazione del lobo frontale, e quindi a quelle adibite al linguaggio.\\
Il difetto principale dell'Alzheimer riguarda la via colinergica: difetti di colina-acetiltransferasi e di altri componenti del sistema di AcH porta a degenerazione dei neuroni interessati. La patologia ha prevalenza del 47\% oltre gli 85 anni.\\
Si ipotizza l'uso di inibitori di \textbeta- e \textgamma- secretsi per ridurre il processo di formazione delle fibrille.

\subsection{Parkinson}
\textalpha-sinucleina forma fibrille insolubili all'interno delle cellule, che formano i cosiddetti \textbf{corpi di Lewy}.

\subsection{Altre}
Sono dovute a misfolding la \textbf{fibrosi cistica} (CFTR), l'\textbf{amiloidosi}, la sindrome di Marfan (fibrillina del connettivo dei vasi), la malattia di Fabry, Gaucher e altre.

\end{document}
