\documentclass[a4paper, 12pt]{article}

\usepackage[italian]{babel}
\usepackage{textgreek}
\usepackage{gensymb}
\usepackage{fullpage}
\usepackage[utf8]{inputenc}

\date{}

\title{%
  Lipidosi \\
  \large Malattie genetiche da accumulo di lipidi
}

\begin{document}

\maketitle

\section{Introduzione}
Si tratta di patologie che portano ad un accumulo di metaboliti o sostanze all'interno delle cellule, con la conseguente compromissione della loro funzionalità.\\
Derivano da difetti ad enzimi coinvolti nel metabolismo dei lipidi, che ne riducono o inibiscono la funzionalità portando all'accumulo patologico di tali composti nei tessuti.\\
In particolare, interessano soprattutto gli \textbf{sfingolipidi}.
Dato che tali enzimi, e quindi la degradazione dei lipidi coinvolti in queste patologie, sono solitamente collocate nei lisosomi, si parla anche di \textbf{malattie da accumulo lisosomale}.

\section{Caratteristiche}
Sintomi comuni comprendono:
\begin{enumerate}
\item l'accumulo di un solo sfingolipide negli organi interessato
\item la porzione ceramidica dei lipidi coinvolti è la stessa
\item la normalità della velocità di biosintesi del lipide che si deposita
\item la mancanza di un enzima implicato nella via catabolica
\item la diffusione in tutti i tessuti del deficit enzimatico
\end{enumerate}

\section{Malattia di Tay-Sachs}
Il lipide accumulato è il \textbf{ganglioside G\textsubscript{M2}}, a causa di un difetto dell'enzima lisosomale \textbf{esosamminidasi A}. Essa si occupa normalmente del distacco di una N-acetilgalattosammina da GM2, generando GM3.\\
Le cellule gangliari della corteccia si rigonfiano, a causa dello riempimento dei lisosomi con il ganglioside. Ciò causa diminuzione di tali cellule, proliferazione della glia e demielinizzazioone dei nervi periferici.\\
I sintomi principali sono ritardo mentale, cecità, macchie rosso-ciliegia a livello della macula.\\
La morte sopraggiunge fra i 2 e 3 anni di vita.

\section{Malattia di Gaucher}
Il lipide accumulato è il \textbf{glucocerebroside}, a causa della ridotta funzionalità dell'enzima \textbf{glucocerebrosidasi}.\\
Le cellule subiscono ingrossamento per l'accumulo lisosomale, e inducono il rilascio di citochine con il conseguente stato infiammatorio negli organi colpiti.\\
I segni clinici comprendono ingrossamento di fegato e milza, erosione delle ossa lunghe e della pelvi, e ritardo mentale nella sola forma infantile.

\section{Malattia di Fabry}
La patologia insorge per accumulo di \textbf{ceramide triesoside}, per difetto della \textbf{\textalpha-galattosidasi A} codificata da un gene sul cromosoma X. Essa di occupa del distacco della porzione triesosica del lipide.\\
Causa rash cutaneo, insufficienza renale e dolore agli arti inferiore.

\section{Malattie di Niemann-Pick}
È caratterizzata dall'accumulo di \textbf{sfingomielina} per la ridotta attività dell'enzima \textbf{sfingomielinasi}.\\
Anche in questo caso si manifestano ingrossamento di fegato e milza e ritardo mentale.

\section{Diagnosi}
Si effettua prevalentemente mediante biopsia degli organi interessati, come fegato, midollo osseo ed encefalo, e viene ulteriormente confermata con saggi di attività enzimatica.\\
Possono essere utili campioni di leucociti, fibroblasti o villi coriali e, in alcuni casi, anche il siero e le lacrime.

\section{Trasmissione}
Le sfingolipidosi sono in gran parte malattie autosomiche recessivev, la cui manifestazione richiede quindi omozigosi del difetto. Due genitori eterozigoti avranno \/4 di probabilità di avere un figlio malato, e 1/2 di avere un portatore.

\section{Terapie}
Per la malattia di Gaucher e per quella di Fabry è disponibile un trattamento enzimatico sostitutivo.\\
È possibile attuare un percorso di prevenzione dei sintomi attraverso la consulenza genetica basata su saggi enzimatici e analisi del DNA.

\end{document}
