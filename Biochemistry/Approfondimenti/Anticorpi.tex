\documentclass[a4paper, 12pt]{article}

\usepackage[italian]{babel}
\usepackage{textgreek}
\usepackage{gensymb}
\usepackage{fullpage}
\usepackage[utf8]{inputenc}

\date{}

\title{%
  Anticorpi \\
  \large Monoclonali e policlonali
}

\begin{document}

\maketitle

\section{Introduzione}
La risposta immunitaria \textbf{umorale} umana è mediata dagli anticorpi, proteine prodotte dai linfociti B in grado di riconoscere uno specifico antigene.\\
L'antigene è una molecola riconosciuta come estranea dall'organismo, una cui specifica frazione, definita \textbf{determinante antigenico} o \textbf{epitopo}, è legata e riconosciuta da un opportuno anticorpo.

\section{Policlonali}
Un estratto di anticorpi policlonali è una miscela eterogenea di anticorpi prodotti da differenti plasmacellule contro un antigene, ciascuna in grado di produrre una specifica classe di immunoglobuline.\\
Queste ultime si legano perciò a epitopi diversi dello stesso antigene.\\
Gli anticorpi policlonali s itrovano nell'\textbf{antisiero}, ottenuto dal sangue di animali esposti a uno specifico antigene.

\subsection{Produzione}
\begin{enumerate}
\item \textbf{preparazione dell'antigene} in una soluzione priva di contaminanti
\item \textbf{selezione e preparazione dell'adiuvante} che coniugato all'antigene amplifica la risposta immunitaria specifica
\item \textbf{selezione dell'animale} in base all'anticorpo di interesse, alla quantità richiesta e all'organismo di destinazione
\item \textbf{prelievo del campione di siero di controllo}
\item \textbf{iniezione di antigene-adiuvante}
\item \textbf{estrazione del siero con gli anticorpi}
\item \textbf{confronto fra siero e controllo con ELISA}
\end{enumerate}

\subsection{Impieghi}
L'antisiero ad alta concentrazione e affinità viene impiegato prevalentemente per sperimentazione e test diagnostici.

\subsubsection{Eritroblastosi fetale}
Per il suo trattamento si iniettano anticorpi anti-D in una madre Rh- che abbia avuto un figlio Rh+. L'anticorpo lega i globuli rossi Rh+ fetali, evitando che si inneschi una risposta immunitaria che porti alla produzione di anticorpi anti-Rh+ che comprometterebbero un secondo feto Rh+.

\subsubsection{Epatite C}
HCV ha un ciclo di replicazione che coinvolge il citoplasma, possedendo proteine adibite a penetrazione, infezione, replicazione e maturazione. Gli anticorpi per il trattamento sono inibitori di tali proteine.\\
Gli inibitori della proteasi \textbf{NS5A} bloccano la replicazione e mediano altri effetti che aiutano a sopprimere la resistenza del virus.\\
Anti-HCV possono essere estratti dal fluido ascitico di pazienti affetti, e usati a fine diagnostico e immunoistochimico.

\section{Monoclonali}
Sono immunoglobuline \textbf{omogenee}, dirette contro un \textbf{unico epitopo} della molecola antigenica. Vengono prodotti selezionando un linfocita specifico, che produce anticorpi con le caratteristiche desiderate, fra quelli ottenuti in una normale risposta policlonale.

\subsection{Produzione}
\begin{enumerate}
\item \textbf{immunizzazione}, analoga a quella utilizzata per gli anticorpi policlonali
\item \textbf{selezione e prelievo della milza}, per l'elevato titolo anticorpale
\item \textbf{scelta del mieloma}, un tumore maligno del sangue che causa una proliferazione incontrollata di plasmacellule (difetto di HGPRT)
\item \textbf{fusione} dei linfociti selezionati con quelli maligni, con generazione di un \textbf{ibridoma} che contiene i geni di entrambi permettendo una produzione potenzialmente illimitata dell'anticorpo di interesse
\item \textbf{crescita selettiva dell'ibridoma} in un terreno di coltura individuale per ogni cellula, studiato per eliminare cellule non fuse e ibridi mieloma/mieloma o milza/milza
\item \textbf{screening dell'anticorpo} con \textbf{ELISA}, un dosaggio enzimo-immunologico, o \textbf{RIA}, radio-immunologico
\item \textbf{clonazione} precoce per eliminare ibridi non producenti anticorpi, selezionando quello più stabile
\item \textbf{caratterizzazione dell'anticorpo} mediante studi di specificità e purezza
\item \textbf{produzione dell'anticorpo} che può continuare indefinitamente in vitro. In vivo, oggi raro, gli anticorpi possono essere estratti dal fluido ascitico di topi in cui sono stati iniettati gli ibridomi. Gli anticorpi vengono \textbf{crioconservati}.
\end{enumerate}
In alternativa gli anticorpi monoclonali vengono prodotti con tecniche di \textbf{ingegneria genetica}, ad esempio qualora non sia disponibile un mieloma adatto. Essa prevede la sintesi di un DNA ricombinante codificante per una Ig formata dalla regione costante della Ig umana, e dalla regione variabile di quella di topo.\\
Ciò permette un minor rischio di reazione immunitaria nel paziente.\\
È possibile infine utilizzare topi transgenici in cui il gene per le Ig venga silenziato e sostiuito da omologhi codificanti per le varianti umane.

\subsection{Impieghi}
Trattamenti contro il cancro prevedono l'uso di anticorpi monoclonali che si leghino agli antigene tumore-specifici espresse dalle cellule cancerose.\\
Tali anticorpi possono essere coniugati a chemoterapici quali tossine, radionuclidi e citochine.
\subsubsection{Cetuximab}
Si tratta di un anticorpo monoclonale IgG1 chimerico uomo/topo ad ampia azione, utilizzato per il trattamento del carcinoma colon-rettale.\\
Viene prodotto in una linea cellulare di mammifero mediante DNA ricombinante ed è diretto contro \textbf{EGFr}.\\
Tale RTK è il mediatore di segnali extracellulari di crescita, differenziamento, sopravvivenza e angiogenesi, ed è frequentemente sovra-espresso o iperattivato nei tumori di origne epiteliale.\\
L'anticorpo compete con EGF, TGF\textalpha e neoregulina e induce l'internalizzazione e degradazione del recettore, e bloccando i segnali da esso trasdotti.\\
In sintesi, blocca l'attività riparatoria e angiogenica attuata delle cellule tumorali in risposta al trattamento chemo/radioterapico, sopprimento la crescita tumorale.

\end{document}
