\documentclass[a4paper, 12pt]{article}

\usepackage[italian]{babel}
\usepackage{textgreek}
\usepackage{gensymb}
\usepackage{fullpage}
\usepackage[utf8]{inputenc}

\date{}

\title{%
  Via Wnt \\
  \large Trasduzione del segnale e molecole correlate
}

\begin{document}

\maketitle

\section{Introduzione}
Le proteine Wnt sono molecole segnale che agiscono come \textbf{mediatori locali} e fattori di \textbf{morfogenesi} per controllare vari aspetti dello sviluppo dei tessuti animali.\\
Furono scoperti nelle mosche e nel topo, codificati rispettivamente dal gene \textit{Wingless} di Drosophila e da \text{Int1} dei topi.

\section{Struttura}
Le Wnt sono proteine che presentano una catena di acido grasso legata covalentemente a N-ter, che le attrae più fortemente alla superficie delle membrane cellulari.\\
L'uomo possiede 19 Wnt, con funzioni distinte ma spesso parzialmente sovrapponibili

\section{Vie attivate}
Le due principali vie hanno inizio con il legame di Wnt ai recettori \textbf{Frizzled}, dei 7TMS di superficie che ricordano i GPCR, ma agiscono reclutando proteine scaffold \textbf{Dishevelled} che mediano la trasduzione.

\subsection{Wnt/\textbeta-catenina}
È anche definita \textbf{via canonica}, e agisce sul regolatore di trascrizione \textbeta-catenina.\\
Una parte della \textbeta-catenina è localizzata presso le giunzioni cellula-cellula, dove controlla l'adesione cellulare, mentre la quota restante viene normalmente degradata nel citoplasma.\\
La degradazione è condotta da un complesso di grandi dimensioni, che lega la molecola e la mantiene lontana dal nucleo promuovendone il catabolismo. Tale complesso comprende anche:
\begin{itemize}
  \item la \textbf{casein chinasi CK1}, che fosforila la catenina in serina
  \item la \textbf{glicogeno sintasi chinasi GSK3}, che fosforila la catenina fosforilata marcandola per l'ubiquitinazione, e dunque per la degradazione proteasomica
  \item l'\textbf{assina} e \textbf{APC} che stabilizzano il complesso
\end{itemize}
In assenza di Wnt, la catenina viene degradata dal complesso, e i geni Wnt-responsivi da essa regolati sono mantenuti spenti dalle proteine \textbf{Groucho}.\\
Quando Wnt lega i Frizzled con la partecipazione del corecettore LRP, i due dimerizzano e subiscono fosforilazione da parte di GSK3 e CK1. Così, l'assina lega il complesso Frizzled-LRP fosforilato, mentre quello di degradazione non può formarsi e la catenina svolge la sua azione di FdT inibendo Groucho.

\subsection{Polarità planare}
Coordina la polarizzazione di cellule epiteliali e dipende dalle GTPasi della famiglia \textbf{Rho}.

\section{Geni stimolati}
Fra essi troviami \textbf{Myc}, che codifica un importante regolatore di crescita e proliferazione cellulare.\\
In conseguenza a mutazioni di APC, riscontrate ad esempio nell'adenoma colico, il complesso di degradazione non è funzionale e non metabolizza la catenina, che attiva costitutivamente la trascrizione di geni come Myc promuovendo la proliferazione incontrollata.

\section{Regolazione}
Proteine inibitorie secrete bloccano le vie di Wnt legandosi ai recettori LRP e promuovendo la loro down-regolazione, oppure competendo con i Frizzled per le Wnt secrete.\\
Inoltre Wnt attiva meccanismi di feedback negativo, per cui alcuni geni target codificano per proteine che spengono la via di segnalazione, ad esempio bloccando le Dishevelled.

\end{document}
